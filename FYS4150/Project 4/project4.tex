\documentclass[11pt,english,a4paper]{article}
\usepackage[latin1]{inputenc}
\usepackage[T1]{fontenc}
\usepackage[titles]{tocloft}
\usepackage[auth-sc]{authblk}
\usepackage[usenames]{color}
\usepackage[document]{ragged2e}
\usepackage{babel}
\usepackage[dvips]{epsfig}
\usepackage{graphicx}
\usepackage{amsmath}
\usepackage{amsfonts}
\usepackage{amssymb}
\usepackage{accents}
\usepackage{verbatim}
\usepackage[nodayofweek]{datetime}
\usepackage{epic,eepic}
\usepackage[mathscr]{euscript}
\usepackage{gastex}
\usepackage{supertabular}
\usepackage{nextpage}
\usepackage{listings}
\usepackage{color}
\usepackage{url}
\usepackage{ifthen,calc}
\usepackage{varioref, ifthen}
\usepackage{stmaryrd}
\usepackage{wasysym}
\usepackage{array}
\usepackage{booktabs}
\usepackage{gensymb}
\usepackage{ifsym}
\usepackage{trfsigns}
\usepackage{txfonts}
\usepackage{chemarrow}
\usepackage{textcomp}
\usepackage{cancel}
\usepackage{extarrows}
\usepackage{rotating}
\usepackage{varioref}
\usepackage{slashed}
\usepackage{setspace}
\PassOptionsToPackage{hyphens}{url}\usepackage[dvips,letterpaper,linktocpage,hidelinks=true,colorlinks=true]{hyperref}
\usepackage{multirow}
\usepackage{fancyhdr}
\usepackage{pstricks}
\usepackage{pst-node}
\usepackage{pst-blur}
\usepackage{xypic}
\input xy
\xyoption{all}

% Tekst kommandoer
\newcommand{\sitat}[1]{\textquotedblleft{#1}\textquotedblright} 
\newcommand{\anf}[1]{\textquotedbl{#1}\textquotedbl}

% Matematikk kommandoer
\newcommand{\ans}[1]{\underline{\underline{\,#1\,}}}
\newcommand{\p}[1]{\left({#1}\right)}
\newcommand{\kl}[1]{\left[{#1}\right]}
\newcommand{\bag}[1]{\Lbag{#1}\Rbag}
\newcommand{\Det}[1]{\left|{#1}\right|}
\newcommand{\Norm}[1]{\left\Vert{#1}\right\Vert}
\newcommand{\abs}[1]{\left\vert{\,#1\,}\right\vert}
\newcommand{\ceil}[1]{\left\lceil{\,#1\,}\right\rceil}
\newcommand{\floor}[1]{\left\lfloor{\,#1\,}\right\rfloor}
\newcommand{\kr}[1]{\left\lbrace{#1}\right\rbrace}
\let\vec\relax
\newcommand{\vec}[1]{\accentset{\rightarrow}{#1}}
\let\Vec\relax
\newcommand{\Vec}[2]{\accentset{#2\rightarrow}{#1}}
\newcommand{\T}[2]{\accentset{#2}{#1}}
\newcommand{\U}[2]{\underaccent{#2}{#1}}
\newcommand{\TU}[3]{\underaccent{#3}{\accentset{#2}{#1}}}
\newcommand{\grad}{\vec{\nabla}}
\newcommand{\Grad}[1]{\Vec{\nabla}{#1}}
\newcommand{\fvec}[1]{\vec{\mathrm{#1}}}
\newcommand{\vvec}[1]{\accentset{\rightrightarrows}{#1}}
\newcommand{\vek}[1]{\mathbf{#1}}
\newcommand{\Vek}[1]{\vec{\vek{#1}}}
\newcommand{\ivec}{\hat{\boldsymbol{\imath}}}
\newcommand{\jvec}{\hat{\boldsymbol{\jmath}}}
\newcommand{\kvec}{\hat{\boldsymbol{k}}}
\newcommand{\rhovec}{\hat{\boldsymbol{\rho}}}
\newcommand{\thetavec}{\hat{\boldsymbol{\theta}}}
\newcommand{\phivec}{\hat{\boldsymbol{\phi}}}
\newcommand{\unit}[1]{\,[\,\mathrm{#1}\,]}
\newcommand{\pow}[2]{{#1}^{#2}}
\newcommand{\ppow}[2]{\p{#1}^{#2}}
\newcommand{\abspow}[2]{\abs{#1}^{#2}}
\newcommand{\klpow}[2]{\pow{\kl{#1}}{#2}}
\newcommand{\krpow}[2]{\pow{\kr{#1}}{#2}}
\newcommand{\E}[1]{\times \pow{10}{#1}}
\newcommand{\Exp}[1]{\pow{\e}{#1}}
\newcommand{\snitt}[1]{\left\langle{\,#1\,}\right\rangle}
\newcommand{\D}[3]{\frac{\mathrm{d}^{#3}{#1}}{\mathrm{d}{#2}^{#3}}}
\let\P\relax
\newcommand{\P}[3]{\frac{\partial^{#3}{#1}}{\partial{#2}^{#3}}}
\newcommand{\I}[4]{\int_{\,#1}^{\,#2} {#3} \, \mathrm{d}{#4}}
\newcommand{\II}[4]{\int_{\,#1}^{\,#2} {#3} \, \boldsymbol{\mathrm{d}}{#4}}
\newcommand{\bfrac}[2]{{#1}\,/\,{#2}}
\newcommand{\mberegn}[1]{\qquad\p{\,#1\,}}
\newcommand{\bra}[1]{\langle\,#1\,\vert}
\newcommand{\ket}[1]{\,\vert\,#1\,\rangle}
\newcommand{\braket}[2]{\langle\,#1\,\vert\,#2\,\rangle}
\newcommand{\ketbra}[2]{\vert\,#1\,\rangle\,\langle\,#2\,\vert}
\newcommand{\Sim}[1]{\begin{matrix}\scriptsize{\text{#1}}\normalsize\\\sim\\\end{matrix}}
\newcommand{\norm}[1]{\Vert\,#1\,\Vert}
\newcommand{\Op}[1]{\hat{\mathrm{#1}}}
\newcommand{\OP}[1]{\hat{\mathbf{#1}}}
\newcommand{\set}[2]{\left. #1 \right\vert_{#2}}
\newcommand{\nlessdot}{\,\slashed{\lessdot}\,}
\newcommand{\verden}{\textbf{?}}
\newcommand{\fantasi}{\reflectbox{\verden}}
\newcommand{\hyperverden}{\reflectbox{\rotatebox[origin=lc]{180}{\verden}}}
\newcommand{\jeg}{\textcircled{\scriptsize{I}}}
\newcommand{\rightlefteqEA}{ \,\TU{\rightleftharpoons}{\hspace{-4pt}\exists}{\hspace{4pt}\forall}\,}
\newcommand{\nmapsto}{\displaystyle\mapsto\hspace{-11.5pt}\arrownot\hspace{10.5pt}}
\newcommand{\nmapsfrom}{\displaystyle\mapsfrom\hspace{-10pt}\arrownot\hspace{9pt}}
\newcommand{\maptofrom}{\displaystyle\leftrightarrow\hspace{-7pt}\mmapstochar\hspace{6pt}}
\newcommand{\nmaptofrom}{\displaystyle\leftrightarrow\hspace{-7pt}\mmapstochar\hspace{-5pt}\Arrownot\hspace{11pt}}
\newcommand{\mapstofrom}{\begin{array}{c}\displaystyle\mapsto\\[-9pt]\displaystyle\mapsfrom\end{array}}
\newcommand{\mapsfromto}{\begin{array}{c}\displaystyle\mapsfrom\\[-9pt]\displaystyle\mapsto\end{array}}
\declareslashed{}{$$ \mbox{\tiny $\boldsymbol{/}$ } $$}{0.01}{0}{\mapsto}
\declareslashed{}{$$ \mbox{\tiny $\boldsymbol{/}$ } $$}{0.13}{0}{\mapsfrom}
\newcommand{\nmapstofrom}{\begin{array}{c}\displaystyle\slashed\mapsto\\[-9pt]\displaystyle\slashed\mapsfrom\end{array}}
\newcommand{\rightleftarrowtail}{\begin{array}{c}\displaystyle\rightarrowtail\\[-9pt]\displaystyle\leftarrowtail\end{array}}
\newcommand{\leftrightarrowtail}{\begin{array}{c}\displaystyle\leftarrowtail\\[-9pt]\displaystyle\rightarrowtail\end{array}}
\newcommand{\rightleftsquigarrows}{\begin{array}{c}\displaystyle\rightsquigarrow\\[-9pt]\displaystyle\leftsquigarrow\end{array}}
\newcommand{\leftrightsquigarrows}{\begin{array}{c}\displaystyle\leftsquigarrow\\[-9pt]\displaystyle\rightsquigarrow\end{array}}
\newcommand{\leftrightdiamondarrow}{\displaystyle\leftrightarrow\hspace{-11.1pt}\diamond\hspace{5pt}}
\newcommand{\tworightarrowsTF}[2]{
	\mathrel{-\hspace{-0.3mm}\vcenter{
		\xymatrix@!=1pc@*[r]@C=0ex@R=2ex@M=0ex@W=0ex
		{
			& & #1 \\
			& \ar@{<->}'[l]^{\mathtt{F}}'[lu]'[lur]^{\mathtt{T}}  &	#2}}}}
%\newcommand{\paradoks}{\mathtt{P}}
%\newcommand{\toparadoks}{\xrightarrow{\paradoks}}
\newcommand{\Set}[1]{\boldsymbol{\mathrm{#1}}}
\newcommand{\eq}[1]{$\begin{matrix}{#1}\end{matrix}$}
\newcommand{\upp}{\hspace{0.2ex}{^{\wedge}}\hspace{0.2ex}}
\newcommand{\asinh}{\mathrm{asinh}}
\newcommand{\sgn}{\mathrm{sgn}}
\newcommand{\Forall}{\mathop{\vphantom{\sum}\mathchoice
  {\vcenter{\hbox{\huge $\forall$}}}
  {\vcenter{\hbox{\Large $\forall$}}}{\forall}{\forall}}\displaylimits}
\newcommand{\Recursion}{\mathop{\vphantom{\sum}\mathchoice
  {\vcenter{\hbox{\huge $\Omega$}}}
  {\vcenter{\hbox{\Large $\Omega$}}}{\Omega}{\Omega}}\displaylimits}
\newcommand{\SW}{\mathop{\vphantom{\sum}\mathchoice
  {\vcenter{\hbox{\huge $\Xi$}}}
  {\vcenter{\hbox{\Large $\Xi$}}}{\Xi}{\Xi}}\displaylimits}
\newcommand{\Rightleftarrowtail}[2]{\T{\T{\mathop{\rightleftarrowtail}}{\vspace{-5pt}#1}}{\vspace{-24pt}#2}}
\newcommand{\Leftrightarrowtail}[2]{\T{\T{\mathop{\leftrightarrowtail}}{\vspace{-5pt}#1}}{\vspace{-24pt}#2}}

\def\kvadratsum{\,\textcircled{$\mathsf{c}$}\,}

% Elektromagnetisme kommandoer
\def\ems{\xi}
\newcommand{\RaisingEdge}{\textifsym{L|H}}

% Astrofysikk kommandoer
\def\Earth{\varoplus}
\def\Sun{\astrosun}

% Kjemi
\newcommand{\nuklide}[2]{$^{#2}\mathrm{#1}$}
\newcommand{\molekyl}[1]{$\mathrm{#1}$}

\delimitershortfall=-1pt

\definecolor{light-gray}{gray}{0.95}
%---------------------------------------------------------------------------------------------------------
\renewcommand\maketitle{
\begin{center}
\LARGE\textsc{\title\ifx \rev \undefined  \else \, - \rev \fi} \normalsize \linebreak

\textsc{\author} \linebreak
\small
\ifx \adress \undefined \else \textsc{\adress} \linebreak \fi
\textsc{Email:} \texttt{\email}\linebreak
\textsc{\date} \linebreak
\rule{\textwidth}{2pt}
\end{center}
}
%---------------------------------------------------------------------------------------------------------
\newcounter{referanse}
\newcommand\reference[2]{\refstepcounter{referanse} \begin{tabular}{p{0.5cm}p{\textwidth - 0.5cm}}[\thereferanse] & {#1} \end{tabular}\label{#2}}
\newcommand\REF[1]{[\ref{#1}]}
%---------------------------------------------------------------------------------------------------------
%part

\makeatletter
\let\OLDpart = \part
\def\part{\@ifstar\unnumberedpart\numberedpart}
\def\numberedpart{\@ifnextchar[%]
  \numberedpartwithtwoarguments\numberedpartwithoneargument}
\def\unnumberedpart{\@ifnextchar[%]
  \unnumberedpartwithtwoarguments\unnumberedpartwithoneargument}
\def\numberedpartwithoneargument#1{\numberedpartwithtwoarguments[#1]{#1}}
\def\unnumberedpartwithoneargument#1{\unnumberedpartwithtwoarguments[#1]{#1}}
\def\numberedpartwithtwoarguments[#1]#2{%
  \ifhmode\par\fi
  \removelastskip
  \vskip 3ex\goodbreak
  \refstepcounter{part}%
  \noindent
  \begingroup
  \leavevmode\huge\bfseries\raggedright
  \thepart\quad 
  #2
  \par
  \endgroup
  \vskip 2ex\nobreak
  \setcounter{section}{0}
  \addcontentsline{toc}{part}{%
    \protect\numberline{\thepart}%
    #1}%
  }
\def\unnumberedpartwithtwoarguments[#1]#2{%
  \ifhmode\par\fi
  \removelastskip
  \vskip 3ex\goodbreak
    \refstepcounter{part}%
  \noindent
  \begingroup
  \leavevmode\huge\bfseries\raggedright
  \leavevmode\huge\bfseries\raggedright
  %\thepart\quad 
  #2
  \par
  \endgroup
  \vskip 2ex\nobreak
  \setcounter{section}{0}
  \addcontentsline{toc}{part}{%
    %\protect\numberline{\thepart}%
    #1}%
}
\makeatother

%---------------------------------------------------------------------------------------------------------

% section

\makeatletter
\let\OLDsection = \section
\def\section{\@ifstar\unnumberedsection\numberedsection}
\def\numberedsection{\@ifnextchar[%]
  \numberedsectionwithtwoarguments\numberedsectionwithoneargument}
\def\unnumberedsection{\@ifnextchar[%]
  \unnumberedsectionwithtwoarguments\unnumberedsectionwithoneargument}
\def\numberedsectionwithoneargument#1{\numberedsectionwithtwoarguments[#1]{#1}}
\def\unnumberedsectionwithoneargument#1{\unnumberedsectionwithtwoarguments[#1]{#1}}
\def\numberedsectionwithtwoarguments[#1]#2{%
  \ifhmode\par\fi
  \removelastskip
  \vskip 3ex\goodbreak
  \refstepcounter{section}%
  \noindent
  \begingroup
  \leavevmode\Large\bfseries\raggedright
  \thesection\quad 
  #2
  \par
  \endgroup
  \vskip 2ex\nobreak
  \addcontentsline{toc}{section}{%
    \protect\numberline{\thesection}%
    #1}%
  }
\def\unnumberedsectionwithtwoarguments[#1]#2{%
  \ifhmode\par\fi
  \removelastskip
  \vskip 3ex\goodbreak
    \refstepcounter{section}%
  \noindent
  \begingroup
  \leavevmode\Large\bfseries\raggedright
  \leavevmode\Large\bfseries\raggedright
  %\thesection\quad 
  #2
  \par
  \endgroup
  \vskip 2ex\nobreak
  \addcontentsline{toc}{section}{%
    %\protect\numberline{\thesection}%
    #1}%
}
\makeatother

%---------------------------------------------------------------------------------------------------------
% subsection

\makeatletter
\let\OLDsubsection = \subsection
\def\subsection{\@ifstar\unnumberedsubsection\numberedsubsection}
\def\numberedsubsection{\@ifnextchar[%]
  \numberedsubsectionwithtwoarguments\numberedsubsectionwithoneargument}
\def\unnumberedsubsection{\@ifnextchar[%]
  \unnumberedsubsectionwithtwoarguments\unnumberedsubsectionwithoneargument}
\def\numberedsubsectionwithoneargument#1{\numberedsubsectionwithtwoarguments[#1]{#1}}
\def\unnumberedsubsectionwithoneargument#1{\unnumberedsubsectionwithtwoarguments[#1]{#1}}
\def\numberedsubsectionwithtwoarguments[#1]#2{%
  \ifhmode\par\fi
  \removelastskip
  \vskip 3ex\goodbreak
  \refstepcounter{subsection}%
  \noindent
  \begingroup
  \leavevmode\large\bfseries\raggedright
  \thesubsection\quad 
  #2
  \par
  \endgroup
  \vskip 2ex\nobreak
  \addcontentsline{toc}{subsection}{%
    \protect\numberline{\thesubsection}%
    #1}%
  }
\def\unnumberedsubsectionwithtwoarguments[#1]#2{%
  \ifhmode\par\fi
  \removelastskip
  \vskip 3ex\goodbreak
    \refstepcounter{subsection}%
  \noindent
  \begingroup
  \leavevmode\large\bfseries\raggedright
  \leavevmode\large\bfseries\raggedright
  %\thesection\quad 
  #2
  \par
  \endgroup
  \vskip 2ex\nobreak
  \addcontentsline{toc}{subsection}{%
    %\protect\numberline{\thesection}%
    #1}%
}
\makeatother

% subsubsection

\makeatletter
\let\OLDsubsubsection = \subsubsection
\def\subsubsection{\@ifstar\unnumberedsubsubsection\numberedsubsubsection}
\def\numberedsubsubsection{\@ifnextchar[%]
  \numberedsubsubsectionwithtwoarguments\numberedsubsubsectionwithoneargument}
\def\unnumberedsubsubsection{\@ifnextchar[%]
  \unnumberedsubsubsectionwithtwoarguments\unnumberedsubsubsectionwithoneargument}
\def\numberedsubsubsectionwithoneargument#1{\numberedsubsubsectionwithtwoarguments[#1]{#1}}
\def\unnumberedsubsubsectionwithoneargument#1{\unnumberedsubsubsectionwithtwoarguments[#1]{#1}}
\def\numberedsubsubsectionwithtwoarguments[#1]#2{%
  \ifhmode\par\fi
  \removelastskip
  \vskip 3ex\goodbreak
  \refstepcounter{subsubsection}%
  \noindent
  \begingroup
  \leavevmode\normalsize\bfseries\raggedright
  \thesubsubsection\quad 
  #2
  \par
  \endgroup
  \vskip 2ex\nobreak
  \addcontentsline{toc}{subsubsection}{%
    \protect\numberline{\thesubsubsection}%
    #1}%
  }
\def\unnumberedsubsubsectionwithtwoarguments[#1]#2{%
  \ifhmode\par\fi
  \removelastskip
  \vskip 3ex\goodbreak
    \refstepcounter{subsubsection}%
  \noindent
  \begingroup
  \leavevmode\normalsize\bfseries\raggedright
  \leavevmode\normalsize\bfseries\raggedright
  %\thesection\quad 
  #2
  \par
  \endgroup
  \vskip 2ex\nobreak
  \addcontentsline{toc}{subsubsection}{%
    %\protect\numberline{\thesection}%
    #1}%
}
\makeatother

% subsubsubsection
\makeatletter
\newcounter{subsubsubsection}[subsubsection]
\def\subsubsubsectionmark#1{}
\def\thesubsubsubsection {\thesubsubsection.\arabic{subsubsubsection}}
\def\subsubsubsection{\@startsection{subsubsubsection}{4}{\z@} {-3.25ex plus -1ex minus -.2ex}{1.5ex plus .2ex}{\normalsize\bf}}
\def\l@subsubsubsection{\@dottedtocline{4}{4.8em}{4.2em}}
\makeatother



% section depth
\setcounter{secnumdepth}{6}
\setcounter{tocdepth}{6}

%---------------------------------------------------------------------------------------------------------
% Appendix

\newcounter{appendix}[part]
\renewcommand \theappendix{\Alph{appendix}}
\newcounter{subappendix}[appendix]
\renewcommand \thesubappendix{\theappendix.\arabic{subappendix}}
\newcounter{subsubappendix}[subappendix]
\renewcommand \thesubsubappendix{\thesubappendix.\arabic{subsubappendix}}

\newcommand{\Appendix}[1]
{
	\ifthenelse{\value{appendix} = 0}
	{
		\renewcommand \thesection{\theappendix}
		\renewcommand \thesubsection{\theappendix.\arabic{subsection}}
		\renewcommand \thesecsetning{\rev.\theappendix.\arabic{secsetning}}
		\renewcommand \thesubsetning{\rev.\thesubappendix.\arabic{subsetning}}		
		\renewcommand \thesubsubsetning{\rev.\thesubsubappendix.\arabic{subsubsetning}}
		\renewcommand \thekode{\rev.\theappendix.\arabic{seckode}}
		\renewcommand \thesubkode{\rev.\thesubappendix.\arabic{subkode}}		
		\renewcommand \thesubsubkode{\rev.\thesubsubappendix.\arabic{subsubkode}}
	}{}
	\refstepcounter{appendix}
	\section*{Appendix \theappendix:\quad #1}
}
\newcommand{\subAppendix}[1]{\refstepcounter{subappendix}\subsection*{\thesubappendix\quad #1}}
\newcommand{\subsubAppendix}[1]{\refstepcounter{subsubappendix}\subsubsection*{\thesubsubappendix\quad #1}}
\newcommand\refapp[1]{Appendix: \ref{#1}}
%---------------------------------------------------------------------------------------------------------

% Definisjon/Setning boks

\newcommand{\mengde}[2]{\mathbb{#1}_{\text{\ref{#2}}}}
\newcommand{\hypermengde}[2]{\boldsymbol{\mathrm{#1}}_{\text{\ref{#2}}}}
\newcommand{\funk}[2]{\mathcal{#1}_{\text{\ref{#2}}}}
\newcommand{\refS}[2]{$\mathcal{#1}$:\ref{#2}}
\newcommand{\refT}[1]{$\lbrace\ref{#1}\rbrace$}
\newcommand{\refpos}[1]{$\mathcal{P}$:\ref{#1}}
\newcommand{\mathpos}[1]{\mathcal{P}_{\text{\ref{#1}}}}
\newcommand{\refspes}[1]{$\mathcal{S}$:\ref{#1}}
\newcommand{\mathspes}[1]{\mathcal{S}_{\text{\ref{#1}}}}

\newcounter{setning}[part]
\renewcommand \thesetning{\ifx \rev \undefined \arabic{setning} \else \rev.\arabic{setning} \fi }
\newcounter{secsetning}[section]
\renewcommand \thesecsetning{\ifx \rev \undefined \thesection.\arabic{secsetning} \else \rev.\thesection.\arabic{secsetning} \fi }
\newcounter{subsetning}[subsection]
\renewcommand \thesubsetning{\ifx \rev \undefined \thesubsection.\arabic{subsetning} \else \rev.\thesubsection.\arabic{subsetning} \fi }
\newcounter{subsubsetning}[subsubsection]
\renewcommand \thesubsubsetning{\ifx \rev \undefined \thesubsubsection.\arabic{subsubsetning} \else \rev.\thesubsubsection.\arabic{subsubsetning} \fi }
\newcounter{partsetning}[part]
\renewcommand \thepartsetning{\ifx \rev \undefined \thepart.\arabic{partsetning} \else \rev.\thepart.\arabic{partsetning} \fi }
\newcounter{partsecsetning}[section]
\renewcommand \thepartsecsetning{\ifx \rev \undefined \thepart.\thesection.\arabic{partsecsetning} \else \rev.\thepart.\thesection.\arabic{partsecsetning} \fi }
\newcounter{partsubsetning}[subsection]
\renewcommand \thepartsubsetning{\ifx \rev \undefined \thepart.\thesubsection.\arabic{partsubsetning} \else \rev.\thepart.\thesubsection.\arabic{partsubsetning} \fi }
\newcounter{partsubsubsetning}[subsubsection]
\renewcommand \thepartsubsubsetning{\ifx \rev \undefined \thepart.\thesubsubsection.\arabic{partsubsubsetning} \else \rev.\thepart.\thesubsubsection.\arabic{partsubsubsetning} \fi }

\newsavebox{\nr}
\newsavebox{\fcolbox}
\newenvironment{Setning}[3]
{%
	\sbox\nr{\scriptsize\emph{#2}}
    \vspace{0.25cm}%
	\setlength{\fboxrule}{0.3mm}%
	\setlength{\fboxsep}{3mm}%
	%
	\scriptsize #3 \normalsize 
	\begin{lrbox}{\fcolbox}%
	\begin{minipage}{\textwidth-0.7cm}%
%	%
	\ifthenelse{\value{part} = 0} %
	{%
		\ifthenelse{\value{section} = 0}%
		{%
			\refstepcounter{setning}%
			\textbf{\thesetning~#1~~}%
		}%
		{%
			\ifthenelse{\value{subsection} = 0}%
			{%
				\refstepcounter{secsetning}%
				\textbf{\thesecsetning~#1~~}%
			}%
			{%
				\ifthenelse{\value{subsubsection} = 0}%
				{%
					\refstepcounter{subsetning}%
					\textbf{\thesubsetning~#1~~}%
				}%
				{%
					\refstepcounter{subsubsetning}%
					\textbf{\thesubsubsetning~#1~~}%
					
				}%
			}%
		}%
	}%
	{%
		\ifthenelse{\value{section} = 0}%
		{%						
			\refstepcounter{partsetning}%
			\textbf{\thepartsetning~#1~~}%
		}%
		{%
			\ifthenelse{\value{subsection} = 0}%
			{%
				\refstepcounter{partsecsetning}%
				\textbf{\thepartsecsetning~#1~~}%
			}%
			{%
				\ifthenelse{\value{subsubsection} = 0}%
				{%
					\refstepcounter{partsubsetning}%
					\textbf{\thepartsubsetning~#1~~}%
				}%
				{%
					\refstepcounter{partsubsubsetning}%
					\textbf{\thepartsubsubsetning~#1~~}%
				}%
			}%
		}%
	}%
}%
{%
	\vspace{-0.9cm}
	\begin{flushright}%
		\usebox{\nr}%
	\end{flushright}%	
	\vspace{-0.5cm}%
	\end{minipage}%
	\end{lrbox}\fbox{\usebox{\fcolbox}}%
    
	%\end{mbox}
	\vspace{0.3cm}%
}%

%---------------------------------------------------------------------------------------------------------
\newcommand\Setn[1] {
	\ifthenelse{\value{part} = 0} %
	{%
		\ifthenelse{\value{section} = 0}%
		{%
			\refstepcounter{setning}%
			\textbf{#1 \thesetning~~}%
		}%
		{%
			\ifthenelse{\value{subsection} = 0}%
			{%
				\refstepcounter{secsetning}%
				\textbf{#1 \thesecsetning~~}%
			}%
			{%
				\ifthenelse{\value{subsubsection} = 0}%
				{%
					\refstepcounter{subsetning}%
					\textbf{#1 \thesubsetning~~}%
				}%
				{%
					\refstepcounter{subsubsetning}%
					\textbf{#1 \thesubsubsetning~~}%
					
				}%
			}%
		}%
	}%
	{%
		\ifthenelse{\value{section} = 0}%
		{%						
			\refstepcounter{partsetning}%
			\textbf{#1 \thepartsetning~~}%
		}%
		{%
			\ifthenelse{\value{subsection} = 0}%
			{%
				\refstepcounter{partsecsetning}%
				\textbf{#1 \thepartsecsetning~~}%
			}%
			{%
				\ifthenelse{\value{subsubsection} = 0}%
				{%
					\refstepcounter{partsubsetning}%
					\textbf{#1 \thepartsubsetning~~}%
				}%
				{%
					\refstepcounter{partsubsubsetning}%
					\textbf{#1 \thepartsubsubsetning~~}%
				}%
			}%
		}%
	}%
}%

% Bevis innramming

\newenvironment{Bevis}[1]%
{%
	\sbox\nr{\scriptsize\emph{Q.E.D. #1}}	
	\vspace{0.25cm}%
	\textbf{Bevis #1}%
	\par \nointerlineskip \vspace{0.1cm}%
	\rule{\textwidth}{0.4pt}%
	\par \nointerlineskip \vspace{0.3cm}%
	%\begin{minipage}{\textwidth}%
}%
{%
	%\end{minipage}%
	\begin{flushright}%
		\par \nointerlineskip
		\rule{\textwidth}{0.4pt}%
		\par \nointerlineskip \vspace{0.1cm}%
		\usebox{\nr}%
	\end{flushright}%
	\vspace{-0.2cm}%
}%

\newenvironment{Proof}[1]%
{%
	\sbox\nr{\scriptsize\emph{Q.E.D. #1}}	
	\vspace{0.25cm}%
	\textbf{Proof #1}%
	\par \nointerlineskip \vspace{0.1cm}%
	\rule{\textwidth}{0.4pt}%
	\par \nointerlineskip \vspace{0.3cm}%
	%\begin{minipage}{\textwidth}%
}%
{%
	%\end{minipage}%
	\begin{flushright}%
		\par \nointerlineskip
		\rule{\textwidth}{0.4pt}%
		\par \nointerlineskip \vspace{0.1cm}%
		\usebox{\nr}%
	\end{flushright}%
	\vspace{-0.2cm}%
}%

% Eksempel innramming

\newcounter{eksempel}[part]
\renewcommand \theeksempel{\rev.\arabic{eksempel}}
\newcounter{seceksempel}[section]
\renewcommand \theseceksempel{\rev.\thesection.\arabic{seceksempel}}
\newcounter{subeksempel}[subsection]
\renewcommand \thesubeksempel{\rev.\thesubsection.\arabic{subeksempel}}
\newcounter{subsubeksempel}[subsubsection]
\renewcommand \thesubsubeksempel{\rev.\thesubsubsection.\arabic{subsubeksempel}}
\newcounter{parteksempel}[part]
\renewcommand \theparteksempel{\rev.\thepart.\arabic{parteksempel}}
\newcounter{partseceksempel}[section]
\renewcommand \thepartseceksempel{\rev.\thepart.\thesection.\arabic{partseceksempel}}
\newcounter{partsubeksempel}[subsection]
\renewcommand \thepartsubeksempel{\rev.\thepart.\thesubsection.\arabic{partsubeksempel}}
\newcounter{partsubsubeksempel}[subsubsection]
\renewcommand \thepartsubeksempel{\rev.\thepart.\thesubsubsection.\arabic{partsubsubeksempel}}

\newenvironment{Eksempel}[1]%
{%
	\sbox\nr{\scriptsize\emph{Eksempel #1}}	
	\vspace{0.25cm}%
	\textbf{Eksempel #1}%
	\par \nointerlineskip \vspace{0.1cm}%
	\rule{\textwidth}{0.4pt}%
	\par \nointerlineskip \vspace{0.3cm}%
	%\begin{minipage}{\textwidth}%
}%
{%
	%\end{minipage}%
	\begin{flushright}%
		\par \nointerlineskip
		\rule{\textwidth}{0.4pt}%
		\par \nointerlineskip \vspace{0.1cm}%
		\usebox{\nr}%
	\end{flushright}%
	\vspace{-0.2cm}%
}%

\newenvironment{Example}[1]%
{%
	\sbox\nr{\scriptsize\emph{Example} 
	\ifthenelse{\value{part} = 0} %
	{%
		\ifthenelse{\value{section} = 0}%
		{%
			\refstepcounter{eksempel}%
			\textit{\theeksempel}%
		}%
		{%
			\ifthenelse{\value{subsection} = 0}%
			{%
				\refstepcounter{seceksempel}%
				\textit{\theseceksempel}%
			}%
			{%
				\ifthenelse{\value{subsubsection} = 0}%
				{%
					\refstepcounter{subeksempel}%
					\textit{\thesubeksempel}%
				}%
				{%
					\refstepcounter{subsubeksempel}%
					\textit{\thesubsubeksempel}%
				}%
			}%
		}%
	}%
	{%
		\ifthenelse{\value{section} = 0}%
		{%						
			\refstepcounter{parteksempel}%
			\textit{\theparteksempel}%
		}%
		{%
			\ifthenelse{\value{subsection} = 0}%
			{%
				\refstepcounter{partseceksempel}%
				\textit{\thepartseceksempel}%
			}%
			{%
				\ifthenelse{\value{subsubsection} = 0}%
				{%
					\refstepcounter{partsubeksempel}%
					\textit{\thepartsubeksempel}%
				}%
				{%
					\refstepcounter{partsubsubeksempel}%
					\textit{\thepartsubsubeksempel}%
				}%
			}%
		}%
	}%	
	}%
		
	\vspace{0.25cm}%
	\ifthenelse{\value{part} = 0} %
	{%
		\ifthenelse{\value{section} = 0}%
		{%
			\textbf{Example \theeksempel :~#1}%
		}%
		{%
			\ifthenelse{\value{subsection} = 0}%
			{%
				\textbf{Example \theseceksempel :~#1}%
			}%
			{%
				\ifthenelse{\value{subsubsection} = 0}%
				{%
					\textbf{Example \thesubeksempel :~#1}%
				}%
				{%
					\textbf{Example \thesubsubeksempel :~#1}%
				}%
			}%
		}%
	}%
	{%
		\ifthenelse{\value{section} = 0}%
		{%						
			\textbf{Example \theparteksempel :~#1}%
		}%
		{%
			\ifthenelse{\value{subsection} = 0}%
			{%
				\textbf{Example \thepartseceksempel :~#1}%
			}%
			{%
				\ifthenelse{\value{subsubsection} = 0}%
				{%
					\textbf{Example \thepartsubeksempel :~#1}%
				}%
				{%
					\textbf{Example \thepartsubsubeksempel :~#1}%
				}%
			}%
		}%
	}%
	\par \nointerlineskip \vspace{0.1cm}%
	\rule{\textwidth}{0.4pt}%
	\par \nointerlineskip \vspace{0.3cm}%
	%\begin{minipage}{\textwidth}%
}%
{%
	%\end{minipage}%
	\begin{flushright}%
		\par \nointerlineskip
		\rule{\textwidth}{0.4pt}%
		\par \nointerlineskip \vspace{0.1cm}%
		\usebox{\nr}%
	\end{flushright}%
	\vspace{-0.2cm}%
}%

%---------------------------------------------------------------------------------------------------------

% Oppsett av dokumentet

\tolerance = 5000 % LaTeX er normalt streng n�r det gjelder linjebrytingen.
\hbadness = \tolerance % Vi vil v�re litt mildere, s�rlig fordi norsk har s�
\pretolerance = 2000 % mange lange sammensatte ord.

\let\OLDtableofcontents = \tableofcontents
\def\tableofcontents
{
	\setlength{\cftpartnumwidth}{4em}	
	\let\Section = \section
	\let\section = \OLDsection
	\OLDtableofcontents
	\let\section = \Section
}

\let\OLDlistoftables = \listoftables
\def\listoftables{
	\setlength{\cfttabindent}{0em}
	\setlength{\cfttabnumwidth}{4em} 
	\let\Section = \section
	\let\section = \OLDsection
	\OLDlistoftables
	\let\section = \Section
}

\let\OLDlistoffigures = \listoffigures
\def\listoffigures{
	\setlength{\cftfigindent}{0em}
	\setlength{\cftfignumwidth}{4em} 
	\let\Section = \section
	\let\section = \OLDsection
	\OLDlistoffigures
	\let\section = \Section
}

\def\innledende_sider
{
    \pagestyle{empty}
	\maketitle
	\thispagestyle{empty}
	\cleartooddpage 
	\thispagestyle{empty}
	\cleartooddpage
	\tableofcontents
	\cleartooddpage
	\listoftables
	\cleartooddpage
	\listoffigures
	\cleartooddpage
    \pagestyle{plain}
	\pagenumbering{arabic}
}

\def\innholdsfortegnelse
{
    \pagestyle{empty}
	\maketitle
	\thispagestyle{empty}
	\cleartooddpage 
	\thispagestyle{empty}
	\cleartooddpage
	\tableofcontents
	\cleartooddpage
    \pagestyle{plain}
	\pagenumbering{arabic}
}
%---------------------------------------------------------------------------------------------------------

% Journalfremside

\newcommand\journalforside[7]
{
	\pagestyle{empty}
	\begin{center}
		\LARGE \textbf{#1}
	\end{center}
	\setlength\extrarowheight{0.5cm}
	\begin{tabular}{| p{1.0 cm} p{0.8cm} | p{2.1cm} p{0.5cm} | p{2.4cm} p{3cm} |}
	\multicolumn{6}{l}{\textbf{Student:}} \\
	\hline
	\multicolumn{1}{|l}{\textbf{Navn:}} & \multicolumn{5}{l|}{#2}\\ 
	\hline
	\multicolumn{1}{|l}{\textbf{Epost:}} & \multicolumn{5}{l|}{\texttt{#3}}\\ 
	\hline
	\textbf{Gruppe:} & {#4} & \textbf{�velse nr.:} & {#5} & \textbf{Utf�rt dato:} & {#6}\\
	\hline
	\end{tabular}
	
	\setlength\extrarowheight{0.5cm}
	\begin{tabular}{| p{7cm} | p{4.5cm} |}
	\multicolumn{2}{l}{\textbf{Veileder/retter:}} \\
	\hline
	\textbf{Godkjent/rettet:} & \textbf{Dato:}   \\
	\hline
	\end{tabular}
	
	\vspace{0.5cm}
	\setlength\extrarowheight{0cm}
	\begin{tabular}{|p{11.92cm}|}
	\multicolumn{1}{l}{\textbf{Om du vil:}}\\
	\hline
	Spesielle forhold som retter b�r ta hensyn til:\\\\
	{#7}\\
	\hline
	\end{tabular}

	\vspace{1cm}
	\setlength{\unitlength}{3947sp}%
	\begin{picture}(5000,0)
	{
	\drawline[AHnb=0,linewidth=15,dash={100}0](-2000,0)(8000,0)
	}
	\end{picture}
	\vspace{0.3cm}

	\setlength\extrarowheight{0.5cm}
	\begin{tabular}{| p{1.0 cm} p{0.8cm} | p{2.1cm} p{0.5cm} | p{2.4cm} p{3cm} |}
	\multicolumn{6}{l}{\textbf{Student:}} \\
	\hline
	\multicolumn{1}{|l}{\textbf{Navn:}} & \multicolumn{5}{l|}{#2}\\ 
	\hline
	\multicolumn{1}{|l}{\textbf{Epost:}} & \multicolumn{5}{l|}{\texttt{#3}}\\ 
	\hline
	\textbf{Gruppe:} & {#4} & \textbf{�velse nr.:} & {#5} & \textbf{Utf�rt dato:} & {#6}\\
	\hline
	\end{tabular}
	
	\setlength\extrarowheight{0.5cm}
	\begin{tabular}{| p{7cm} | p{4.5cm} |}
	\multicolumn{2}{l}{\textbf{Veileder/retter:}} \\
	\hline
	\textbf{Godkjent/rettet:} & \textbf{Dato:}   \\
	\hline
	\end{tabular}
	
	\vspace{0.5cm}
	\setlength\extrarowheight{0cm}
	\begin{tabular}{|p{11.92cm}|}
	\multicolumn{1}{l}{\textbf{Om du vil:}}\\
	\hline
	Spesielle forhold som retter b�r ta hensyn til:\\\\
	{#7}\\
	\hline
	\end{tabular}
	\cleartooddpage
}

%---------------------------------------------------------------------------------------------------------
\newcounter{figur}[part]
\newcounter{secfigur}[section]
\renewcommand \thesecfigur{\thesection.\arabic{secfigur}}
\newcounter{subfigur}[subsection]
\renewcommand \thesubfigur{\thesubsection.\arabic{subfigur}}
\newcounter{subsubfigur}[subsubsection]
\renewcommand \thesubsubfigur{\thesubsubsection.\arabic{subsubfigur}}
\newcounter{partfigur}[part]
\renewcommand \thepartfigur{\thepart.\arabic{partfigur}}
\newcounter{partsecfigur}[section]
\renewcommand \thepartsecfigur{\thepart.\thesection.\arabic{partsecfigur}}
\newcounter{partsubfigur}[subsection]
\renewcommand \thepartsubfigur{\thepart.\thesubsection.\arabic{partsubfigur}}
\newcounter{partsubsubfigur}[subsubsection]
\renewcommand \thepartsubsubfigur{\thepart.\thesubsubsection.\arabic{partsubsubfigur}}

% Sette inn bilde
\newcommand\figur[4]
{
	\begin{figure}[!htp]%
		\vspace{-0.1cm}
		\begin{center}%
            %\includegraphics[scale = {#1}]{#2}
			\includegraphics[width = {#1}\textwidth]{#2}% 
		\end{center}%
		\ifthenelse{\value{part} = 0} %
		{%
			\ifthenelse{\value{section} = 0}%
			{%
				\refstepcounter{figur}%
				\renewcommand\thefigure{\thefigur}%
			}%
			{%
				\ifthenelse{\value{subsection} = 0}%
				{%
					\refstepcounter{secfigur}%
					\renewcommand\thefigure{\thesecfigur}%
				}%
				{%
					\ifthenelse{\value{subsubsection} = 0}%
					{%
						\refstepcounter{subfigur}%
						\renewcommand\thefigure{\thesubfigur}%
					}%
					{%
						\refstepcounter{subsubfigur}%
						\renewcommand\thefigure{\thesubsubfigur}%
					}%
				}%
			}%
		}%
		{%
			\ifthenelse{\value{section} = 0}%
			{%						
				\refstepcounter{partfigur}%
				\renewcommand\thefigure{\thepartfigur}%
			}%
			{%
				\ifthenelse{\value{subsection} = 0}%
				{%
					\refstepcounter{partsecfigur}%
					\renewcommand\thefigure{\thepartsecfigur}%
				}%
				{%
					\ifthenelse{\value{subsubsection} = 0}%
					{%
						\refstepcounter{partsubfigur}%
						\renewcommand\thefigure{\thepartsubfigur}%
					}%
					{%
						\refstepcounter{partsubsubfigur}%
						\renewcommand\thefigure{\thepartsubsubfigur}%
					}%
				}%
			}%
		}%	
		\vspace{-0.7cm}	
		\caption{\textit{#3}}%
		\label{#4}
		\vspace{-0.4cm}
	\end{figure}%
}%

\newcommand\multifigur[3]
{
	\begin{figure}[!htp]%
		\vspace{-0.7cm} 
		\begin{center}%
            #1
		\end{center}%
		\ifthenelse{\value{part} = 0} %
		{%
			\ifthenelse{\value{section} = 0}%
			{%
				\refstepcounter{figur}%
				\renewcommand\thefigure{\thefigur}%
			}%
			{%
				\ifthenelse{\value{subsection} = 0}%
				{%
					\refstepcounter{secfigur}%
					\renewcommand\thefigure{\thesecfigur}%
				}%
				{%
					\ifthenelse{\value{subsubsection} = 0}%
					{%
						\refstepcounter{subfigur}%
						\renewcommand\thefigure{\thesubfigur}%
					}%
					{%
						\refstepcounter{subsubfigur}%
						\renewcommand\thefigure{\thesubsubfigur}%
					}%
				}%
			}%
		}%
		{%
			\ifthenelse{\value{section} = 0}%
			{%						
				\refstepcounter{partfigur}%
				\renewcommand\thefigure{\thepartfigur}%
			}%
			{%
				\ifthenelse{\value{subsection} = 0}%
				{%
					\refstepcounter{partsecfigur}%
					\renewcommand\thefigure{\thepartsecfigur}%
				}%
				{%
					\ifthenelse{\value{subsubsection} = 0}%
					{%
						\refstepcounter{partsubfigur}%
						\renewcommand\thefigure{\thepartsubfigur}%
					}%
					{%
						\refstepcounter{partsubsubfigur}%
						\renewcommand\thefigure{\thepartsubsubfigur}%
					}%
				}%
			}%
		}%		
		\vspace{-0.7cm}
		\caption{\textit{#2}}%
		\label{#3}
		%\vspace{-0.4cm}
	\end{figure}%
}%

\newcommand\subfigur[3]
{
	\begin{tabular}[t]{l}
		\begin{tabular}{c} \\ \normalsize \textbf{#1} \\ \end{tabular}  \\
	 	\includegraphics[width = {#2}\textwidth]{#3} 
	\end{tabular}
}

\newcommand\flowchart[5]
{
	\begin{figure}[!htp]%
		\vspace{-0.1cm}
		\tiny
		\begin{center}
		\begin{psmatrix}[rowsep={#1},colsep={#2}] 
			#3
		\end{psmatrix}
		\end{center}%
		\ifthenelse{\value{part} = 0} %
		{%
			\ifthenelse{\value{section} = 0}%
			{%
				\refstepcounter{figur}%
				\renewcommand\thefigure{\thefigur}%
			}%
			{%
				\ifthenelse{\value{subsection} = 0}%
				{%
					\refstepcounter{secfigur}%
					\renewcommand\thefigure{\thesecfigur}%
				}%
				{%
					\ifthenelse{\value{subsubsection} = 0}%
					{%
						\refstepcounter{subfigur}%
						\renewcommand\thefigure{\thesubfigur}%
					}%
					{%
						\refstepcounter{subsubfigur}%
						\renewcommand\thefigure{\thesubsubfigur}%
					}%
				}%
			}%
		}%
		{%
			\ifthenelse{\value{section} = 0}%
			{%						
				\refstepcounter{partfigur}%
				\renewcommand\thefigure{\thepartfigur}%
			}%
			{%
				\ifthenelse{\value{subsection} = 0}%
				{%
					\refstepcounter{partsecfigur}%
					\renewcommand\thefigure{\thepartsecfigur}%
				}%
				{%
					\ifthenelse{\value{subsubsection} = 0}%
					{%
						\refstepcounter{partsubfigur}%
						\renewcommand\thefigure{\thepartsubfigur}%
					}%
					{%
						\refstepcounter{partsubsubfigur}%
						\renewcommand\thefigure{\thepartsubsubfigur}%
					}%
				}%
			}%
		}%	
		\vspace{-0.7cm}	
		\caption{\textit{#4}}%
		\label{#5}
		\vspace{-0.4cm}
	\end{figure}%
}%

\newcommand\reffig[1]{\figurename\,\ref{#1}}
%---------------------------------------------------------------------------------------------------------
\newcolumntype{I}{!{\vrule width 1pt}}
\newlength\savewidth
\newcommand\whline
{
	\noalign{
		\global\savewidth\arrayrulewidth
		\global\arrayrulewidth 1pt}%
	\hline
	\noalign
	{
		\global\arrayrulewidth\savewidth
	}%
}%

\newcounter{tabell}[part]
\newcounter{sectabell}[section]
\renewcommand \thesectabell{\thesection.\arabic{sectabell}}
\newcounter{subtabell}[subsection]
\renewcommand \thesubtabell{\thesubsection.\arabic{subtabell}}
\newcounter{subsubtabell}[subsubsection]
\renewcommand \thesubsubtabell{\thesubsubsection.\arabic{subsubtabell}}
\newcounter{parttabell}[part]
\renewcommand \theparttabell{\thepart.\arabic{parttabell}}
\newcounter{partsectabell}[section]
\renewcommand \thepartsectabell{\thepart.\thesection.\arabic{partsectabell}}
\newcounter{partsubtabell}[subsection]
\renewcommand \thepartsubtabell{\thepart.\thesubsection.\arabic{partsubtabell}}
\newcounter{partsubsubtabell}[subsubsection]
\renewcommand \thepartsubsubtabell{\thepart.\thesubsubsection.\arabic{partsubsubtabell}}
								
% Tabell
\newcommand\tabell[6]
{
  	\begin{table}[!htp]%
  		\begin{center}
			\setlength\extrarowheight{2pt}
			#2
			\begin{tabular}{#1}
				\whline
				#3
				\whline
				#4
				\whline
			\end{tabular}
		\end{center}
		\ifthenelse{\value{part} = 0} %
		{%
			\ifthenelse{\value{section} = 0}%
			{%
				\refstepcounter{tabell}%
				\renewcommand\thetable{\thetabell}%
			}%
			{%
				\ifthenelse{\value{subsection} = 0}%
				{%
					\refstepcounter{sectabell}%
					\renewcommand\thetable{\thesectabell}%
				}%
				{%
					\ifthenelse{\value{subsubsection} = 0}%
					{%
						\refstepcounter{subtabell}%
						\renewcommand\thetable{\thesubtabell}%
					}%
					{%
						\refstepcounter{subsubtabell}%
						\renewcommand\thetable{\thesubsubtabell}%
					}%
				}%
			}%
		}%
		{%
			\ifthenelse{\value{section} = 0}%
			{%						
				\refstepcounter{parttabell}%
				\renewcommand\thetable{\theparttabell}%
			}%
			{%
				\ifthenelse{\value{subsection} = 0}%
				{%
					\refstepcounter{partsectabell}%
					\renewcommand\thetable{\thepartsectabell}%
				}%
				{%
					\ifthenelse{\value{subsubsection} = 0}%
					{%
						\refstepcounter{partsubtabell}%
						\renewcommand\thetable{\thepartsubtabell}%
					}%
					{%
						\refstepcounter{partsubsubtabell}%
						\renewcommand\thetable{\thepartsubsubtabell}%
					}%
				}%
			}%
		}%
		\vspace{-0.5cm}
    	\caption{\textit{#5}}
		\label{#6}
	\end{table}
	%\vspace{-0.4cm}
}
\newcommand\multitabell[5]
{
  	\begin{table}[!htp]%
	\vspace{-0.4cm}  	
  		\begin{center}
			\setlength\extrarowheight{2pt}
			#2
			\begin{tabular}{#1}
				#3
			\end{tabular}
		\end{center}
		\ifthenelse{\value{part} = 0} %
		{%
			\ifthenelse{\value{section} = 0}%
			{%
				\refstepcounter{tabell}%
				\renewcommand\thetable{\thetabell}%
			}%
			{%
				\ifthenelse{\value{subsection} = 0}%
				{%
					\refstepcounter{sectabell}%
					\renewcommand\thetable{\thesectabell}%
				}%
				{%
					\ifthenelse{\value{subsubsection} = 0}%
					{%
						\refstepcounter{subtabell}%
						\renewcommand\thetable{\thesubtabell}%
					}%
					{%
						\refstepcounter{subsubtabell}%
						\renewcommand\thetable{\thesubsubtabell}%
					}%
				}%
			}%
		}%
		{%
			\ifthenelse{\value{section} = 0}%
			{%						
				\refstepcounter{parttabell}%
				\renewcommand\thetable{\theparttabell}%
			}%
			{%
				\ifthenelse{\value{subsection} = 0}%
				{%
					\refstepcounter{partsectabell}%
					\renewcommand\thetable{\thepartsectabell}%
				}%
				{%
					\ifthenelse{\value{subsubsection} = 0}%
					{%
						\refstepcounter{partsubtabell}%
						\renewcommand\thetable{\thepartsubtabell}%
					}%
					{%
						\refstepcounter{partsubsubtabell}%
						\renewcommand\thetable{\thepartsubsubtabell}%
					}%
				}%
			}%
		}%
		\vspace{-0.5cm}
    	\caption{\textit{#4}}
		\label{#5}
	\end{table}
	\vspace{-0.4cm}
}
\newcommand\subtabell[4]
{
	\begin{tabular}[t]{rl}
	\begin{tabular}{c} \\ \normalsize \textbf{#1} \\ \end{tabular} &
	\begin{tabular}[t]{#2}
		\whline
		#3
		\whline
		#4
		\whline
	\end{tabular}
	\end{tabular}
}
\newcommand\reftab[1]{\tablename\,\ref{#1}}

%---------------------------------------------------------------------------------------------------------
% Kode innramming
\newcommand\kode[1]{\verbatiminput{#1}}

\newcounter{kode}[part]
\renewcommand \thekode{\rev.\arabic{kode}}
\newcounter{seckode}[section]
\renewcommand \theseckode{\rev.\thesection.\arabic{seckode}}
\newcounter{subkode}[subsection]
\renewcommand \thesubkode{\rev.\thesubsection.\arabic{subkode}}
\newcounter{subsubkode}[subsubsection]
\renewcommand \thesubsubkode{\rev.\thesubsubsection.\arabic{subsubkode}}
\newcounter{partkode}[part]
\renewcommand \thepartkode{\rev.\thepart.\arabic{partkode}}
\newcounter{partseckode}[section]
\renewcommand \thepartseckode{\rev.\thepart.\thesection.\arabic{partseckode}}
\newcounter{partsubkode}[subsection]
\renewcommand \thepartsubkode{\rev.\thepart.\thesubsection.\arabic{partsubkode}}
\newcounter{partsubsubkode}[subsubsection]
\renewcommand \thepartsubsubkode{\rev.\thepart.\thesubsubsection.\arabic{partsubsubkode}}

\newcommand\code[2]%
{%
	\sbox\nr{\scriptsize\emph{Code} 
	\ifthenelse{\value{part} = 0} %
	{%
		\ifthenelse{\value{section} = 0}%
		{%
			\refstepcounter{kode}%
			\textit{\thekode}%
		}%
		{%
			\ifthenelse{\value{subsection} = 0}%
			{%
				\refstepcounter{seckode}%
				\textit{\theseckode}%
			}%
			{%
				\ifthenelse{\value{subsubsection} = 0}%
				{%
					\refstepcounter{subkode}%
					\textit{\thesubkode}%
				}%
				{%
					\refstepcounter{subsubeksempel}%
					\textit{\thesubsubkode}%
				}%
			}%
		}%
	}%
	{%
		\ifthenelse{\value{section} = 0}%
		{%						
			\refstepcounter{partkode}%
			\textit{\thepartkode}%
		}%
		{%
			\ifthenelse{\value{subsection} = 0}%
			{%
				\refstepcounter{partseckode}%
				\textit{\thepartseckode}%
			}%
			{%
				\ifthenelse{\value{subsubsection} = 0}%
				{%
					\refstepcounter{partsubkode}%
					\textit{\thepartsubkode}%
				}%
				{%
					\refstepcounter{partsubsubkode}%
					\textit{\thepartsubsubkode}%
				}%
			}%
		}%
	}%	
	}%
	\vspace{0.25cm}%
	\ifthenelse{\value{part} = 0} %
	{%
		\ifthenelse{\value{section} = 0}%
		{%
			\textbf{Code \thekode :~{#2}}%
		}%
		{%
			\ifthenelse{\value{subsection} = 0}%
			{%
				\textbf{Code \theseckode :~#2}%
			}%
			{%
				\ifthenelse{\value{subsubsection} = 0}%
				{%
					\textbf{Code \thesubkode :~#2}%
				}%
				{%
					\textbf{Code \thesubsubkode :~#2}%
				}%
			}%
		}%
	}%
	{%
		\ifthenelse{\value{section} = 0}%
		{%						
			\textbf{Code \thepartkode :~#2}%
		}%
		{%
			\ifthenelse{\value{subsection} = 0}%
			{%
				\textbf{Code \thepartseckode :~#2}%
			}%
			{%
				\ifthenelse{\value{subsubsection} = 0}%
				{%
				\textbf{Code \thepartsubkode :~#2}%
				}%
				{%
					\textbf{Code \thepartsubsubkode :~#2}%
				}%
			}%
		}%
	}%
	\par \nointerlineskip \vspace{0.1cm}%
	\rule{\textwidth}{0.4pt}%
	\par \nointerlineskip \vspace{0.3cm}%
	%\begin{minipage}{\textwidth}%
	\tiny\verbatiminput{#1}\normalsize
	%\end{minipage}%
	\begin{flushright}%
		\par \nointerlineskip
		\rule{\textwidth}{0.4pt}%
		\par \nointerlineskip \vspace{0.1cm}%
		\usebox{\nr}%
	\end{flushright}%
	\vspace{-0.2cm}%
}%
%---------------------------------------------------------------------------------------------------------
% Oscilloskopbilde
\newcommand\oscilloskop[6]
{
	\begin{figure}[!htp]
		\begin{center}
			\setlength{\unitlength}{3947sp}%
			%
				%\begingroup\makeatletter\ifx\SetFigFont\undefined%
				%\gdef\SetFigFont#1#2#3#4#5{%
	  			%\reset@font\fontsize{#1}{#2pt}%
	  			%\fontfamily{#3}\fontseries{#4}\fontshape{#5}%
	  			%\selectfont}%
			%\fi\endgroup%
			\begin{picture}(6000,2500)
			{
				\color[rgb]{0,0,0}
				\scriptsize
	
				\put(0,0){\includegraphics[width = 0.6\textwidth]{#1}}
				\put(3500,2000)
				{
				\begin{tabular}{l p{3cm}}
					\vspace{0.5cm}
					Signal: 		& {#2} \\
					\vspace{0.1cm}
			    	V/div: 		& {#3} \\
					sec/div: 	& {#4}
				\end{tabular}
				} 
			}
			\end{picture}%
		\end{center}
		\caption{\textit{#5}}
		\label{#6}
	\end{figure}
}{}
%---------------------------------------------------------------------------------------------------------
\newcommand{\tegning}[5]
{
	\begin{figure}[!htp]%
		\begin{center}
			\setlength{\unitlength}{3947sp}%
			\begin{picture}(#1,#2)
				\color[rgb]{0,0,0}
				\scriptsize 
				{#3}
			\end{picture}
		\end{center}%
		\ifthenelse{\value{part} = 0} %
		{%
			\ifthenelse{\value{section} = 0}%
			{%
				\refstepcounter{figur}%
				\renewcommand\thefigure{\thefigur}%
			}%
			{%
				\ifthenelse{\value{subsection} = 0}%
				{%
					\refstepcounter{secfigur}%
					\renewcommand\thefigure{\thesecfigur}%
				}%
				{%
					\ifthenelse{\value{subsubsection} = 0}%
					{%
						\refstepcounter{subfigur}%
						\renewcommand\thefigure{\thesubfigur}%
					}%
					{%
						\refstepcounter{subsubfigur}%
						\renewcommand\thefigure{\thesubsubfigur}%
					}%
				}%
			}%
		}%
		{%
			\ifthenelse{\value{section} = 0}%
			{%						
				\refstepcounter{partfigur}%
				\renewcommand\thefigure{\thepartfigur}%
			}%
			{%
				\ifthenelse{\value{subsection} = 0}%
				{%
					\refstepcounter{partsecfigur}%
					\renewcommand\thefigure{\thepartsecfigur}%
				}%
				{%
					\ifthenelse{\value{subsubsection} = 0}%
					{%
						\refstepcounter{partsubfigur}%
						\renewcommand\thefigure{\thepartsubfigur}%
					}%
					{%
						\refstepcounter{partsubsubfigur}%
						\renewcommand\thefigure{\thepartsubsubfigur}%
					}%
				}%
			}%
		}%		
		\caption{\textit{#4}}%
		\label{#5}
	\end{figure}%
}
\newenvironment{Tegning}[3]
{%
	\sbox\nr{\textit{#3}}
	\ifthenelse{\value{part} = 0} %
	{%
		\ifthenelse{\value{section} = 0}%
		{%
			\refstepcounter{figur}%
		}%
		{%
			\ifthenelse{\value{subsection} = 0}%
			{%
				\refstepcounter{secfigur}%
			}%
			{%
				\ifthenelse{\value{subsubsection} = 0}%
				{%
					\refstepcounter{subfigur}%
				}%
				{%
					\refstepcounter{subsubfigur}%
				}%
			}%
		}%
	}%
	{%
		\ifthenelse{\value{section} = 0}%
		{%						
			\refstepcounter{partfigur}%
		}%
		{%
			\ifthenelse{\value{subsection} = 0}%
			{%
				\refstepcounter{partsecfigur}%
			}%
			{%
				\ifthenelse{\value{subsubsection} = 0}%
				{%
					\refstepcounter{partsubfigur}%
				}%
				{%
					\refstepcounter{partsubsubfigur}%
				}%
			}%
		}%
	}%	
	\begin{figure}[!htp]
		\begin{center}
			\setlength{\unitlength}{3947sp}%
			%
				%\begingroup\makeatletter\ifx\SetFigFont\undefined%
				%\gdef\SetFigFont#1#2#3#4#5{%
	  			%\reset@font\fontsize{#1}{#2pt}%
	  			%\fontfamily{#3}\fontseries{#4}\fontshape{#5}%
	  			%\selectfont}%
			%\fi\endgroup%
			\begin{picture}(#1,#2)
				\color[rgb]{0,0,0}
				\scriptsize
	
}%
{
			\end{picture}%
		\end{center}
		\begin{center}%	
			\ifthenelse{\value{part} = 0} %
			{%
				\ifthenelse{\value{section} = 0}%
				{%
					\text{Figur \thefigur:~~}%
				}%
				{%
					\ifthenelse{\value{subsection} = 0}%
					{%
						\text{Figur \thesecfigur:~~}%
					}%
					{%
						\ifthenelse{\value{subsubsection} = 0}%
						{%
							\text{Figur \thesubfigur:~~}%
						}%
						{%
							\text{Figur \thesubsubfigur:~~}%
						}%
					}%
				}%
			}%
			{%
				\ifthenelse{\value{section} = 0}%
				{%						
					\text{Figur \thepartfigur:~~}%
				}%
				{%
					\ifthenelse{\value{subsection} = 0}%
					{%
						\text{Figur \thepartsecfigur:~~}%
					}%
					{%
						\ifthenelse{\value{subsubsection} = 0}%
						{%
							\text{Figur \thepartsubfigur:~~}%
						}%
						{%
							\text{Figur \thepartsubsubfigur:~~}%
						}%
					}%
				}%
			}%
			\usebox{\nr}%
		\end{center}%
	\end{figure}
}

\newcommand\measureline[8]
{
	\drawline[AHnb=1,AHangle=30,AHLength=75,AHlength=0,ATnb=1,ATangle=30,ATLength=75,ATlength=0](#1,#2)(#3,#4)
	\drawline[AHnb=1,AHangle=90,AHLength=50,AHlength=0,ATnb=1,ATangle=90,ATLength=50,ATlength=0](#1,#2)(#3,#4)
	\drawline[AHnb=0,dash={50}0](#5,#6)(#1,#2)
	\drawline[AHnb=0,dash={50}0](#7,#8)(#3,#4)
}

\newcommand\textoval[5]
{
  \node[Nadjust=wh](#1)(#2,#3)
	{
		\begin{tabular}{#4}
			#5
		\end{tabular}
	}
}
\newcommand\textbox[5]
{
  \node[Nadjust=wh,Nmr=0](#1)(#2,#3)
	{
		\begin{tabular}{#4}
			#5
		\end{tabular}
	}
}
%---------------------------------------------------------------------------------------------------------
\newcommand\blankfigur[2]
{
  \cleartooddpage
  \tegning{5000}{9000}{}{#1}{#2}
  \cleartooddpage 
}
%---------------------------------------------------------------------------------------------------------
\usepackage{fullpage}

\renewcommand\title{FYS4150 - Computational Physics - Project 4}

\renewcommand\author{Eimund Smestad}
%\newcommand\adress{}
\renewcommand\date{\today}
\newcommand\email{\href{mailto:eimundsm@fys.uio.no}{eimundsm@fys.uio.no}}

%\lstset{language=[Visual]C++,caption={Descriptive Caption Text},label=DescriptiveLabel}
\lstset{language=c++}
\lstset{basicstyle=\small}
\lstset{backgroundcolor=\color{white}}
\lstset{frame=single}
\lstset{stringstyle=\ttfamily}
\lstset{keywordstyle=\color{black}\bfseries}
\lstset{commentstyle=\itshape\color{black}}
\lstset{showspaces=false}
\lstset{showstringspaces=false}
\lstset{showtabs=false}
\lstset{breaklines}

\begin{document}
\maketitle
\begin{flushleft}

\begin{abstract}

\end{abstract}

\section{Diffusion of neurotransmitters}

I will study diffusion as a transport process for neurotransmitters  across synaptic cleft separating the cell membrane of two neurons, for more detail see \cite{project4}. The diffusion equation is the partial differential equation 

\begin{align*}
\frac{\partial u\p{\textbf{x},t}}{\partial t} = \nabla \cdot\p{D\p{\textbf{x},t}\boldsymbol{\nabla} u\p{\textbf{x},t}}\,,
\end{align*}

where $u$ is the concentration of particular neurotransmitters at location $\textbf{x}$ and time $t$ with the diffusion coefficient $D$. In this study I consider the diffusion coefficient as constant, which simplify the diffusion equation to the heat equation

\begin{align*}
\frac{\partial u\p{\textbf{x},t}}{\partial t} = D\nabla^2 u\p{\textbf{x},t}\,.
\end{align*}

It is further assumed that the neurotransmitter concentration $u$ is only dependent on the distance $x$ in direction between the presynaptic to the postsynaptic across the synaptic cleft. Hence we have the differential equation

\begin{align}
\frac{\partial u\p{x,t}}{\partial t} = D\frac{\partial^2 u\p{x,t}}{\partial x^2}\,.
\label{eq_1}
\end{align}

The boundary and initial condition that I'm going to study is 

\begin{align}
\forall t \in\mathbb{R}_{0}: u\p{0,t} = u_0\,, \quad \forall  t\in\mathbb{R} : u\p{d, t} = 0 \quad \text{and} \quad \forall x\in\mathbb{R}_{0+}^{d-} \forall t \in \mathbb{R}^{0}:u\p{x,t}=0\,,
\label{eq_2}
\end{align}

where $u_0$ are kept constant at the presynaptic, $d$ is the distance between the presynaptic and the postsynaptic. Note that the notation $\forall x\in\mathbb{R}_{a+}^{b-} \Leftrightarrow a < x < b$, where as $\forall x\in\mathbb{R}_{a}^{b} \Leftrightarrow a \leq x \leq b$. Note also that these boundary conditions implies that the neurotransmitters are immediately absorbed at the postsynaptic, and for $t<0$ there are no neurotransmitters between the pre- and postsynaptic. \linebreak

To solve the differential equation \eqref{eq_1} with the boundary condition \eqref{eq_2} we start by separating the concentration $u\p{x,t}$ into two functions $u_1\p{x}$ and $u_2\p{x,t}$ in the time and space of the signal transmission of the neurotransmitters

\begin{align}
\forall x \in \mathbb{R}_0^d \forall t \in \mathbb{R}_{0}: u\p{x,t} = u_1\p{x} + u_2\p{x,t} \,,
\label{eq_3}
\end{align}

such that $u_2$ satisfies the Dirichlet boundary condition $u_2\p{0,t}=u_2\p{d,t}=0$, which forces $u_2\p{x,t}$ to be separated into two functions $u_3\p{x}$ and $u_4\p{t}$ as follows

\begin{align}
u_2\p{0,t} = u_2\p{d,t} = 0 \quad \Rightarrow \quad u_2\p{x,t} = u_3\p{x}u_4\p{t} \qquad \text{when $d\neq 0$ and $u_3\p{0}=u_3\p{d}=0$.}
\label{eq_4}
\end{align}

Now putting \eqref{eq_2} and \eqref{eq_3} into \eqref{eq_1} yields

\begin{align*}
u_3\p{x}\frac{\partial u_4\p{t}}{\partial t} = D\p{u_4\p{t}\frac{\partial^2 u_3\p{x}}{\partial x^2} + \frac{\partial^2 u_1\p{x}}{\partial x^2}} \,.
\end{align*}

We wish that $\frac{\partial^2 u_1\p{x}}{\partial x^2} =0$ because then we can separate this partial differential equation by variables, which puts the requirement $u_1\p{x} = a_0 + a_1 x$. Since the separation of $u\p{x,t}$ into $u_1\p{x}$ and $u_2\p{x,t}$ in \eqref{eq_3} can be done arbitrarily without changing the solution of $u\p{x,t}$, means that the requirement $u_1\p{x} = a_0 + a_1 x$ is allowed. However we also need to investigate that $u_1\p{x} = a_0 + a_1 x$ satisfies the boundary condition in \eqref{eq_2};

\begin{align*}
u\p{0,t} &= u_1\p{0} + u_2\p{0,t} = u_1\p{0} = u_0
\\
u\p{d,t} &= u_1\p{d} + u_2\p{d,t} = u_1\p{d} = 0\,,
\end{align*}

where the Dirichlet boundary condition in \eqref{eq_4} are used. And we see that the boundary condition can satisfy the requirement $u_1\p{x} = a_0 + a_1 x$ when

\begin{align}
u_1\p{x} = u_0\p{1-\frac{x}{d}}\,.
\label{eq_5}
\end{align} 

The differential equation in \eqref{eq_1} can now be written as

\begin{align*}
\frac{1}{Du_4\p{t}} \frac{\partial u_4\p{t}}{\partial t} = \frac{1}{u_3\p{x}}\frac{\partial^2 u_3\p{x}}{\partial x^2} = -\lambda^2 \,,
\end{align*}

where $\lambda$ is a constant, because $t$ and $x$ can vary independently. These two equations have the following solution

\begin{align*}
u_3\p{x} &= A\sin\p{\lambda x + \varphi} \qquad \text{and}
\\
u_4\p{t} &= C\e^{-D\lambda^2 t}\,.
\end{align*}

Applying the boundary conditions for $u_3$ in \eqref{eq_4} that $u_3\p{0}=u_4\p{d}=0$ gives

\begin{align*}
\lambda = \frac{n\pi}{d} \qquad \text{for } n \in \mathbb{N}/\kr{0}\,,
\end{align*}

where I let $\mathbb{N}=\mathbb{N}_{-\infty}^{\infty}$ represent all positive and negative integers. Hence

\begin{align*}
u_2\p{x,t} = u_3\p{x}u_4\p{t} = \sum_{n=1} A_n \sin\p{n\pi x} \exp\p{-D\p{\frac{n \pi}{d}}^2 t}\,,
\end{align*}

where the negative values of $n$ is absorbed into the coefficient $A_n$. Applying the initial condition from \eqref{eq_2}

\begin{align}
u\p{x,0} = u_0\p{1-\frac{x}{d}} + \sum_{n=1} A_n \sin\p{n\pi \frac{x}{d}} = 0 \qquad \text{for } x\in\mathbb{R}_0^d\,.
\label{eq_6}
\end{align}

We need to determine the coefficients $A_n$, and the trick is to do something with the equation above such that we isolate the $A_n$ coefficients. To achieve this we use the fact that $\sin\p{n\pi\frac{x}{d}}$ is orthogonal with $\sin\p{m\pi\frac{x}{d}}$ under integration 

\begin{align*}
&\int \sin\p{m\pi\frac{x}{d}}\sin\p{n\pi\frac{x}{d}}\,\mathrm{d}x = -\frac{d}{m\pi}\cos\p{m\pi\frac{x}{d}}\sin\p{n\pi\frac{x}{d}} + \frac{n}{m}\int \cos\p{m\pi\frac{x}{d}}\cos\p{n\pi\frac{x}{d}}\,\mathrm{d}x
\\
&\quad = -\frac{d}{m\pi}\cos\p{m\pi\frac{x}{d}}\sin\p{n\pi\frac{x}{d}} + \frac{n}{m}\p{\frac{d}{\pi m} \sin\p{m\pi\frac{x}{d}} \cos\p{n\pi\frac{x}{d}} + \frac{n}{m}\int \sin\p{m\pi\frac{x}{d}}\sin\p{n\pi\frac{x}{d}}\,\mathrm{d}x} \,,
\end{align*}

where I have used integration by parts $\int u\dot{v} = uv - \int \dot{u}v$. Solving this equation with regard to the integral we get

\begin{align*}
\int \sin\p{m\pi\frac{x}{d}}\sin\p{n\pi\frac{x}{d}}\,\mathrm{d}x = \frac{d}{\pi\p{n^2 - m^2}} \p{n \sin\p{m\pi\frac{x}{d}}\cos\p{n\pi\frac{x}{d}} - m \cos\p{m\pi\frac{x}{d}}\sin\p{n\pi\frac{x}{d}}} \,.
\end{align*}

We want to make the result of this integral zero for $n\neq m$, which is the result if $x=\frac{k d}{2}$ and $x=\frac{\ell d}{2}$ where $k,\ell\in\mathbb{N}$ (for zero,negative and positive integers), because then $\sin\cos$ parts above becomes zero. This means that we need to integrate from $x=\frac{k d}{2}$ and $x=\frac{\ell d}{2}$. However the result above is not defined for $n=m$ because we get $\frac{0}{0}$. So we redo the integration for $n=m$;

\begin{align*}
&\int_{\frac{k d}{2}}^{\frac{\ell d}{2}} \sin^2\p{n\pi\frac{x}{d}}\,\mathrm{d}x 
= -\kl{ \frac{d}{\pi n}\cos\p{n\pi\frac{x}{d}}\sin\p{n\pi\frac{x}{d}}}_{\frac{k d}{2}}^{\frac{\ell d}{2}} + \int_{\frac{k d}{2}}^{\frac{\ell d}{2}} \cos^2\p{n\pi\frac{x}{d}}\,\mathrm{d}x
= \int_{\frac{k d}{2}}^{\frac{\ell d}{2}}  \cos^2\p{n\pi\frac{x}{d}}\,\mathrm{d}x
\\ 
&\quad = \int_{\frac{k d}{2}}^{\frac{\ell d}{2}}  \p{1-\sin^2\p{n\pi\frac{x}{d}}}\,\mathrm{d}x 
= \kl{x}_{\frac{k d}{2}}^{\frac{\ell d}{2}} - \int_{\frac{k d}{2}}^{\frac{\ell d}{2}}  \sin^2\p{n\pi\frac{x}{d}}\,\mathrm{d}x 
= \frac{d}{2}\p{\ell-k} - \int_{\frac{k d}{2}}^{\frac{\ell d}{2}}  \sin^2\p{n\pi\frac{x}{d}}\,\mathrm{d}x 
= \frac{d}{4}\p{\ell-k} \,,
\end{align*}

where I solve the equation with regard to in $\int\frac{d}{2}\p{\ell-k} \sin^2\p{n\pi\frac{x}{d}}\,\mathrm{d}x$ in the last step. I have also used integration by parts $\int u\dot{v} = uv - \int \dot{u}v$ and the Pythagoras trigonometric relation $\sin^2 x+ \cos^2 x = 1$. So the solution of the following integral is

\begin{align*}
\int_{\frac{k d}{2}}^{\frac{\ell d}{2}} \sin\p{m\pi\frac{x}{d}}\sin\p{n\pi\frac{x}{d}}\,\mathrm{d}x = \frac{d}{4}\p{\ell-k}\delta_{mn}\,,
\end{align*}

where $\delta_{mn}$ is the Kronecker delta, and hence I have showed the orthogonality of the above integral. Applying this to \eqref{eq_6};

\begin{align*}
\sum_{n=1}\int_{\frac{k d}{2}}^{\frac{\ell d}{2}} A_n\sin\p{m\pi\frac{x}{d}}\sin\p{n\pi\frac{x}{d}}\,\mathrm{d}x = \int_{\frac{k d}{2}}^{\frac{\ell d}{2}}u_0\sin\p{m\pi\frac{x}{d}}\p{\frac{x}{d}-1}\,\mathrm{d}x\,,
\end{align*}

we can isolate $A_n$ at $n=m$ because of the Kronecker delta $\delta_{nm}$;

\begin{align*}
A_n & = \frac{4}{d\p{\ell-k}} \int_{\frac{k d}{2}}^{\frac{\ell d}{2}}u_0\sin\p{n\pi\frac{x}{d}}\p{\frac{x}{d}-1}\,\mathrm{d}x 
= \frac{4 u_0}{n\p{\ell-k}\pi} \p{-\kl{\p{\frac{x}{d}-1}\cos\p{n\pi\frac{x}{d}}}_{\frac{k d}{2}}^{\frac{\ell d}{2}} + \frac{1}{d}\int_{\frac{k d}{2}}^{\frac{\ell d}{2}}\cos\p{n\pi\frac{x}{d}}\,\mathrm{d}x }
\\
&= \frac{4 u_0}{n\p{\ell-k}\pi} \p{-\kl{\p{\frac{x}{d}-1}\cos\p{n\pi\frac{x}{d}}}_{\frac{k d}{2}}^{\frac{\ell d}{2}} + \frac{1}{n\pi} \kl{\sin\p{n\pi\frac{x}{d}}}_{\frac{k d}{2}}^{\frac{\ell d}{2}} } 
\\
&= \frac{4 u_0}{n\p{\ell-k}\pi}\p{\p{\frac{k}{2}-1}\cos\p{\frac{k n\pi}{2}} - \p{\frac{\ell}{2}-1}\cos\p{\frac{\ell n \pi}{2}} + \frac{1}{n\pi}\p{\sin\p{\frac{\ell n \pi}{2}}-\sin\p{\frac{k n \pi}{2}}}}
\\
&= \frac{4 u_0}{n\p{\ell-k}\pi}\begin{cases} \frac{k-\ell}{2} & \text{when $k n$ and $\ell n$ is even,} \\ \frac{k}{2}-1 + \frac{1}{n\pi} & \text{when $k n$ is even and $\ell n$ is odd,} \\ 1-\frac{\ell}{2} + \frac{1}{n\pi} & \text{when $k n$ is odd and $\ell n$ is even,} \\ 0 & \text{when $k n$ and $\ell n$ is odd.}\end{cases}
\end{align*}

Since $x\in\mathbb{R}_0^d$ in \eqref{eq_6} leads to $k,\ell\in\kr{0,2}$ (because of $\frac{kd}{2}$ and $\frac{\ell d}{2}$ in the integration interval) and $k\neq \ell$ for $A_n$ to apply to the whole interval of $x$, which means that $kn$ and $\ell n$ is even; hence there are only one possibility of the solution above

\begin{align}
A_n = -\frac{2 u_0}{n\pi} \qquad \text{for } x\in\mathbb{R}_0^d\,.
\label{eq_7}
\end{align}

Therefore the analytical solution for the concentration of neurotransmitters in \eqref{eq_3} is given by

\begin{align}
\forall x \in \mathbb{R}_0^d \forall t \in \mathbb{R}_{0}: u\p{x,t} = u_0\p{1-\frac{x}{d} - \sum_{n=1} \frac{2}{n\pi}\sin\p{n\pi x}\exp\p{-D\p{\frac{n\pi}{d}}^2 t}}\,.
\label{eq_8}
\end{align}

\section{Numerical methods}

\subsection{The $\theta$-rule}

The Taylor expansion is given by

\begin{align}
u\p{x} = \sum_{n=0} \frac{\T{u}{(n)}\p{x_0}}{n!}\p{x-x_0}^n
\label{eq_9}
\end{align}

where $\T{u}{(n)} = \frac{\mathrm{d}^n u}{\mathrm{d}t^n}$ and $x_0$ is a initial value where we step from to $x$ . If we now use the first order approximation

\begin{align*}
u\p{x} \approx u\p{x_0} + \T{u}{(1)}\p{x_0}\p{x-x_0} \,.
\end{align*}

The first order differential equation $\T{u}{(1)}\p{x} = f\p{x}$ is determined when we have the initial condition $u\p{x_0}$, however $\T{u}{(1)}\p{x_0}$ is not an initial condition, and it depends on how we calculate it numerically from the initial condition. Now note that $\T{u}{(1)}\p{x_0}$ is the same for different values of $x$ in the approximation above and lets say that we calculate it as given from the approximation above;

\begin{align}
\T{u}{(1)}\p{x_0} \approx \frac{u\p{x}-u\p{x_0}}{x-x_0}\,.
\label{eq_10}
\end{align}

So now use this in another point $x_{\theta} = \theta x + \p{1-\theta}x_0$ which we also approximate to the first order, and if we use the expression above for $\T{u}{(1)}\p{x_0}$ we get

\begin{align}
u\p{x_\theta} &\approx u\p{x_0} + \T{u}{(1)}\p{x_0}\p{x_{\theta}-x_0} = u\p{x_0} + \theta\,\T{u}{(1)}\p{x_0}\p{x-x_0} 
\nonumber\\
&\approx u\p{x_0} + \frac{u\p{x}-u\p{x_0}}{x-x_0}\theta\p{x-x_0} = \theta u\p{x} + \p{1-\theta}u\p{x_0} \,,
\label{eq_11}
\end{align}

this is known as the $\theta$-rule. The $\theta$-rule can be used to approximate the solution of the following first order differential equation

\begin{align}
\T{u}{(1)}\p{x} = f\p{u\p{x}} \,,
\label{eq_12}
\end{align}

where we use \eqref{eq_10} to approximate the expression $\T{u}{(1)}\p{x}$ and given an even better or worse approximation to the solution $u\p{x}$ by approximating $f\p{x}\approx f\p{x_{\theta}}$;

\begin{align*}
\frac{u\p{x}-u\p{x_0}}{x-x_0} \approx f\p{u\p{x_{\theta}}} = f\p{\theta u\p{x} + \p{1-\theta}u\p{x_0}}\,,
\end{align*}

which discretize to

\begin{align}
\frac{u_{i+1}-u_i}{x_{i+1}-x_i} = f\p{\theta u_{i+1} + \p{1-\theta}u_i} \qquad \text{where } i\in\mathbb{N}_0 \text{ and $u_0$ is an initial contidion.} 
\label{eq_13}
\end{align}

We can find the the next step in the numerical solution to \eqref{eq_12} by solving this difference equation with regard to $u_{i+1}$. Note the above discretization is known as Forward Euler scheme (Explicit) when $\theta = 0$, Backward Euler scheme (Implicit) when $\theta = 1$ and Crank-Nicolson scheme when $\theta = \frac{1}{2}$. \linebreak

The truncation error of the Forward and Backward Euler scheme can be found by an alternative derivation, where expand the Taylor series in \eqref{eq_9} around the point $x_0\pm\Delta x$ accordingly;

\begin{align}
u\p{x_0\pm\Delta x} = \sum_{n=0} \frac{\T{u}{(n)}\p{x_0}}{n!}\p{\pm\Delta x}^n\,,
\label{eq_14}
\end{align}

and solve it with regard to $\T{u}{(1)}\p{x_0}$

\begin{align*}
\T{u}{(1)}\p{x_0} = \frac{u\p{x_0\pm\Delta x}-u\p{x_0}}{\Delta x} + \mathcal{O}\p{\Delta x} \,,
\end{align*}

which means that we have a local truncation error of $\mathcal{O}\p{\Delta x}$ with the Forward and Backward Euler scheme. However the Crank-Nicolson scheme can be found by subtraction the Taylor expansion above for the two points

\begin{align*}
u\p{x_0+\Delta x} - u\p{x_0-\Delta x} = 2\sum_{n=1} \frac{\T{u}{\vspace{0.1cm}(2n-1)}\p{x_0}}{\p{2n-1}!}\Delta x^{2n-1}\,,
\end{align*}

and solve it with regard to $\T{u}{(1)}\p{x_0}$

\begin{align*}
\T{u}{(1)}\p{x_0} = \frac{u\p{x_0+\Delta x} - u\p{x_0-\Delta x}}{2\Delta x} + \mathcal{O}\p{\Delta x^2}\,,
\end{align*}

which means that we have a local truncation error of $\mathcal{O}\p{\Delta x^2}$. With the $\theta$-rule we can get even better or worse truncation error, because we can change the $\theta$ value to change the approximation.

\subsection{Second order derivative}

We approximated the first order derivative in \eqref{eq_10}, but we need to approximate the second order derivative to be able to solve the diffusion in \eqref{eq_1}.

Now adding the two expansions in \eqref{eq_14}

\begin{align*}
u\p{x_0+\Delta x} + u\p{x_0-\Delta x} = 2\sum_{n=0} \frac{\T{u}{(2n)}\p{x_0}}{(2!)}\Delta x^{2n} = 2 u\p{x_0} + \T{u}{(2)}\p{x_0}\Delta x^2 + 2\sum_{n=2} \frac{\T{u}{(2n)}\p{x_0}}{(2!)}\Delta x^{2n}\,,
\end{align*}

and solve it with

\begin{align*}
\T{u}{(2)}\p{x_0} &= \frac{u\p{x_0+\Delta x} - 2 u\p{x_0} + u\p{x_0-\Delta x}}{\Delta x^2} - 2\sum_{n=2} \frac{\T{u}{(2n)}\p{x_0}}{(2!)}\Delta x^{2(n-1)}
\\ &=  \frac{u\p{x_0+\Delta x} - 2 u\p{x_0} + u\p{x_0-\Delta x}}{\Delta x^2} + \mathcal{O}\p{\Delta x^2}\,,
\end{align*}

So the second order derivative can be approximated with

\begin{align}
\T{u}{(2)}\p{x_0} \approx   \frac{u\p{x_0+\Delta x} - 2 u\p{x_0} + u\p{x_0-\Delta x}}{\Delta x^2}
\label{eq_15}
\end{align}

with the local truncation error $\mathcal{O}\p{\Delta x^2}$.

\subsection{The heat equation}

We want to discretize the dimensionless heat equation from \eqref{eq_1}, where we use $D=1$, $u_0=1$ and $d=1$, 

\begin{align*}
\frac{\partial u\p{x,t}}{\partial t} = \frac{\partial^2 u\p{x,t}}{\partial x^2}\,,
\end{align*}

to numerically solve diffusion of neurotransmitters. First we do the $\theta$-rule discretization in \eqref{eq_13} 

\begin{align*}
\frac{u_{i(j+1)}-u_{ij}}{\Delta t} = \frac{\partial^2 u_{i(j+\theta)}}{\partial x^2} = \frac{\partial^2 \p{\theta u_{i(j+1)}+\p{1-\theta}u_{ij}}}{\partial x^2} = \theta \frac{\partial u_{i(j+1)}}{\partial x^2} + \p{1-\theta}\frac{\partial u_{ij}}{\partial x^2}
\end{align*}

and then implement discretization of the second order in \eqref{eq_15}

\begin{align}
\frac{u_{i(j+1)}-u_{ij}}{\Delta t} = \frac{\theta}{\Delta x^2}\p{u_{(i+1)(j+1)} - 2 u_{i(j+1)} + u_{(i-1)(j+1)}} + \frac{1-\theta}{\Delta x^2}\p{u_{(i+1)j} - 2u_{ij} +u_{(i-1)j}} \,,
\label{eq_16}
\end{align}

where index $i$ is stepping of $x$ and $j$ is stepping of $t$. And from the discussion before we have the local truncation error $\mathcal{O}\p{\Delta t}$ and $\mathcal{O}\p{\Delta x^2}$ for the Forward and Backward Euler scheme, and $\mathcal{O}\p{\Delta t^2}$ and $\mathcal{O}\p{\Delta x^2}$ for Crank-Nicoloson scheme. \linebreak

The dimensionless initial condition from \eqref{eq_4} gives us

\begin{align*}
u_{i0} = \begin{cases} 1 & \text{when $i=0$,} \\ 0 & \text{elsewhere.} \end{cases} 
\end{align*}

The Forward Euler scheme is when $\theta = 0$ which gives explicit solution of \eqref{eq_16}

\begin{align}
u_{i(j+1)} = u_{ij} + \alpha\p{u_{(i+1)j} - 2 u_{ij} + u_{(i-1)j}} \,,
\label{eq_17}
\end{align} 

where

\begin{align*}
\alpha = \frac{\Delta t}{\Delta x^2} \qquad \text{and} \qquad n = \frac{1}{\Delta x} \,.
\end{align*}

When we step forward in time with $j$ Implicitly $\theta > 0$, we use \eqref{eq_16} to find the next time step $j+1$ which is unknown and collect the unknowns on one side of the equation;

\begin{align*}
\begin{array}{l} u_{0(j+1)} \\ -u_{(i-1)(j+1)} +\p{2+\frac{1}{\alpha \theta}}u_{i(j+1)} - u_{(i+1)(j+1)} \\ u_{n(j+1)}\end{array} = \begin{array}{ll} 1 & \text{boundary condition,} \\ \p{\frac{1}{\theta}-1}\p{u_{(i+1)j} - 2u_{ij} + u_{(i-1)j}} + \frac{u_{ij}}{\alpha\theta} & \text{when $i\in\mathbb{N}_1^{n-1}$,}\\ 0  & \text{boundary condition.}\end{array}
\end{align*} 

We can rewrite further to omit the boundary condition

\begin{align}
\begin{array}{l} \p{2+\frac{1}{\alpha \theta}}u_{1(j+1)} - u_{2(j+1)} \\  -u_{(i-1)(j+1)} +\p{2+\frac{1}{\alpha \theta}}u_{i(j+1)} - u_{(i+1)(j+1)} \\ -u_{(n-2)(j+1)} +\p{2+\frac{1}{\alpha \theta}}u_{(n-1)(j+1)} \end{array} = \begin{array}{ll} \p{\frac{1}{\theta}-1}\p{u_{2j} - 2u_{1j} + u_{0j}} + \frac{u_{1j}}{\alpha\theta} + u_{0j} & \text{when $i=1$,} \\ \p{\frac{1}{\theta}-1}\p{u_{(i+1)j} - 2u_{ij} + u_{(i-1)j}} + \frac{u_{ij}}{\alpha\theta} & \text{when $i\in\mathbb{N}_2^{n-2}$,}\\ \p{\frac{1}{\theta}-1}\p{u_{nj} - 2u_{(n-1)j} + u_{(n-2)j}} + \frac{u_{(n-1)j}}{\alpha\theta} + u_{nj} & \text{when $i=n-1$,}\end{array}
\label{eq_18}
\end{align}

where the left side constructs a tridiagonal matrix with the elements $\p{-1,2+\frac{1}{\alpha\theta},-1}$ when we extract the unknowns to a vector. Note that the boundary conditions are added in $i=1$ and $i=n-1$  in addition to the expression for $i\in\mathbb{N}_2^{n-2}$, which we get by taking the Gaussian elimination on row $i=1$ and $i=n-1$ with regard to the row $i=0$ and $i=n$ accordingly.

\subsubsection{Tridiagonal matrix}

We found that heat equation in \eqref{eq_1} can be solved by solving a matrix problem on the form

\begin{align*}
\textbf{A}\textbf{u} = \textbf{v}
\end{align*} 

where $\textbf{u}$ is the unknowns that we want to find, $\textbf{v}$ are given values and $\textbf{A}$ is a tridiagonal matrix on the form $\p{-1,a,-1}$ which we represent with the elements

\begin{align*}
a_{ij} = \begin{cases} a & \text{when $i=j$.} \\ -1 &\text{when $j\in\kr{i-1,i+1}$ and $i,j\in\mathbb{N}_1^{n-1}$,} \\ 0 & \text{otherwise.}\end{cases}
\end{align*}

We then use Gaussian elimination to reduce the tridiagonal matrix $\textbf{A}$ to a upper triangular matrix $\check{\textbf{A}}$ in the following way

\begin{align*}
\check{\textbf{A}}\textbf{u} = \check{\textbf{v}}\,.
\end{align*} 

To find the upper tridiagonal matrix we need to eliminate the elements $a_{i(i-1)}$ so $\check{a}_{i(i-1)} = 0$, which means that we need to multiply the values (without changing them) in row $i-1$ with 

\begin{align*}
\frac{a_{i(i-1)}}{\check{a}_{(i-1)(i-1)}} = -\frac{1}{\check{a}_{(i-1)(i-1)}} 
\end{align*}

and subtract them with the values in row $i$, which leads obviously to

\begin{align*}
\check{a}_{i(i-1)} = a_{i(i-1)} - \frac{a_{i(i-1)}}{\check{a}_{(i-1)(i-1)}} \check{a}_{(i-1)(i-1)} = 0\,.
\end{align*}

We can also show that off-tridiagonal elements in the upper tridiagonal remains zero under this Gaussian elimination

\begin{align*}
\check{a}_{ij} = a_{ij} - \frac{a_{i(i-1)}}{\check{a}_{(i-1)(i-1)}} \check{a}_{(i-1)j} = a_{ij} = 0 \quad \text{when $i\in\mathbb{N}_2^{n-3}$ and $j\in\mathbb{N}_{i+2}^{n-1}$, because $\check{a}_{1j}=a_{1j} = 0$.}
\end{align*}

and similarly that the off-tridiagonal elements in the lower tridiagonal also remains zero

\begin{align*}
\check{a}_{ij} = a_{ij} - \frac{a_{i(i-1)}}{\check{a}_{(i-1)(i-1)}} \check{a}_{(i-1)j} = a_{ij} = 0 \quad \text{when $i\in\mathbb{N}_3^{n-1}$ and $j\in\mathbb{N}_{1}^{i-2}$, because $\check{a}_{i(i-1)}=a_{i(i-1)} = 0$.}
\end{align*}

We can also show that elements $a_{i(i+1)}=-1$ also remains unchanged under this Gaussian elimination

\begin{align*}
\check{a}_{i(i+1)} = a_{i(i+1)} - \frac{a_{i(i-1)}}{\check{a}_{(i-1)(i-1)}} \check{a}_{(i-1)(i+1)} = a_{i(i+1)} = -1 \,.
\end{align*}

This results in that the diagonal elements are then given by

\begin{align}
\check{a}_{ii} = a_{ii} - \frac{a_{i(i-1)}}{\check{a}_{(i-1)(i-1)}} \check{a}_{(i-1)i} = a_{ii} - \frac{1}{\check{a}_{(i-1)(i-1)}} \quad \text{when }\check{a}_{11} = a_{11} = a\,.
\label{eq_19}
\end{align} 

We have now calculated all the elements in the upper triangular matrix $\check{\textbf{A}}$, and we can summarize the the elements as

\begin{align*}
\check{a}_{ij} &= \begin{cases} a_{11} & \text{when $i=j=1$} \\ a_{ii} - \frac{1}{\check{a}_{(i-1)(i-1)}} & \text{when $i=j$ and  $i\in\mathbb{N}_2^{n-1}$} \\ -1 & \text{when $j=i+1$ and $i,j\in\mathbb{N}_1^{n-1}$} \\ 0 & \text{otherwise\,.}\end{cases}
\end{align*}

We also need to do the Gaussian elimination on the vector $\textbf{v}$ as well, which leads to

\begin{align}
\check{v}_i &= \begin{cases} v_1 & \text{when $i=1$} \\ v_i + \frac{\check{v}_{i-1}}{\check{a}_{(i-1)(i-1)}} & \text{when $i\in\mathbb{N}_2^{n-1}$}\,. \end{cases}
\label{eq_20}
\end{align}

We now want to eliminate the upper off-diagonal elements $\check{\textbf{A}}$ with Gaussian elimination so we get at diagonal matrix $\hat{\textbf{A}}$ in the following way

\begin{align*}
\hat{\textbf{A}}\textbf{u} = \hat{\textbf{v}}.
\end{align*}

To find the diagonal matrix we need to eliminate the elements $\check{a}_{i(i+1)}$ so $\hat{a}_{i(i+1)}=0$, which means that we need to multiply the values (without changing them) in row $i+1$ with

\begin{align*}
\frac{\check{a}_{i(i+1)}}{\hat{a}_{(i+1)(i+1)}} = - \frac{1}{\hat{a}_{(i+1)(i+1)}}
\end{align*} 

and subtract them with the values in row $i$, which leads obviously to

\begin{align*}
\hat{a}_{i(i+1)} = \check{a}_{i(i+1)} - \frac{\check{a}_{i(i+1)}}{\hat{a}_{(i+1)(i+1)}}\hat{a}_{(i+1)(i+1)} = 0\,.
\end{align*}

We can also show that off-tridiagonal elements in the upper tridiagonal remains zero under this Gaussian elimination

\begin{align*}
\hat{a}_{ij} = \check{a}_{ij} - \frac{\check{a}_{i(i+1)}}{\hat{a}_{(i+1)(i+1)}} \check{a}_{(i+1)j} = \check{a}_{ij} = 0 \quad \text{when $i\in\mathbb{N}_1^{n-3}$ and $j\in\mathbb{N}_{i+2}^{n-1}$, because $\hat{a}_{i(i+1)}=\check{a}_{i(i+1)} = 0$,}
\end{align*}

and similarly that the off-diagonal elements in the lower tridiagonal also remains zero

\begin{align*}
\hat{a}_{ij} = \check{a}_{ij} - \frac{\check{a}_{i(i+1)}}{\hat{a}_{(i+1)(i+1)}} \hat{a}_{(i+1)j} = a_{ij} = 0 \quad \text{when $i\in\mathbb{N}_2^{n-2}$ and $j\in\mathbb{N}_{1}^{i-1}$, because $\check{a}_{i(i+1)}=a_{i(i+1)} = 0$.}
\end{align*}

This results unchanged diagonal elements

\begin{align*}
\hat{a}_{ii} = \check{a}_{ii} - \frac{\check{a}_{i(i+1)}}{\hat{a}_{(i+1)(i+1)}}\check{a}_{(i+1)i} = \check{a}_{ii} = a_{ii} - \frac{1}{\check{a}_{(i-1)(i-1)}} \qquad \text{where }\check{a}_{11} = a_{11}\,.
\end{align*}

We have now calculated all the elements in the diagonal matrix $\hat{\textbf{A}}$, and we can summarize the elements as

\begin{align*}
\hat{a}_{ij} = \begin{cases} \check{a}_{ii}  & \text{when } i=j \text{ and } i\in\mathbb{N}_{1}^{n-1} \\ 0 & \text{otherwise.} \end{cases}
\end{align*}

We also need to do the Gaussian elimination on the vector $\textbf{v}$ as well, which leads to

\begin{align*}
\hat{v}_i &= \begin{cases} \check{v}_{n-1} & \text{when $i=n-1$} \\ \check{v}_i + \frac{\hat{v}_{i+1}}{\check{a}_{(i+1)(i+1)}} & \text{when $i\in\mathbb{N}_1^{n-2}$}\,. \end{cases}
\end{align*}

To find the unknowns in $\textbf{u}$ we need to make the diagonal matrix $\hat{\textbf{A}}$ to an identity matrix $\textbf{I}$ such that $\bar{\textbf{v}}$

\begin{align*}
\textbf{I}\textbf{u} = \bar{\textbf{v}}\,,
\end{align*}

which we achieve by dividing each row with the diagonal element $a_{ii}$. And the solution is given by

\begin{align}
u_i = \bar{v}_i = \frac{\hat{v}_i}{\check{a}_{ii}} \,.
\label{eq_21}
\end{align}

But we now see that when we calculate $\check{v}_i$ actually uses the previous solution $u_{i+1}$, and we can rewrite $\check{v}_i$ to

\begin{align}
\hat{v}_i &= \begin{cases} \check{v}_{n-1} & \text{when $i=n-1$} \\ \check{v}_i + u_{i+1} & \text{when $i\in\mathbb{N}_1^{n-2}$}\,. \end{cases}
\label{eq_22}
\end{align}

So what we need to calculate the solution in \eqref{eq_21} is first to calculate $\check{a}_{ii}$ in \eqref{eq_19}, then $\check{v}_i$ in \eqref{eq_20} followed by $\hat{v}_i$ \eqref{eq_22}. The calculation of $\check{a}_{ii}$ coefficient in \eqref{eq_19} is actually independent of the input values $v_i$, and can therefore be precalculated. Which means that we need 2N FLOPS to calculate $\check{v}_i$ in \eqref{eq_20}, 1N FLOP to calculate $u_i$ in \eqref{eq_21} and 1N FLOP to calculate $\hat{v}_i$ in \eqref{eq_22}. Hence we need 4N FLOPS to find the solution $u_i$.

\section{Implementation}

\section{Result}

Order of relative error of $\varv$ to $u$ is calculated by

\begin{align*}
\epsilon = \log_{10}\abs{\frac{\varv - u}{u}} \,.
\end{align*}

\begin{tabell}{|r|c|c|c|}{\small}{\input{build-project4-Desktop-Debug/time_header.dat}}{\input{build-project4-Desktop-Debug/time_forward_euler.dat}}{Calculation time for the Forward Euler scheme in seconds.}{tab_euler_forward}
\end{tabell}

\begin{tabell}{|r|c|c|c|}{\small}{\input{build-project4-Desktop-Debug/time_header.dat}}{\input{build-project4-Desktop-Debug/time_backward_euler.dat}}{Calculation time for the Backward Euler scheme in seconds.}{tab_euler_backward}
\end{tabell}

\begin{tabell}{|r|c|c|c|}{\small}{\input{build-project4-Desktop-Debug/time_header.dat}}{\input{build-project4-Desktop-Debug/time_crank_nicoloson.dat}}{Calculation time for the Crank-Nicoloson scheme in seconds.}{tab_crank_nicoloson}
\end{tabell}

\begin{tabell}{|r|c|c|c|}{\small}{\input{build-project4-Desktop-Debug/time_header.dat}}{\input{build-project4-Desktop-Debug/time_forward_euler_error.dat}}{Order of relative error of Forward Euler scheme to the exact solution in \eqref{eq_8}, for $t=0.01$, $u_0 = 1$, $d = 1$ and $D = 1$.}{tab_euler_forward_error}
\end{tabell}

\begin{tabell}{|r|c|c|c|}{\small}{\input{build-project4-Desktop-Debug/time_header.dat}}{\input{build-project4-Desktop-Debug/time_backward_euler_error.dat}}{Order of relative error of Backward Euler scheme to the exact solution in \eqref{eq_8}, for $t=0.01$, $u_0 = 1$, $d = 1$ and $D = 1$.}{tab_euler_backward_error}
\end{tabell}

\begin{tabell}{|r|c|c|c|}{\small}{\input{build-project4-Desktop-Debug/time_header.dat}}{\input{build-project4-Desktop-Debug/time_crank_nicoloson_error.dat}}{Order of relative error of Crank-Nicoloson scheme to the exact solution in \eqref{eq_8}, for $t=0.01$, $u_0 = 1$, $d = 1$ and $D = 1$.}{tab_crank_nicoloson_error}
\end{tabell}

\figur{0.8}{nx10_a0.49_t0.01.eps}{Exact solution calculated \eqref{eq_8}, FE is Forward Euler, BE is Back Euler and CN is Crank Nicholoson scheme. The following values is set as $u_0 = 1$, $d = 1$ and $D = 1$.}{fig1} 

\figur{0.8}{nx10_a0.1_t0.01.eps}{Exact solution calculated \eqref{eq_8}, FE is Forward Euler, BE is Back Euler and CN is Crank Nicholoson scheme. The following values is set as $u_0 = 1$, $d = 1$ and $D = 1$.}{fig2} 

\figur{0.8}{nx100_a0.49_t0.01.eps}{Exact solution calculated \eqref{eq_8}, FE is Forward Euler, BE is Back Euler and CN is Crank Nicholoson scheme. The following values is set as $u_0 = 1$, $d = 1$ and $D = 1$.}{fig3} 

\figur{0.8}{nx100_a0.51_t0.01.eps}{Exact solution calculated \eqref{eq_8}, FE is Forward Euler, BE is Back Euler and CN is Crank Nicholoson scheme. The following values is set as $u_0 = 1$, $d = 1$ and $D = 1$.}{fig4} 

\figur{0.8}{nx10_a0.49_t1.eps}{Exact solution calculated \eqref{eq_8}, FE is Forward Euler, BE is Back Euler and CN is Crank Nicholoson scheme. The following values is set as $u_0 = 1$, $d = 1$ and $D = 1$.}{fig5} 

\section{Attachments}

The files produced in working with this project can be found at \linebreak

The source files developed are


\section{Resources}

\begin{enumerate}
\item{\href{http://qt-project.org/downloads}{QT Creator 5.3.1 with C11}}
\item{\href{https://www.eclipse.org/downloads/}{Eclipse Standard/SDK  - Version: Luna Release (4.4.0) with PyDev for Python}}
\item{\href{http://www.ubuntu.com/download/desktop}{Ubuntu 14.04.1 LTS}}
\item{\href{http://shop.lenovo.com/no/en/laptops/thinkpad/w-series/w540/#tab-reseller}{ThinkPad W540 P/N: 20BG0042MN with 32 GB RAM}}
\end{enumerate}

\begin{thebibliography}{1}
\bibitem{project4}{\href{mailto:morten.hjorth-jensen@fys.uio.no}{Morten Hjorth-Jensen}, \href{http://www.uio.no/studier/emner/matnat/fys/FYS3150/h14/undervisningsmateriale/projects/project-4-deadline-november-10/project4\_2014.pdf}{\emph{FYS4150 - Project 4}} - \emph{Diffusion of neurotransmitters in the synaptic
cleft}, \href{http://www.uio.no}{University of Oslo}, 2014} 
\bibitem{lecture}{\href{mailto:morten.hjorth-jensen@fys.uio.no}{Morten Hjorth-Jensen}, \href{http://www.uio.no/studier/emner/matnat/fys/FYS3150/h14/undervisningsmateriale/Lecture\%20Notes/lecture2014.pdf}{\emph{Computational Physics - Lecture Notes Fall 2014}}, \href{http://www.uio.no}{University of Oslo}, 2014} 
\bibitem{Diffusion}\href{http://en.wikipedia.org/wiki/Diffusion\_equation}{http://en.wikipedia.org/wiki/Diffusion\_equation}
\bibitem{Heat}\href{http://en.wikipedia.org/wiki/Heat\_equation}{http://en.wikipedia.org/wiki/Heat\_equation}
\bibitem{Dirichlet}\href{http://en.wikipedia.org/wiki/Dirichlet\_boundary\_condition}{http://en.wikipedia.org/wiki/Dirichlet\_boundary\_condition}
\bibitem{IntegrationByParts}\href{http://en.wikipedia.org/wiki/Integration\_by\_parts}{http://en.wikipedia.org/wiki/Integration\_by\_parts}
\bibitem{Taylor}\href{http://en.wikipedia.org/wiki/Taylor\_series}{http://en.wikipedia.org/wiki/Taylor\_series}
\bibitem{Forward_Euler}\href{http://en.wikipedia.org/wiki/Euler\_method}{http://en.wikipedia.org/wiki/Euler\_method}
\bibitem{Backward_Euler}\href{http://en.wikipedia.org/wiki/Backward\_Euler\_method}{http://en.wikipedia.org/wiki/Backward\_Euler\_method}
\bibitem{Crank-Nicolson}\href{http://en.wikipedia.org/wiki/Crank\%E2\%80\%93Nicolson\_method}{http://en.wikipedia.org/wiki/Crank\%E2\%80\%93Nicolson\_method}
\bibitem{Gaussian}\href{http://en.wikipedia.org/wiki/Gaussian\_elimination}{http://en.wikipedia.org/wiki/Gaussian\_elimination}
\end{thebibliography}

\end{flushleft}
\end{document}
