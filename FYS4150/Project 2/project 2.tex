\documentclass[11pt,english,a4paper]{article}
\usepackage[latin1]{inputenc}
\usepackage[T1]{fontenc}
\usepackage[titles]{tocloft}
\usepackage[auth-sc]{authblk}
\usepackage[usenames]{color}
\usepackage[document]{ragged2e}
\usepackage{babel}
\usepackage[dvips]{epsfig}
\usepackage{graphicx}
\usepackage{amsmath}
\usepackage{amsfonts}
\usepackage{amssymb}
\usepackage{accents}
\usepackage{verbatim}
\usepackage[nodayofweek]{datetime}
\usepackage{epic,eepic}
\usepackage[mathscr]{euscript}
\usepackage{gastex}
\usepackage{supertabular}
\usepackage{nextpage}
\usepackage{listings}
\usepackage{color}
\usepackage{url}
\usepackage{ifthen,calc}
\usepackage{varioref, ifthen}
\usepackage{stmaryrd}
\usepackage{wasysym}
\usepackage{array}
\usepackage{booktabs}
\usepackage{gensymb}
\usepackage{ifsym}
\usepackage{trfsigns}
\usepackage{txfonts}
\usepackage{chemarrow}
\usepackage{textcomp}
\usepackage{cancel}
\usepackage{extarrows}
\usepackage{rotating}
\usepackage{varioref}
\usepackage{slashed}
\usepackage{setspace}
\PassOptionsToPackage{hyphens}{url}\usepackage[dvips,letterpaper,linktocpage,hidelinks=true,colorlinks=true]{hyperref}
\usepackage{multirow}
\usepackage{fancyhdr}
\usepackage{pstricks}
\usepackage{pst-node}
\usepackage{pst-blur}
\usepackage{xypic}
\input xy
\xyoption{all}

% Tekst kommandoer
\newcommand{\sitat}[1]{\textquotedblleft{#1}\textquotedblright} 
\newcommand{\anf}[1]{\textquotedbl{#1}\textquotedbl}

% Matematikk kommandoer
\newcommand{\ans}[1]{\underline{\underline{\,#1\,}}}
\newcommand{\p}[1]{\left({#1}\right)}
\newcommand{\kl}[1]{\left[{#1}\right]}
\newcommand{\bag}[1]{\Lbag{#1}\Rbag}
\newcommand{\Det}[1]{\left|{#1}\right|}
\newcommand{\Norm}[1]{\left\Vert{#1}\right\Vert}
\newcommand{\abs}[1]{\left\vert{\,#1\,}\right\vert}
\newcommand{\ceil}[1]{\left\lceil{\,#1\,}\right\rceil}
\newcommand{\floor}[1]{\left\lfloor{\,#1\,}\right\rfloor}
\newcommand{\kr}[1]{\left\lbrace{#1}\right\rbrace}
\let\vec\relax
\newcommand{\vec}[1]{\accentset{\rightarrow}{#1}}
\let\Vec\relax
\newcommand{\Vec}[2]{\accentset{#2\rightarrow}{#1}}
\newcommand{\T}[2]{\accentset{#2}{#1}}
\newcommand{\U}[2]{\underaccent{#2}{#1}}
\newcommand{\TU}[3]{\underaccent{#3}{\accentset{#2}{#1}}}
\newcommand{\grad}{\vec{\nabla}}
\newcommand{\Grad}[1]{\Vec{\nabla}{#1}}
\newcommand{\fvec}[1]{\vec{\mathrm{#1}}}
\newcommand{\vvec}[1]{\accentset{\rightrightarrows}{#1}}
\newcommand{\vek}[1]{\mathbf{#1}}
\newcommand{\Vek}[1]{\vec{\vek{#1}}}
\newcommand{\ivec}{\hat{\boldsymbol{\imath}}}
\newcommand{\jvec}{\hat{\boldsymbol{\jmath}}}
\newcommand{\kvec}{\hat{\boldsymbol{k}}}
\newcommand{\rhovec}{\hat{\boldsymbol{\rho}}}
\newcommand{\thetavec}{\hat{\boldsymbol{\theta}}}
\newcommand{\phivec}{\hat{\boldsymbol{\phi}}}
\newcommand{\unit}[1]{\,[\,\mathrm{#1}\,]}
\newcommand{\pow}[2]{{#1}^{#2}}
\newcommand{\ppow}[2]{\p{#1}^{#2}}
\newcommand{\abspow}[2]{\abs{#1}^{#2}}
\newcommand{\klpow}[2]{\pow{\kl{#1}}{#2}}
\newcommand{\krpow}[2]{\pow{\kr{#1}}{#2}}
\newcommand{\E}[1]{\times \pow{10}{#1}}
\newcommand{\Exp}[1]{\pow{\e}{#1}}
\newcommand{\snitt}[1]{\left\langle{\,#1\,}\right\rangle}
\newcommand{\D}[3]{\frac{\mathrm{d}^{#3}{#1}}{\mathrm{d}{#2}^{#3}}}
\let\P\relax
\newcommand{\P}[3]{\frac{\partial^{#3}{#1}}{\partial{#2}^{#3}}}
\newcommand{\I}[4]{\int_{\,#1}^{\,#2} {#3} \, \mathrm{d}{#4}}
\newcommand{\II}[4]{\int_{\,#1}^{\,#2} {#3} \, \boldsymbol{\mathrm{d}}{#4}}
\newcommand{\bfrac}[2]{{#1}\,/\,{#2}}
\newcommand{\mberegn}[1]{\qquad\p{\,#1\,}}
\newcommand{\bra}[1]{\langle\,#1\,\vert}
\newcommand{\ket}[1]{\,\vert\,#1\,\rangle}
\newcommand{\braket}[2]{\langle\,#1\,\vert\,#2\,\rangle}
\newcommand{\ketbra}[2]{\vert\,#1\,\rangle\,\langle\,#2\,\vert}
\newcommand{\Sim}[1]{\begin{matrix}\scriptsize{\text{#1}}\normalsize\\\sim\\\end{matrix}}
\newcommand{\norm}[1]{\Vert\,#1\,\Vert}
\newcommand{\Op}[1]{\hat{\mathrm{#1}}}
\newcommand{\OP}[1]{\hat{\mathbf{#1}}}
\newcommand{\set}[2]{\left. #1 \right\vert_{#2}}
\newcommand{\nlessdot}{\,\slashed{\lessdot}\,}
\newcommand{\verden}{\textbf{?}}
\newcommand{\fantasi}{\reflectbox{\verden}}
\newcommand{\hyperverden}{\reflectbox{\rotatebox[origin=lc]{180}{\verden}}}
\newcommand{\jeg}{\textcircled{\scriptsize{I}}}
\newcommand{\rightlefteqEA}{ \,\TU{\rightleftharpoons}{\hspace{-4pt}\exists}{\hspace{4pt}\forall}\,}
\newcommand{\nmapsto}{\displaystyle\mapsto\hspace{-11.5pt}\arrownot\hspace{10.5pt}}
\newcommand{\nmapsfrom}{\displaystyle\mapsfrom\hspace{-10pt}\arrownot\hspace{9pt}}
\newcommand{\maptofrom}{\displaystyle\leftrightarrow\hspace{-7pt}\mmapstochar\hspace{6pt}}
\newcommand{\nmaptofrom}{\displaystyle\leftrightarrow\hspace{-7pt}\mmapstochar\hspace{-5pt}\Arrownot\hspace{11pt}}
\newcommand{\mapstofrom}{\begin{array}{c}\displaystyle\mapsto\\[-9pt]\displaystyle\mapsfrom\end{array}}
\newcommand{\mapsfromto}{\begin{array}{c}\displaystyle\mapsfrom\\[-9pt]\displaystyle\mapsto\end{array}}
\declareslashed{}{$$ \mbox{\tiny $\boldsymbol{/}$ } $$}{0.01}{0}{\mapsto}
\declareslashed{}{$$ \mbox{\tiny $\boldsymbol{/}$ } $$}{0.13}{0}{\mapsfrom}
\newcommand{\nmapstofrom}{\begin{array}{c}\displaystyle\slashed\mapsto\\[-9pt]\displaystyle\slashed\mapsfrom\end{array}}
\newcommand{\rightleftarrowtail}{\begin{array}{c}\displaystyle\rightarrowtail\\[-9pt]\displaystyle\leftarrowtail\end{array}}
\newcommand{\leftrightarrowtail}{\begin{array}{c}\displaystyle\leftarrowtail\\[-9pt]\displaystyle\rightarrowtail\end{array}}
\newcommand{\rightleftsquigarrows}{\begin{array}{c}\displaystyle\rightsquigarrow\\[-9pt]\displaystyle\leftsquigarrow\end{array}}
\newcommand{\leftrightsquigarrows}{\begin{array}{c}\displaystyle\leftsquigarrow\\[-9pt]\displaystyle\rightsquigarrow\end{array}}
\newcommand{\leftrightdiamondarrow}{\displaystyle\leftrightarrow\hspace{-11.1pt}\diamond\hspace{5pt}}
\newcommand{\tworightarrowsTF}[2]{
	\mathrel{-\hspace{-0.3mm}\vcenter{
		\xymatrix@!=1pc@*[r]@C=0ex@R=2ex@M=0ex@W=0ex
		{
			& & #1 \\
			& \ar@{<->}'[l]^{\mathtt{F}}'[lu]'[lur]^{\mathtt{T}}  &	#2}}}}
%\newcommand{\paradoks}{\mathtt{P}}
%\newcommand{\toparadoks}{\xrightarrow{\paradoks}}
\newcommand{\Set}[1]{\boldsymbol{\mathrm{#1}}}
\newcommand{\eq}[1]{$\begin{matrix}{#1}\end{matrix}$}
\newcommand{\upp}{\hspace{0.2ex}{^{\wedge}}\hspace{0.2ex}}
\newcommand{\asinh}{\mathrm{asinh}}
\newcommand{\sgn}{\mathrm{sgn}}
\newcommand{\Forall}{\mathop{\vphantom{\sum}\mathchoice
  {\vcenter{\hbox{\huge $\forall$}}}
  {\vcenter{\hbox{\Large $\forall$}}}{\forall}{\forall}}\displaylimits}
\newcommand{\Recursion}{\mathop{\vphantom{\sum}\mathchoice
  {\vcenter{\hbox{\huge $\Omega$}}}
  {\vcenter{\hbox{\Large $\Omega$}}}{\Omega}{\Omega}}\displaylimits}
\newcommand{\SW}{\mathop{\vphantom{\sum}\mathchoice
  {\vcenter{\hbox{\huge $\Xi$}}}
  {\vcenter{\hbox{\Large $\Xi$}}}{\Xi}{\Xi}}\displaylimits}
\newcommand{\Rightleftarrowtail}[2]{\T{\T{\mathop{\rightleftarrowtail}}{\vspace{-5pt}#1}}{\vspace{-24pt}#2}}
\newcommand{\Leftrightarrowtail}[2]{\T{\T{\mathop{\leftrightarrowtail}}{\vspace{-5pt}#1}}{\vspace{-24pt}#2}}

\def\kvadratsum{\,\textcircled{$\mathsf{c}$}\,}

% Elektromagnetisme kommandoer
\def\ems{\xi}
\newcommand{\RaisingEdge}{\textifsym{L|H}}

% Astrofysikk kommandoer
\def\Earth{\varoplus}
\def\Sun{\astrosun}

% Kjemi
\newcommand{\nuklide}[2]{$^{#2}\mathrm{#1}$}
\newcommand{\molekyl}[1]{$\mathrm{#1}$}

\delimitershortfall=-1pt

\definecolor{light-gray}{gray}{0.95}
%---------------------------------------------------------------------------------------------------------
\renewcommand\maketitle{
\begin{center}
\LARGE\textsc{\title\ifx \rev \undefined  \else \, - \rev \fi} \normalsize \linebreak

\textsc{\author} \linebreak
\small
\ifx \adress \undefined \else \textsc{\adress} \linebreak \fi
\textsc{Email:} \texttt{\email}\linebreak
\textsc{\date} \linebreak
\rule{\textwidth}{2pt}
\end{center}
}
%---------------------------------------------------------------------------------------------------------
\newcounter{referanse}
\newcommand\reference[2]{\refstepcounter{referanse} \begin{tabular}{p{0.5cm}p{\textwidth - 0.5cm}}[\thereferanse] & {#1} \end{tabular}\label{#2}}
\newcommand\REF[1]{[\ref{#1}]}
%---------------------------------------------------------------------------------------------------------
%part

\makeatletter
\let\OLDpart = \part
\def\part{\@ifstar\unnumberedpart\numberedpart}
\def\numberedpart{\@ifnextchar[%]
  \numberedpartwithtwoarguments\numberedpartwithoneargument}
\def\unnumberedpart{\@ifnextchar[%]
  \unnumberedpartwithtwoarguments\unnumberedpartwithoneargument}
\def\numberedpartwithoneargument#1{\numberedpartwithtwoarguments[#1]{#1}}
\def\unnumberedpartwithoneargument#1{\unnumberedpartwithtwoarguments[#1]{#1}}
\def\numberedpartwithtwoarguments[#1]#2{%
  \ifhmode\par\fi
  \removelastskip
  \vskip 3ex\goodbreak
  \refstepcounter{part}%
  \noindent
  \begingroup
  \leavevmode\huge\bfseries\raggedright
  \thepart\quad 
  #2
  \par
  \endgroup
  \vskip 2ex\nobreak
  \setcounter{section}{0}
  \addcontentsline{toc}{part}{%
    \protect\numberline{\thepart}%
    #1}%
  }
\def\unnumberedpartwithtwoarguments[#1]#2{%
  \ifhmode\par\fi
  \removelastskip
  \vskip 3ex\goodbreak
    \refstepcounter{part}%
  \noindent
  \begingroup
  \leavevmode\huge\bfseries\raggedright
  \leavevmode\huge\bfseries\raggedright
  %\thepart\quad 
  #2
  \par
  \endgroup
  \vskip 2ex\nobreak
  \setcounter{section}{0}
  \addcontentsline{toc}{part}{%
    %\protect\numberline{\thepart}%
    #1}%
}
\makeatother

%---------------------------------------------------------------------------------------------------------

% section

\makeatletter
\let\OLDsection = \section
\def\section{\@ifstar\unnumberedsection\numberedsection}
\def\numberedsection{\@ifnextchar[%]
  \numberedsectionwithtwoarguments\numberedsectionwithoneargument}
\def\unnumberedsection{\@ifnextchar[%]
  \unnumberedsectionwithtwoarguments\unnumberedsectionwithoneargument}
\def\numberedsectionwithoneargument#1{\numberedsectionwithtwoarguments[#1]{#1}}
\def\unnumberedsectionwithoneargument#1{\unnumberedsectionwithtwoarguments[#1]{#1}}
\def\numberedsectionwithtwoarguments[#1]#2{%
  \ifhmode\par\fi
  \removelastskip
  \vskip 3ex\goodbreak
  \refstepcounter{section}%
  \noindent
  \begingroup
  \leavevmode\Large\bfseries\raggedright
  \thesection\quad 
  #2
  \par
  \endgroup
  \vskip 2ex\nobreak
  \addcontentsline{toc}{section}{%
    \protect\numberline{\thesection}%
    #1}%
  }
\def\unnumberedsectionwithtwoarguments[#1]#2{%
  \ifhmode\par\fi
  \removelastskip
  \vskip 3ex\goodbreak
    \refstepcounter{section}%
  \noindent
  \begingroup
  \leavevmode\Large\bfseries\raggedright
  \leavevmode\Large\bfseries\raggedright
  %\thesection\quad 
  #2
  \par
  \endgroup
  \vskip 2ex\nobreak
  \addcontentsline{toc}{section}{%
    %\protect\numberline{\thesection}%
    #1}%
}
\makeatother

%---------------------------------------------------------------------------------------------------------
% subsection

\makeatletter
\let\OLDsubsection = \subsection
\def\subsection{\@ifstar\unnumberedsubsection\numberedsubsection}
\def\numberedsubsection{\@ifnextchar[%]
  \numberedsubsectionwithtwoarguments\numberedsubsectionwithoneargument}
\def\unnumberedsubsection{\@ifnextchar[%]
  \unnumberedsubsectionwithtwoarguments\unnumberedsubsectionwithoneargument}
\def\numberedsubsectionwithoneargument#1{\numberedsubsectionwithtwoarguments[#1]{#1}}
\def\unnumberedsubsectionwithoneargument#1{\unnumberedsubsectionwithtwoarguments[#1]{#1}}
\def\numberedsubsectionwithtwoarguments[#1]#2{%
  \ifhmode\par\fi
  \removelastskip
  \vskip 3ex\goodbreak
  \refstepcounter{subsection}%
  \noindent
  \begingroup
  \leavevmode\large\bfseries\raggedright
  \thesubsection\quad 
  #2
  \par
  \endgroup
  \vskip 2ex\nobreak
  \addcontentsline{toc}{subsection}{%
    \protect\numberline{\thesubsection}%
    #1}%
  }
\def\unnumberedsubsectionwithtwoarguments[#1]#2{%
  \ifhmode\par\fi
  \removelastskip
  \vskip 3ex\goodbreak
    \refstepcounter{subsection}%
  \noindent
  \begingroup
  \leavevmode\large\bfseries\raggedright
  \leavevmode\large\bfseries\raggedright
  %\thesection\quad 
  #2
  \par
  \endgroup
  \vskip 2ex\nobreak
  \addcontentsline{toc}{subsection}{%
    %\protect\numberline{\thesection}%
    #1}%
}
\makeatother

% subsubsection

\makeatletter
\let\OLDsubsubsection = \subsubsection
\def\subsubsection{\@ifstar\unnumberedsubsubsection\numberedsubsubsection}
\def\numberedsubsubsection{\@ifnextchar[%]
  \numberedsubsubsectionwithtwoarguments\numberedsubsubsectionwithoneargument}
\def\unnumberedsubsubsection{\@ifnextchar[%]
  \unnumberedsubsubsectionwithtwoarguments\unnumberedsubsubsectionwithoneargument}
\def\numberedsubsubsectionwithoneargument#1{\numberedsubsubsectionwithtwoarguments[#1]{#1}}
\def\unnumberedsubsubsectionwithoneargument#1{\unnumberedsubsubsectionwithtwoarguments[#1]{#1}}
\def\numberedsubsubsectionwithtwoarguments[#1]#2{%
  \ifhmode\par\fi
  \removelastskip
  \vskip 3ex\goodbreak
  \refstepcounter{subsubsection}%
  \noindent
  \begingroup
  \leavevmode\normalsize\bfseries\raggedright
  \thesubsubsection\quad 
  #2
  \par
  \endgroup
  \vskip 2ex\nobreak
  \addcontentsline{toc}{subsubsection}{%
    \protect\numberline{\thesubsubsection}%
    #1}%
  }
\def\unnumberedsubsubsectionwithtwoarguments[#1]#2{%
  \ifhmode\par\fi
  \removelastskip
  \vskip 3ex\goodbreak
    \refstepcounter{subsubsection}%
  \noindent
  \begingroup
  \leavevmode\normalsize\bfseries\raggedright
  \leavevmode\normalsize\bfseries\raggedright
  %\thesection\quad 
  #2
  \par
  \endgroup
  \vskip 2ex\nobreak
  \addcontentsline{toc}{subsubsection}{%
    %\protect\numberline{\thesection}%
    #1}%
}
\makeatother

% subsubsubsection
\makeatletter
\newcounter{subsubsubsection}[subsubsection]
\def\subsubsubsectionmark#1{}
\def\thesubsubsubsection {\thesubsubsection.\arabic{subsubsubsection}}
\def\subsubsubsection{\@startsection{subsubsubsection}{4}{\z@} {-3.25ex plus -1ex minus -.2ex}{1.5ex plus .2ex}{\normalsize\bf}}
\def\l@subsubsubsection{\@dottedtocline{4}{4.8em}{4.2em}}
\makeatother



% section depth
\setcounter{secnumdepth}{6}
\setcounter{tocdepth}{6}

%---------------------------------------------------------------------------------------------------------
% Appendix

\newcounter{appendix}[part]
\renewcommand \theappendix{\Alph{appendix}}
\newcounter{subappendix}[appendix]
\renewcommand \thesubappendix{\theappendix.\arabic{subappendix}}
\newcounter{subsubappendix}[subappendix]
\renewcommand \thesubsubappendix{\thesubappendix.\arabic{subsubappendix}}

\newcommand{\Appendix}[1]
{
	\ifthenelse{\value{appendix} = 0}
	{
		\renewcommand \thesection{\theappendix}
		\renewcommand \thesubsection{\theappendix.\arabic{subsection}}
		\renewcommand \thesecsetning{\rev.\theappendix.\arabic{secsetning}}
		\renewcommand \thesubsetning{\rev.\thesubappendix.\arabic{subsetning}}		
		\renewcommand \thesubsubsetning{\rev.\thesubsubappendix.\arabic{subsubsetning}}
		\renewcommand \thekode{\rev.\theappendix.\arabic{seckode}}
		\renewcommand \thesubkode{\rev.\thesubappendix.\arabic{subkode}}		
		\renewcommand \thesubsubkode{\rev.\thesubsubappendix.\arabic{subsubkode}}
	}{}
	\refstepcounter{appendix}
	\section*{Appendix \theappendix:\quad #1}
}
\newcommand{\subAppendix}[1]{\refstepcounter{subappendix}\subsection*{\thesubappendix\quad #1}}
\newcommand{\subsubAppendix}[1]{\refstepcounter{subsubappendix}\subsubsection*{\thesubsubappendix\quad #1}}
\newcommand\refapp[1]{Appendix: \ref{#1}}
%---------------------------------------------------------------------------------------------------------

% Definisjon/Setning boks

\newcommand{\mengde}[2]{\mathbb{#1}_{\text{\ref{#2}}}}
\newcommand{\hypermengde}[2]{\boldsymbol{\mathrm{#1}}_{\text{\ref{#2}}}}
\newcommand{\funk}[2]{\mathcal{#1}_{\text{\ref{#2}}}}
\newcommand{\refS}[2]{$\mathcal{#1}$:\ref{#2}}
\newcommand{\refT}[1]{$\lbrace\ref{#1}\rbrace$}
\newcommand{\refpos}[1]{$\mathcal{P}$:\ref{#1}}
\newcommand{\mathpos}[1]{\mathcal{P}_{\text{\ref{#1}}}}
\newcommand{\refspes}[1]{$\mathcal{S}$:\ref{#1}}
\newcommand{\mathspes}[1]{\mathcal{S}_{\text{\ref{#1}}}}

\newcounter{setning}[part]
\renewcommand \thesetning{\ifx \rev \undefined \arabic{setning} \else \rev.\arabic{setning} \fi }
\newcounter{secsetning}[section]
\renewcommand \thesecsetning{\ifx \rev \undefined \thesection.\arabic{secsetning} \else \rev.\thesection.\arabic{secsetning} \fi }
\newcounter{subsetning}[subsection]
\renewcommand \thesubsetning{\ifx \rev \undefined \thesubsection.\arabic{subsetning} \else \rev.\thesubsection.\arabic{subsetning} \fi }
\newcounter{subsubsetning}[subsubsection]
\renewcommand \thesubsubsetning{\ifx \rev \undefined \thesubsubsection.\arabic{subsubsetning} \else \rev.\thesubsubsection.\arabic{subsubsetning} \fi }
\newcounter{partsetning}[part]
\renewcommand \thepartsetning{\ifx \rev \undefined \thepart.\arabic{partsetning} \else \rev.\thepart.\arabic{partsetning} \fi }
\newcounter{partsecsetning}[section]
\renewcommand \thepartsecsetning{\ifx \rev \undefined \thepart.\thesection.\arabic{partsecsetning} \else \rev.\thepart.\thesection.\arabic{partsecsetning} \fi }
\newcounter{partsubsetning}[subsection]
\renewcommand \thepartsubsetning{\ifx \rev \undefined \thepart.\thesubsection.\arabic{partsubsetning} \else \rev.\thepart.\thesubsection.\arabic{partsubsetning} \fi }
\newcounter{partsubsubsetning}[subsubsection]
\renewcommand \thepartsubsubsetning{\ifx \rev \undefined \thepart.\thesubsubsection.\arabic{partsubsubsetning} \else \rev.\thepart.\thesubsubsection.\arabic{partsubsubsetning} \fi }

\newsavebox{\nr}
\newsavebox{\fcolbox}
\newenvironment{Setning}[3]
{%
	\sbox\nr{\scriptsize\emph{#2}}
    \vspace{0.25cm}%
	\setlength{\fboxrule}{0.3mm}%
	\setlength{\fboxsep}{3mm}%
	%
	\scriptsize #3 \normalsize 
	\begin{lrbox}{\fcolbox}%
	\begin{minipage}{\textwidth-0.7cm}%
%	%
	\ifthenelse{\value{part} = 0} %
	{%
		\ifthenelse{\value{section} = 0}%
		{%
			\refstepcounter{setning}%
			\textbf{\thesetning~#1~~}%
		}%
		{%
			\ifthenelse{\value{subsection} = 0}%
			{%
				\refstepcounter{secsetning}%
				\textbf{\thesecsetning~#1~~}%
			}%
			{%
				\ifthenelse{\value{subsubsection} = 0}%
				{%
					\refstepcounter{subsetning}%
					\textbf{\thesubsetning~#1~~}%
				}%
				{%
					\refstepcounter{subsubsetning}%
					\textbf{\thesubsubsetning~#1~~}%
					
				}%
			}%
		}%
	}%
	{%
		\ifthenelse{\value{section} = 0}%
		{%						
			\refstepcounter{partsetning}%
			\textbf{\thepartsetning~#1~~}%
		}%
		{%
			\ifthenelse{\value{subsection} = 0}%
			{%
				\refstepcounter{partsecsetning}%
				\textbf{\thepartsecsetning~#1~~}%
			}%
			{%
				\ifthenelse{\value{subsubsection} = 0}%
				{%
					\refstepcounter{partsubsetning}%
					\textbf{\thepartsubsetning~#1~~}%
				}%
				{%
					\refstepcounter{partsubsubsetning}%
					\textbf{\thepartsubsubsetning~#1~~}%
				}%
			}%
		}%
	}%
}%
{%
	\vspace{-0.9cm}
	\begin{flushright}%
		\usebox{\nr}%
	\end{flushright}%	
	\vspace{-0.5cm}%
	\end{minipage}%
	\end{lrbox}\fbox{\usebox{\fcolbox}}%
    
	%\end{mbox}
	\vspace{0.3cm}%
}%

%---------------------------------------------------------------------------------------------------------
\newcommand\Setn[1] {
	\ifthenelse{\value{part} = 0} %
	{%
		\ifthenelse{\value{section} = 0}%
		{%
			\refstepcounter{setning}%
			\textbf{#1 \thesetning~~}%
		}%
		{%
			\ifthenelse{\value{subsection} = 0}%
			{%
				\refstepcounter{secsetning}%
				\textbf{#1 \thesecsetning~~}%
			}%
			{%
				\ifthenelse{\value{subsubsection} = 0}%
				{%
					\refstepcounter{subsetning}%
					\textbf{#1 \thesubsetning~~}%
				}%
				{%
					\refstepcounter{subsubsetning}%
					\textbf{#1 \thesubsubsetning~~}%
					
				}%
			}%
		}%
	}%
	{%
		\ifthenelse{\value{section} = 0}%
		{%						
			\refstepcounter{partsetning}%
			\textbf{#1 \thepartsetning~~}%
		}%
		{%
			\ifthenelse{\value{subsection} = 0}%
			{%
				\refstepcounter{partsecsetning}%
				\textbf{#1 \thepartsecsetning~~}%
			}%
			{%
				\ifthenelse{\value{subsubsection} = 0}%
				{%
					\refstepcounter{partsubsetning}%
					\textbf{#1 \thepartsubsetning~~}%
				}%
				{%
					\refstepcounter{partsubsubsetning}%
					\textbf{#1 \thepartsubsubsetning~~}%
				}%
			}%
		}%
	}%
}%

% Bevis innramming

\newenvironment{Bevis}[1]%
{%
	\sbox\nr{\scriptsize\emph{Q.E.D. #1}}	
	\vspace{0.25cm}%
	\textbf{Bevis #1}%
	\par \nointerlineskip \vspace{0.1cm}%
	\rule{\textwidth}{0.4pt}%
	\par \nointerlineskip \vspace{0.3cm}%
	%\begin{minipage}{\textwidth}%
}%
{%
	%\end{minipage}%
	\begin{flushright}%
		\par \nointerlineskip
		\rule{\textwidth}{0.4pt}%
		\par \nointerlineskip \vspace{0.1cm}%
		\usebox{\nr}%
	\end{flushright}%
	\vspace{-0.2cm}%
}%

\newenvironment{Proof}[1]%
{%
	\sbox\nr{\scriptsize\emph{Q.E.D. #1}}	
	\vspace{0.25cm}%
	\textbf{Proof #1}%
	\par \nointerlineskip \vspace{0.1cm}%
	\rule{\textwidth}{0.4pt}%
	\par \nointerlineskip \vspace{0.3cm}%
	%\begin{minipage}{\textwidth}%
}%
{%
	%\end{minipage}%
	\begin{flushright}%
		\par \nointerlineskip
		\rule{\textwidth}{0.4pt}%
		\par \nointerlineskip \vspace{0.1cm}%
		\usebox{\nr}%
	\end{flushright}%
	\vspace{-0.2cm}%
}%

% Eksempel innramming

\newcounter{eksempel}[part]
\renewcommand \theeksempel{\rev.\arabic{eksempel}}
\newcounter{seceksempel}[section]
\renewcommand \theseceksempel{\rev.\thesection.\arabic{seceksempel}}
\newcounter{subeksempel}[subsection]
\renewcommand \thesubeksempel{\rev.\thesubsection.\arabic{subeksempel}}
\newcounter{subsubeksempel}[subsubsection]
\renewcommand \thesubsubeksempel{\rev.\thesubsubsection.\arabic{subsubeksempel}}
\newcounter{parteksempel}[part]
\renewcommand \theparteksempel{\rev.\thepart.\arabic{parteksempel}}
\newcounter{partseceksempel}[section]
\renewcommand \thepartseceksempel{\rev.\thepart.\thesection.\arabic{partseceksempel}}
\newcounter{partsubeksempel}[subsection]
\renewcommand \thepartsubeksempel{\rev.\thepart.\thesubsection.\arabic{partsubeksempel}}
\newcounter{partsubsubeksempel}[subsubsection]
\renewcommand \thepartsubeksempel{\rev.\thepart.\thesubsubsection.\arabic{partsubsubeksempel}}

\newenvironment{Eksempel}[1]%
{%
	\sbox\nr{\scriptsize\emph{Eksempel #1}}	
	\vspace{0.25cm}%
	\textbf{Eksempel #1}%
	\par \nointerlineskip \vspace{0.1cm}%
	\rule{\textwidth}{0.4pt}%
	\par \nointerlineskip \vspace{0.3cm}%
	%\begin{minipage}{\textwidth}%
}%
{%
	%\end{minipage}%
	\begin{flushright}%
		\par \nointerlineskip
		\rule{\textwidth}{0.4pt}%
		\par \nointerlineskip \vspace{0.1cm}%
		\usebox{\nr}%
	\end{flushright}%
	\vspace{-0.2cm}%
}%

\newenvironment{Example}[1]%
{%
	\sbox\nr{\scriptsize\emph{Example} 
	\ifthenelse{\value{part} = 0} %
	{%
		\ifthenelse{\value{section} = 0}%
		{%
			\refstepcounter{eksempel}%
			\textit{\theeksempel}%
		}%
		{%
			\ifthenelse{\value{subsection} = 0}%
			{%
				\refstepcounter{seceksempel}%
				\textit{\theseceksempel}%
			}%
			{%
				\ifthenelse{\value{subsubsection} = 0}%
				{%
					\refstepcounter{subeksempel}%
					\textit{\thesubeksempel}%
				}%
				{%
					\refstepcounter{subsubeksempel}%
					\textit{\thesubsubeksempel}%
				}%
			}%
		}%
	}%
	{%
		\ifthenelse{\value{section} = 0}%
		{%						
			\refstepcounter{parteksempel}%
			\textit{\theparteksempel}%
		}%
		{%
			\ifthenelse{\value{subsection} = 0}%
			{%
				\refstepcounter{partseceksempel}%
				\textit{\thepartseceksempel}%
			}%
			{%
				\ifthenelse{\value{subsubsection} = 0}%
				{%
					\refstepcounter{partsubeksempel}%
					\textit{\thepartsubeksempel}%
				}%
				{%
					\refstepcounter{partsubsubeksempel}%
					\textit{\thepartsubsubeksempel}%
				}%
			}%
		}%
	}%	
	}%
		
	\vspace{0.25cm}%
	\ifthenelse{\value{part} = 0} %
	{%
		\ifthenelse{\value{section} = 0}%
		{%
			\textbf{Example \theeksempel :~#1}%
		}%
		{%
			\ifthenelse{\value{subsection} = 0}%
			{%
				\textbf{Example \theseceksempel :~#1}%
			}%
			{%
				\ifthenelse{\value{subsubsection} = 0}%
				{%
					\textbf{Example \thesubeksempel :~#1}%
				}%
				{%
					\textbf{Example \thesubsubeksempel :~#1}%
				}%
			}%
		}%
	}%
	{%
		\ifthenelse{\value{section} = 0}%
		{%						
			\textbf{Example \theparteksempel :~#1}%
		}%
		{%
			\ifthenelse{\value{subsection} = 0}%
			{%
				\textbf{Example \thepartseceksempel :~#1}%
			}%
			{%
				\ifthenelse{\value{subsubsection} = 0}%
				{%
					\textbf{Example \thepartsubeksempel :~#1}%
				}%
				{%
					\textbf{Example \thepartsubsubeksempel :~#1}%
				}%
			}%
		}%
	}%
	\par \nointerlineskip \vspace{0.1cm}%
	\rule{\textwidth}{0.4pt}%
	\par \nointerlineskip \vspace{0.3cm}%
	%\begin{minipage}{\textwidth}%
}%
{%
	%\end{minipage}%
	\begin{flushright}%
		\par \nointerlineskip
		\rule{\textwidth}{0.4pt}%
		\par \nointerlineskip \vspace{0.1cm}%
		\usebox{\nr}%
	\end{flushright}%
	\vspace{-0.2cm}%
}%

%---------------------------------------------------------------------------------------------------------

% Oppsett av dokumentet

\tolerance = 5000 % LaTeX er normalt streng n�r det gjelder linjebrytingen.
\hbadness = \tolerance % Vi vil v�re litt mildere, s�rlig fordi norsk har s�
\pretolerance = 2000 % mange lange sammensatte ord.

\let\OLDtableofcontents = \tableofcontents
\def\tableofcontents
{
	\setlength{\cftpartnumwidth}{4em}	
	\let\Section = \section
	\let\section = \OLDsection
	\OLDtableofcontents
	\let\section = \Section
}

\let\OLDlistoftables = \listoftables
\def\listoftables{
	\setlength{\cfttabindent}{0em}
	\setlength{\cfttabnumwidth}{4em} 
	\let\Section = \section
	\let\section = \OLDsection
	\OLDlistoftables
	\let\section = \Section
}

\let\OLDlistoffigures = \listoffigures
\def\listoffigures{
	\setlength{\cftfigindent}{0em}
	\setlength{\cftfignumwidth}{4em} 
	\let\Section = \section
	\let\section = \OLDsection
	\OLDlistoffigures
	\let\section = \Section
}

\def\innledende_sider
{
    \pagestyle{empty}
	\maketitle
	\thispagestyle{empty}
	\cleartooddpage 
	\thispagestyle{empty}
	\cleartooddpage
	\tableofcontents
	\cleartooddpage
	\listoftables
	\cleartooddpage
	\listoffigures
	\cleartooddpage
    \pagestyle{plain}
	\pagenumbering{arabic}
}

\def\innholdsfortegnelse
{
    \pagestyle{empty}
	\maketitle
	\thispagestyle{empty}
	\cleartooddpage 
	\thispagestyle{empty}
	\cleartooddpage
	\tableofcontents
	\cleartooddpage
    \pagestyle{plain}
	\pagenumbering{arabic}
}
%---------------------------------------------------------------------------------------------------------

% Journalfremside

\newcommand\journalforside[7]
{
	\pagestyle{empty}
	\begin{center}
		\LARGE \textbf{#1}
	\end{center}
	\setlength\extrarowheight{0.5cm}
	\begin{tabular}{| p{1.0 cm} p{0.8cm} | p{2.1cm} p{0.5cm} | p{2.4cm} p{3cm} |}
	\multicolumn{6}{l}{\textbf{Student:}} \\
	\hline
	\multicolumn{1}{|l}{\textbf{Navn:}} & \multicolumn{5}{l|}{#2}\\ 
	\hline
	\multicolumn{1}{|l}{\textbf{Epost:}} & \multicolumn{5}{l|}{\texttt{#3}}\\ 
	\hline
	\textbf{Gruppe:} & {#4} & \textbf{�velse nr.:} & {#5} & \textbf{Utf�rt dato:} & {#6}\\
	\hline
	\end{tabular}
	
	\setlength\extrarowheight{0.5cm}
	\begin{tabular}{| p{7cm} | p{4.5cm} |}
	\multicolumn{2}{l}{\textbf{Veileder/retter:}} \\
	\hline
	\textbf{Godkjent/rettet:} & \textbf{Dato:}   \\
	\hline
	\end{tabular}
	
	\vspace{0.5cm}
	\setlength\extrarowheight{0cm}
	\begin{tabular}{|p{11.92cm}|}
	\multicolumn{1}{l}{\textbf{Om du vil:}}\\
	\hline
	Spesielle forhold som retter b�r ta hensyn til:\\\\
	{#7}\\
	\hline
	\end{tabular}

	\vspace{1cm}
	\setlength{\unitlength}{3947sp}%
	\begin{picture}(5000,0)
	{
	\drawline[AHnb=0,linewidth=15,dash={100}0](-2000,0)(8000,0)
	}
	\end{picture}
	\vspace{0.3cm}

	\setlength\extrarowheight{0.5cm}
	\begin{tabular}{| p{1.0 cm} p{0.8cm} | p{2.1cm} p{0.5cm} | p{2.4cm} p{3cm} |}
	\multicolumn{6}{l}{\textbf{Student:}} \\
	\hline
	\multicolumn{1}{|l}{\textbf{Navn:}} & \multicolumn{5}{l|}{#2}\\ 
	\hline
	\multicolumn{1}{|l}{\textbf{Epost:}} & \multicolumn{5}{l|}{\texttt{#3}}\\ 
	\hline
	\textbf{Gruppe:} & {#4} & \textbf{�velse nr.:} & {#5} & \textbf{Utf�rt dato:} & {#6}\\
	\hline
	\end{tabular}
	
	\setlength\extrarowheight{0.5cm}
	\begin{tabular}{| p{7cm} | p{4.5cm} |}
	\multicolumn{2}{l}{\textbf{Veileder/retter:}} \\
	\hline
	\textbf{Godkjent/rettet:} & \textbf{Dato:}   \\
	\hline
	\end{tabular}
	
	\vspace{0.5cm}
	\setlength\extrarowheight{0cm}
	\begin{tabular}{|p{11.92cm}|}
	\multicolumn{1}{l}{\textbf{Om du vil:}}\\
	\hline
	Spesielle forhold som retter b�r ta hensyn til:\\\\
	{#7}\\
	\hline
	\end{tabular}
	\cleartooddpage
}

%---------------------------------------------------------------------------------------------------------
\newcounter{figur}[part]
\newcounter{secfigur}[section]
\renewcommand \thesecfigur{\thesection.\arabic{secfigur}}
\newcounter{subfigur}[subsection]
\renewcommand \thesubfigur{\thesubsection.\arabic{subfigur}}
\newcounter{subsubfigur}[subsubsection]
\renewcommand \thesubsubfigur{\thesubsubsection.\arabic{subsubfigur}}
\newcounter{partfigur}[part]
\renewcommand \thepartfigur{\thepart.\arabic{partfigur}}
\newcounter{partsecfigur}[section]
\renewcommand \thepartsecfigur{\thepart.\thesection.\arabic{partsecfigur}}
\newcounter{partsubfigur}[subsection]
\renewcommand \thepartsubfigur{\thepart.\thesubsection.\arabic{partsubfigur}}
\newcounter{partsubsubfigur}[subsubsection]
\renewcommand \thepartsubsubfigur{\thepart.\thesubsubsection.\arabic{partsubsubfigur}}

% Sette inn bilde
\newcommand\figur[4]
{
	\begin{figure}[!htp]%
		\vspace{-0.1cm}
		\begin{center}%
            %\includegraphics[scale = {#1}]{#2}
			\includegraphics[width = {#1}\textwidth]{#2}% 
		\end{center}%
		\ifthenelse{\value{part} = 0} %
		{%
			\ifthenelse{\value{section} = 0}%
			{%
				\refstepcounter{figur}%
				\renewcommand\thefigure{\thefigur}%
			}%
			{%
				\ifthenelse{\value{subsection} = 0}%
				{%
					\refstepcounter{secfigur}%
					\renewcommand\thefigure{\thesecfigur}%
				}%
				{%
					\ifthenelse{\value{subsubsection} = 0}%
					{%
						\refstepcounter{subfigur}%
						\renewcommand\thefigure{\thesubfigur}%
					}%
					{%
						\refstepcounter{subsubfigur}%
						\renewcommand\thefigure{\thesubsubfigur}%
					}%
				}%
			}%
		}%
		{%
			\ifthenelse{\value{section} = 0}%
			{%						
				\refstepcounter{partfigur}%
				\renewcommand\thefigure{\thepartfigur}%
			}%
			{%
				\ifthenelse{\value{subsection} = 0}%
				{%
					\refstepcounter{partsecfigur}%
					\renewcommand\thefigure{\thepartsecfigur}%
				}%
				{%
					\ifthenelse{\value{subsubsection} = 0}%
					{%
						\refstepcounter{partsubfigur}%
						\renewcommand\thefigure{\thepartsubfigur}%
					}%
					{%
						\refstepcounter{partsubsubfigur}%
						\renewcommand\thefigure{\thepartsubsubfigur}%
					}%
				}%
			}%
		}%	
		\vspace{-0.7cm}	
		\caption{\textit{#3}}%
		\label{#4}
		\vspace{-0.4cm}
	\end{figure}%
}%

\newcommand\multifigur[3]
{
	\begin{figure}[!htp]%
		\vspace{-0.7cm} 
		\begin{center}%
            #1
		\end{center}%
		\ifthenelse{\value{part} = 0} %
		{%
			\ifthenelse{\value{section} = 0}%
			{%
				\refstepcounter{figur}%
				\renewcommand\thefigure{\thefigur}%
			}%
			{%
				\ifthenelse{\value{subsection} = 0}%
				{%
					\refstepcounter{secfigur}%
					\renewcommand\thefigure{\thesecfigur}%
				}%
				{%
					\ifthenelse{\value{subsubsection} = 0}%
					{%
						\refstepcounter{subfigur}%
						\renewcommand\thefigure{\thesubfigur}%
					}%
					{%
						\refstepcounter{subsubfigur}%
						\renewcommand\thefigure{\thesubsubfigur}%
					}%
				}%
			}%
		}%
		{%
			\ifthenelse{\value{section} = 0}%
			{%						
				\refstepcounter{partfigur}%
				\renewcommand\thefigure{\thepartfigur}%
			}%
			{%
				\ifthenelse{\value{subsection} = 0}%
				{%
					\refstepcounter{partsecfigur}%
					\renewcommand\thefigure{\thepartsecfigur}%
				}%
				{%
					\ifthenelse{\value{subsubsection} = 0}%
					{%
						\refstepcounter{partsubfigur}%
						\renewcommand\thefigure{\thepartsubfigur}%
					}%
					{%
						\refstepcounter{partsubsubfigur}%
						\renewcommand\thefigure{\thepartsubsubfigur}%
					}%
				}%
			}%
		}%		
		\vspace{-0.7cm}
		\caption{\textit{#2}}%
		\label{#3}
		%\vspace{-0.4cm}
	\end{figure}%
}%

\newcommand\subfigur[3]
{
	\begin{tabular}[t]{l}
		\begin{tabular}{c} \\ \normalsize \textbf{#1} \\ \end{tabular}  \\
	 	\includegraphics[width = {#2}\textwidth]{#3} 
	\end{tabular}
}

\newcommand\flowchart[5]
{
	\begin{figure}[!htp]%
		\vspace{-0.1cm}
		\tiny
		\begin{center}
		\begin{psmatrix}[rowsep={#1},colsep={#2}] 
			#3
		\end{psmatrix}
		\end{center}%
		\ifthenelse{\value{part} = 0} %
		{%
			\ifthenelse{\value{section} = 0}%
			{%
				\refstepcounter{figur}%
				\renewcommand\thefigure{\thefigur}%
			}%
			{%
				\ifthenelse{\value{subsection} = 0}%
				{%
					\refstepcounter{secfigur}%
					\renewcommand\thefigure{\thesecfigur}%
				}%
				{%
					\ifthenelse{\value{subsubsection} = 0}%
					{%
						\refstepcounter{subfigur}%
						\renewcommand\thefigure{\thesubfigur}%
					}%
					{%
						\refstepcounter{subsubfigur}%
						\renewcommand\thefigure{\thesubsubfigur}%
					}%
				}%
			}%
		}%
		{%
			\ifthenelse{\value{section} = 0}%
			{%						
				\refstepcounter{partfigur}%
				\renewcommand\thefigure{\thepartfigur}%
			}%
			{%
				\ifthenelse{\value{subsection} = 0}%
				{%
					\refstepcounter{partsecfigur}%
					\renewcommand\thefigure{\thepartsecfigur}%
				}%
				{%
					\ifthenelse{\value{subsubsection} = 0}%
					{%
						\refstepcounter{partsubfigur}%
						\renewcommand\thefigure{\thepartsubfigur}%
					}%
					{%
						\refstepcounter{partsubsubfigur}%
						\renewcommand\thefigure{\thepartsubsubfigur}%
					}%
				}%
			}%
		}%	
		\vspace{-0.7cm}	
		\caption{\textit{#4}}%
		\label{#5}
		\vspace{-0.4cm}
	\end{figure}%
}%

\newcommand\reffig[1]{\figurename\,\ref{#1}}
%---------------------------------------------------------------------------------------------------------
\newcolumntype{I}{!{\vrule width 1pt}}
\newlength\savewidth
\newcommand\whline
{
	\noalign{
		\global\savewidth\arrayrulewidth
		\global\arrayrulewidth 1pt}%
	\hline
	\noalign
	{
		\global\arrayrulewidth\savewidth
	}%
}%

\newcounter{tabell}[part]
\newcounter{sectabell}[section]
\renewcommand \thesectabell{\thesection.\arabic{sectabell}}
\newcounter{subtabell}[subsection]
\renewcommand \thesubtabell{\thesubsection.\arabic{subtabell}}
\newcounter{subsubtabell}[subsubsection]
\renewcommand \thesubsubtabell{\thesubsubsection.\arabic{subsubtabell}}
\newcounter{parttabell}[part]
\renewcommand \theparttabell{\thepart.\arabic{parttabell}}
\newcounter{partsectabell}[section]
\renewcommand \thepartsectabell{\thepart.\thesection.\arabic{partsectabell}}
\newcounter{partsubtabell}[subsection]
\renewcommand \thepartsubtabell{\thepart.\thesubsection.\arabic{partsubtabell}}
\newcounter{partsubsubtabell}[subsubsection]
\renewcommand \thepartsubsubtabell{\thepart.\thesubsubsection.\arabic{partsubsubtabell}}
								
% Tabell
\newcommand\tabell[6]
{
  	\begin{table}[!htp]%
  		\begin{center}
			\setlength\extrarowheight{2pt}
			#2
			\begin{tabular}{#1}
				\whline
				#3
				\whline
				#4
				\whline
			\end{tabular}
		\end{center}
		\ifthenelse{\value{part} = 0} %
		{%
			\ifthenelse{\value{section} = 0}%
			{%
				\refstepcounter{tabell}%
				\renewcommand\thetable{\thetabell}%
			}%
			{%
				\ifthenelse{\value{subsection} = 0}%
				{%
					\refstepcounter{sectabell}%
					\renewcommand\thetable{\thesectabell}%
				}%
				{%
					\ifthenelse{\value{subsubsection} = 0}%
					{%
						\refstepcounter{subtabell}%
						\renewcommand\thetable{\thesubtabell}%
					}%
					{%
						\refstepcounter{subsubtabell}%
						\renewcommand\thetable{\thesubsubtabell}%
					}%
				}%
			}%
		}%
		{%
			\ifthenelse{\value{section} = 0}%
			{%						
				\refstepcounter{parttabell}%
				\renewcommand\thetable{\theparttabell}%
			}%
			{%
				\ifthenelse{\value{subsection} = 0}%
				{%
					\refstepcounter{partsectabell}%
					\renewcommand\thetable{\thepartsectabell}%
				}%
				{%
					\ifthenelse{\value{subsubsection} = 0}%
					{%
						\refstepcounter{partsubtabell}%
						\renewcommand\thetable{\thepartsubtabell}%
					}%
					{%
						\refstepcounter{partsubsubtabell}%
						\renewcommand\thetable{\thepartsubsubtabell}%
					}%
				}%
			}%
		}%
		\vspace{-0.5cm}
    	\caption{\textit{#5}}
		\label{#6}
	\end{table}
	%\vspace{-0.4cm}
}
\newcommand\multitabell[5]
{
  	\begin{table}[!htp]%
	\vspace{-0.4cm}  	
  		\begin{center}
			\setlength\extrarowheight{2pt}
			#2
			\begin{tabular}{#1}
				#3
			\end{tabular}
		\end{center}
		\ifthenelse{\value{part} = 0} %
		{%
			\ifthenelse{\value{section} = 0}%
			{%
				\refstepcounter{tabell}%
				\renewcommand\thetable{\thetabell}%
			}%
			{%
				\ifthenelse{\value{subsection} = 0}%
				{%
					\refstepcounter{sectabell}%
					\renewcommand\thetable{\thesectabell}%
				}%
				{%
					\ifthenelse{\value{subsubsection} = 0}%
					{%
						\refstepcounter{subtabell}%
						\renewcommand\thetable{\thesubtabell}%
					}%
					{%
						\refstepcounter{subsubtabell}%
						\renewcommand\thetable{\thesubsubtabell}%
					}%
				}%
			}%
		}%
		{%
			\ifthenelse{\value{section} = 0}%
			{%						
				\refstepcounter{parttabell}%
				\renewcommand\thetable{\theparttabell}%
			}%
			{%
				\ifthenelse{\value{subsection} = 0}%
				{%
					\refstepcounter{partsectabell}%
					\renewcommand\thetable{\thepartsectabell}%
				}%
				{%
					\ifthenelse{\value{subsubsection} = 0}%
					{%
						\refstepcounter{partsubtabell}%
						\renewcommand\thetable{\thepartsubtabell}%
					}%
					{%
						\refstepcounter{partsubsubtabell}%
						\renewcommand\thetable{\thepartsubsubtabell}%
					}%
				}%
			}%
		}%
		\vspace{-0.5cm}
    	\caption{\textit{#4}}
		\label{#5}
	\end{table}
	\vspace{-0.4cm}
}
\newcommand\subtabell[4]
{
	\begin{tabular}[t]{rl}
	\begin{tabular}{c} \\ \normalsize \textbf{#1} \\ \end{tabular} &
	\begin{tabular}[t]{#2}
		\whline
		#3
		\whline
		#4
		\whline
	\end{tabular}
	\end{tabular}
}
\newcommand\reftab[1]{\tablename\,\ref{#1}}

%---------------------------------------------------------------------------------------------------------
% Kode innramming
\newcommand\kode[1]{\verbatiminput{#1}}

\newcounter{kode}[part]
\renewcommand \thekode{\rev.\arabic{kode}}
\newcounter{seckode}[section]
\renewcommand \theseckode{\rev.\thesection.\arabic{seckode}}
\newcounter{subkode}[subsection]
\renewcommand \thesubkode{\rev.\thesubsection.\arabic{subkode}}
\newcounter{subsubkode}[subsubsection]
\renewcommand \thesubsubkode{\rev.\thesubsubsection.\arabic{subsubkode}}
\newcounter{partkode}[part]
\renewcommand \thepartkode{\rev.\thepart.\arabic{partkode}}
\newcounter{partseckode}[section]
\renewcommand \thepartseckode{\rev.\thepart.\thesection.\arabic{partseckode}}
\newcounter{partsubkode}[subsection]
\renewcommand \thepartsubkode{\rev.\thepart.\thesubsection.\arabic{partsubkode}}
\newcounter{partsubsubkode}[subsubsection]
\renewcommand \thepartsubsubkode{\rev.\thepart.\thesubsubsection.\arabic{partsubsubkode}}

\newcommand\code[2]%
{%
	\sbox\nr{\scriptsize\emph{Code} 
	\ifthenelse{\value{part} = 0} %
	{%
		\ifthenelse{\value{section} = 0}%
		{%
			\refstepcounter{kode}%
			\textit{\thekode}%
		}%
		{%
			\ifthenelse{\value{subsection} = 0}%
			{%
				\refstepcounter{seckode}%
				\textit{\theseckode}%
			}%
			{%
				\ifthenelse{\value{subsubsection} = 0}%
				{%
					\refstepcounter{subkode}%
					\textit{\thesubkode}%
				}%
				{%
					\refstepcounter{subsubeksempel}%
					\textit{\thesubsubkode}%
				}%
			}%
		}%
	}%
	{%
		\ifthenelse{\value{section} = 0}%
		{%						
			\refstepcounter{partkode}%
			\textit{\thepartkode}%
		}%
		{%
			\ifthenelse{\value{subsection} = 0}%
			{%
				\refstepcounter{partseckode}%
				\textit{\thepartseckode}%
			}%
			{%
				\ifthenelse{\value{subsubsection} = 0}%
				{%
					\refstepcounter{partsubkode}%
					\textit{\thepartsubkode}%
				}%
				{%
					\refstepcounter{partsubsubkode}%
					\textit{\thepartsubsubkode}%
				}%
			}%
		}%
	}%	
	}%
	\vspace{0.25cm}%
	\ifthenelse{\value{part} = 0} %
	{%
		\ifthenelse{\value{section} = 0}%
		{%
			\textbf{Code \thekode :~{#2}}%
		}%
		{%
			\ifthenelse{\value{subsection} = 0}%
			{%
				\textbf{Code \theseckode :~#2}%
			}%
			{%
				\ifthenelse{\value{subsubsection} = 0}%
				{%
					\textbf{Code \thesubkode :~#2}%
				}%
				{%
					\textbf{Code \thesubsubkode :~#2}%
				}%
			}%
		}%
	}%
	{%
		\ifthenelse{\value{section} = 0}%
		{%						
			\textbf{Code \thepartkode :~#2}%
		}%
		{%
			\ifthenelse{\value{subsection} = 0}%
			{%
				\textbf{Code \thepartseckode :~#2}%
			}%
			{%
				\ifthenelse{\value{subsubsection} = 0}%
				{%
				\textbf{Code \thepartsubkode :~#2}%
				}%
				{%
					\textbf{Code \thepartsubsubkode :~#2}%
				}%
			}%
		}%
	}%
	\par \nointerlineskip \vspace{0.1cm}%
	\rule{\textwidth}{0.4pt}%
	\par \nointerlineskip \vspace{0.3cm}%
	%\begin{minipage}{\textwidth}%
	\tiny\verbatiminput{#1}\normalsize
	%\end{minipage}%
	\begin{flushright}%
		\par \nointerlineskip
		\rule{\textwidth}{0.4pt}%
		\par \nointerlineskip \vspace{0.1cm}%
		\usebox{\nr}%
	\end{flushright}%
	\vspace{-0.2cm}%
}%
%---------------------------------------------------------------------------------------------------------
% Oscilloskopbilde
\newcommand\oscilloskop[6]
{
	\begin{figure}[!htp]
		\begin{center}
			\setlength{\unitlength}{3947sp}%
			%
				%\begingroup\makeatletter\ifx\SetFigFont\undefined%
				%\gdef\SetFigFont#1#2#3#4#5{%
	  			%\reset@font\fontsize{#1}{#2pt}%
	  			%\fontfamily{#3}\fontseries{#4}\fontshape{#5}%
	  			%\selectfont}%
			%\fi\endgroup%
			\begin{picture}(6000,2500)
			{
				\color[rgb]{0,0,0}
				\scriptsize
	
				\put(0,0){\includegraphics[width = 0.6\textwidth]{#1}}
				\put(3500,2000)
				{
				\begin{tabular}{l p{3cm}}
					\vspace{0.5cm}
					Signal: 		& {#2} \\
					\vspace{0.1cm}
			    	V/div: 		& {#3} \\
					sec/div: 	& {#4}
				\end{tabular}
				} 
			}
			\end{picture}%
		\end{center}
		\caption{\textit{#5}}
		\label{#6}
	\end{figure}
}{}
%---------------------------------------------------------------------------------------------------------
\newcommand{\tegning}[5]
{
	\begin{figure}[!htp]%
		\begin{center}
			\setlength{\unitlength}{3947sp}%
			\begin{picture}(#1,#2)
				\color[rgb]{0,0,0}
				\scriptsize 
				{#3}
			\end{picture}
		\end{center}%
		\ifthenelse{\value{part} = 0} %
		{%
			\ifthenelse{\value{section} = 0}%
			{%
				\refstepcounter{figur}%
				\renewcommand\thefigure{\thefigur}%
			}%
			{%
				\ifthenelse{\value{subsection} = 0}%
				{%
					\refstepcounter{secfigur}%
					\renewcommand\thefigure{\thesecfigur}%
				}%
				{%
					\ifthenelse{\value{subsubsection} = 0}%
					{%
						\refstepcounter{subfigur}%
						\renewcommand\thefigure{\thesubfigur}%
					}%
					{%
						\refstepcounter{subsubfigur}%
						\renewcommand\thefigure{\thesubsubfigur}%
					}%
				}%
			}%
		}%
		{%
			\ifthenelse{\value{section} = 0}%
			{%						
				\refstepcounter{partfigur}%
				\renewcommand\thefigure{\thepartfigur}%
			}%
			{%
				\ifthenelse{\value{subsection} = 0}%
				{%
					\refstepcounter{partsecfigur}%
					\renewcommand\thefigure{\thepartsecfigur}%
				}%
				{%
					\ifthenelse{\value{subsubsection} = 0}%
					{%
						\refstepcounter{partsubfigur}%
						\renewcommand\thefigure{\thepartsubfigur}%
					}%
					{%
						\refstepcounter{partsubsubfigur}%
						\renewcommand\thefigure{\thepartsubsubfigur}%
					}%
				}%
			}%
		}%		
		\caption{\textit{#4}}%
		\label{#5}
	\end{figure}%
}
\newenvironment{Tegning}[3]
{%
	\sbox\nr{\textit{#3}}
	\ifthenelse{\value{part} = 0} %
	{%
		\ifthenelse{\value{section} = 0}%
		{%
			\refstepcounter{figur}%
		}%
		{%
			\ifthenelse{\value{subsection} = 0}%
			{%
				\refstepcounter{secfigur}%
			}%
			{%
				\ifthenelse{\value{subsubsection} = 0}%
				{%
					\refstepcounter{subfigur}%
				}%
				{%
					\refstepcounter{subsubfigur}%
				}%
			}%
		}%
	}%
	{%
		\ifthenelse{\value{section} = 0}%
		{%						
			\refstepcounter{partfigur}%
		}%
		{%
			\ifthenelse{\value{subsection} = 0}%
			{%
				\refstepcounter{partsecfigur}%
			}%
			{%
				\ifthenelse{\value{subsubsection} = 0}%
				{%
					\refstepcounter{partsubfigur}%
				}%
				{%
					\refstepcounter{partsubsubfigur}%
				}%
			}%
		}%
	}%	
	\begin{figure}[!htp]
		\begin{center}
			\setlength{\unitlength}{3947sp}%
			%
				%\begingroup\makeatletter\ifx\SetFigFont\undefined%
				%\gdef\SetFigFont#1#2#3#4#5{%
	  			%\reset@font\fontsize{#1}{#2pt}%
	  			%\fontfamily{#3}\fontseries{#4}\fontshape{#5}%
	  			%\selectfont}%
			%\fi\endgroup%
			\begin{picture}(#1,#2)
				\color[rgb]{0,0,0}
				\scriptsize
	
}%
{
			\end{picture}%
		\end{center}
		\begin{center}%	
			\ifthenelse{\value{part} = 0} %
			{%
				\ifthenelse{\value{section} = 0}%
				{%
					\text{Figur \thefigur:~~}%
				}%
				{%
					\ifthenelse{\value{subsection} = 0}%
					{%
						\text{Figur \thesecfigur:~~}%
					}%
					{%
						\ifthenelse{\value{subsubsection} = 0}%
						{%
							\text{Figur \thesubfigur:~~}%
						}%
						{%
							\text{Figur \thesubsubfigur:~~}%
						}%
					}%
				}%
			}%
			{%
				\ifthenelse{\value{section} = 0}%
				{%						
					\text{Figur \thepartfigur:~~}%
				}%
				{%
					\ifthenelse{\value{subsection} = 0}%
					{%
						\text{Figur \thepartsecfigur:~~}%
					}%
					{%
						\ifthenelse{\value{subsubsection} = 0}%
						{%
							\text{Figur \thepartsubfigur:~~}%
						}%
						{%
							\text{Figur \thepartsubsubfigur:~~}%
						}%
					}%
				}%
			}%
			\usebox{\nr}%
		\end{center}%
	\end{figure}
}

\newcommand\measureline[8]
{
	\drawline[AHnb=1,AHangle=30,AHLength=75,AHlength=0,ATnb=1,ATangle=30,ATLength=75,ATlength=0](#1,#2)(#3,#4)
	\drawline[AHnb=1,AHangle=90,AHLength=50,AHlength=0,ATnb=1,ATangle=90,ATLength=50,ATlength=0](#1,#2)(#3,#4)
	\drawline[AHnb=0,dash={50}0](#5,#6)(#1,#2)
	\drawline[AHnb=0,dash={50}0](#7,#8)(#3,#4)
}

\newcommand\textoval[5]
{
  \node[Nadjust=wh](#1)(#2,#3)
	{
		\begin{tabular}{#4}
			#5
		\end{tabular}
	}
}
\newcommand\textbox[5]
{
  \node[Nadjust=wh,Nmr=0](#1)(#2,#3)
	{
		\begin{tabular}{#4}
			#5
		\end{tabular}
	}
}
%---------------------------------------------------------------------------------------------------------
\newcommand\blankfigur[2]
{
  \cleartooddpage
  \tegning{5000}{9000}{}{#1}{#2}
  \cleartooddpage 
}
%---------------------------------------------------------------------------------------------------------
\usepackage{fullpage}

\renewcommand\title{FYS4150 - Computational Physics - Project 2}

\renewcommand\author{Eimund Smestad}
%\newcommand\adress{}
\renewcommand\date{\today}
\newcommand\email{\href{mailto:eimundsm@fys.uio.no}{eimundsm@fys.uio.no}}

\lstset{language=[Visual]C++,caption={Descriptive Caption Text},label=DescriptiveLabel}

\begin{document}
\maketitle
\begin{flushleft}

\begin{abstract}
This report looks at different algorithms to solve the one-dimensional Poisson's equation with Dirichlet boundary condition. The algorithms are compared by speed and floating point error in the solution. The results show how different ways of writing code effect speed and floating point error, and discusses how the machines architecture and functionally are the reason for the performance difference.
\end{abstract}

\section{Theory}

Side 24

\begin{align*}
\im \hbar \frac{\partial \Psi}{\partial t} =- \frac{\hbar^2}{2m} \nabla^2 \Psi + V \Psi
\end{align*}

\begin{align*}
\Psi\p{\vec{x},t} = \psi\p{\vec{x}} \varphi\p{t}
\end{align*}

\begin{align*}
-\frac{\hbar^2}{2m} \frac{\mathrm{d}^2 \psi}{\mathrm{d} x^2} + V \psi = E \psi
\end{align*}

Side 133

\begin{align*}
\nabla^2 = \frac{1}{r^2} \frac{\partial}{\partial r}\p{r^2 \frac{\partial }{\partial r}} + \frac{1}{r^2 \sin \theta} \frac{\partial}{\partial \theta}\p{\sin \theta \frac{\partial }{\partial \theta}} + \frac{1}{r^2 \sin^2 \theta} \frac{\partial^2}{\partial \phi^2}
\end{align*}

\begin{align*}
-\frac{\hbar^2}{2m}\kl{\frac{1}{r^2} \frac{\partial}{\partial r}\p{r^2 \frac{\partial \psi}{\partial r}} + \frac{1}{r^2 \sin \theta} \frac{\partial}{\partial \theta}\p{\sin\theta \frac{\partial \psi}{\partial \theta}} + \frac{1}{r^2 \sin^2 \theta} \frac{\partial^2 \psi}{\partial \phi}} + V \psi = E \psi
\end{align*}

\begin{align*}
\psi\p{r,\theta,\phi} = R\p{r} Y\p{\theta, \phi} + C
\end{align*}

\begin{align*}
\frac{1}{R}\frac{\mathrm{d}}{\mathrm{d}r}\p{r^2 \frac{\mathrm{d} R}{\mathrm{d} r}} - \frac{2m r^2}{\hbar^2}\p{V-E} + \frac{1}{Y}\p{\frac{1}{\sin \theta} \frac{\partial}{\partial \theta}\p{\sin\theta \frac{\partial Y}{\partial \theta}} + \frac{1}{\sin^2\theta}\frac{\partial^2 Y}{\partial \phi^2}} = \frac{2m r^2 C}{R Y \hbar^2}\p{V-E}
\end{align*}

\section{Eigenvalue algorithms}

\subsection{Diagonalize by matrix similarity}\label{sec_Diagonalize}

If we have a eigenvalue problem for a matrix $\mathbb{A}$ with eigenvector $\mathbf{x}$ and eigenvalue $\lambda$

\begin{align*}
\mathbf{A}\mathbf{x} = \lambda \mathbf{x} \,,
\end{align*}

we can transform it by a basis matrix $\mathbf{S}$ to a similar matrix $\mathbf{B} = \mathbf{S}^{-1}\mathbf{A}\mathbf{S}$ with the same eigenvalue $\lambda$ but with a different eigenvector $\mathbf{S}^{-1} \mathbf{x}$;

\begin{align*}
\lambda \mathbf{S}^{-1} \mathbf{x} = \mathbf{S}^{-1}\mathbf{A}\mathbf{x} = \mathbf{S}^{-1}\mathbf{A}\mathbf{S} \mathbf{S}^{-1}\mathbf{x} = \mathbf{B}\mathbf{S}^{-1}\mathbf{x} \,.
\end{align*}

If we are able to diagonalize out matrix $\mathbf{A}$ the eigenvalue problem becomes trivial to solve. However to directly diagonalize matrix $\mathbf{A}$ of size $n\times n$ requires to solve a $n$-th polynomial problem, which is a very hard task. So we choose a easier way by iteratively zero out more and more non-diagonal elements of matrix $\mathbf{A}$, which will eventually give us the diagonal matrix

\begin{align*}
\mathbf{D} = \prod_i \mathbf{S}_i^{-1} \mathbf{A} \prod_i \mathbf{S}_i\,.
\end{align*}

To make the problem simple to solve we define a basis matrix for transformation as such

\begin{align}
\mathbf{S} = \begin{bmatrix} 1 & 0 & \cdots & 0 & \cdots & 0 & \cdots & 0  \\ 0 & 1 & \cdots & 0 & \cdots & 0 & \cdots & 0 \\ \vdots & \vdots & \ddots & \vdots & \ddots & \vdots & \ddots & \vdots  \\ 0 & 0 & \cdots & s_{ii} & \cdots & s_{ij} & \cdots & 0\\ \vdots & \vdots & \ddots & \vdots & \ddots & \vdots & \ddots & \vdots \\ 0 & 0 & \cdots & s_{ji} & \cdots & s_{jj} & \cdots & 0\\ \vdots & \vdots & \ddots & \vdots & \ddots & \vdots & \ddots & \vdots \\ 0 & 0 & \cdots & 0 & \cdots & 0 & \cdots & 1\\\end{bmatrix} \,,
\label{eq_1}
\end{align}

where the strategy is for this basis matrix $\mathbf{S}$ will make $b_{ij} = b_{ji} = 0$ when we do a matrix similarity transformation of $\mathbf{A}$ to $\mathbf{B}$. The matrix elements of $\mathbf{S}^{-1}$ is then given by

\begin{align*}
s_{k\ell}^- = \begin{cases} \frac{s_{jj}}{s_{ii}s_{jj}-s_{ij}s_{ji}} & \text{for $k=\ell=i$} \\ -\frac{s_{ij}}{s_{ii}s_{jj}-s_{ij}s_{ji}} & \text{for $k=i$ and $\ell=j$} \\ -\frac{s_{ji}}{s_{ii}s_{jj}-s_{ij}s_{ji}} & \text{for $k=j$ and $\ell=i$} \\ \frac{s_{ii}}{s_{ii}s_{jj}-s_{ij}s_{ji}} & \text{for $k=\ell=j$} \\ 1 & \text{for $k=\ell\in\mathbb{N}_1^n/\kr{i,j}$} \\ 0 & \text{elsewhere.} \end{cases}
\end{align*}

To see how this matrix similarity transformation behaves we must do the cumbersome matrix multiplication of the transformation;

\begin{align*}
\mathbf{B} &= \mathbf{S}^{-1} \mathbf{A}\mathbf{S} 
\\
&= \begin{bmatrix} 1 & 0 & \cdots & 0 & \cdots & 0 & \cdots & 0 \\ 0 & 1 & \cdots & 0 & \cdots & 0 & \cdots & 0 \\ \vdots & \vdots & \ddots & \vdots & \ddots & \vdots & \ddots & \vdots \\ 0 & 0 & \cdots & s_{ii}^- & \cdots & s_{ij}^- & \cdots & 0 \\ \vdots & \vdots & \ddots & \vdots & \ddots & \vdots & \ddots & \vdots \\ 0 & 0 & \cdots & s_{ji}^- & \cdots & s_{jj}^- & \cdots & 0 \\ \vdots & \vdots & \ddots & \vdots & \ddots & \vdots & \ddots & \vdots \\ 0 & 0 & \cdots & 0 & \cdots & 0 & \cdots & 1 \end{bmatrix}\begin{bmatrix} a_{11} & a_{12} & \cdots & a_{1i} & \cdots & a_{1j} & \cdots & a_{1n} \\ a_{21} & a_{22} & \cdots & a_{2i} & \cdots & a_{2j} & \cdots & a_{2n} \\ \vdots & \vdots & \ddots & \vdots & \ddots & \vdots & \ddots & \vdots \\ a_{i1} & a_{i2} & \cdots & a_{ii} & \cdots & a_{ij} & \cdots & a_{in} \\ \vdots & \vdots & \ddots & \vdots & \ddots & \vdots & \ddots & \vdots \\ a_{j1} & a_{j2} & \cdots & a_{ji} & \cdots & a_{jj} & \cdots & a_{jn} \\ \vdots & \vdots & \ddots & \vdots & \ddots & \vdots & \ddots & \vdots \\ a_{n1} & a_{n2} & \cdots & a_{ni} & \cdots & a_{nj} & \cdots &a_{nn}\end{bmatrix}\mathbf{S}
\\
&= \begin{bmatrix} a_{11} & a_{12} & \cdots & a_{1i} & \cdots & a_{1j} & \cdots & a_{1n}  \\ a_{21} & a_{22} & \cdots & a_{2i} & \cdots & a_{2j} & \cdots & a_{2n} \\ \vdots & \vdots & \ddots & \vdots & \ddots & \vdots & \ddots & \vdots \\ s_{ii}^- a_{i1} + s_{ij}^- a_{j1} & s_{ii}^- a_{i2} + s_{ij}^- a_{j2} & \cdots & s_{ii}^- a_{ii} + s_{ij}^- a_{ji} & \cdots & s_{ii}^- a_{ij} + s_{ij}^- a_{jj} & \cdots & s_{ii}^- a_{in} + s_{ij}^- a_{jn} \\ \vdots & \vdots & \ddots & \vdots & \ddots & \vdots & \ddots & \vdots \\ s_{ji}^- a_{i1} + s_{jj}^- a_{j1} & s_{ji}^- a_{i2} + s_{jj}^- a_{j2} & \cdots & s_{ji}^- a_{ii} + s_{jj}^- a_{ji} & \cdots & s_{ji}^- a_{ij} + s_{jj}^- a_{jj} & \cdots & s_{ji}^- a_{in} + s_{jj}^- a_{jn} \\ \vdots & \vdots & \ddots & \vdots & \ddots & \vdots & \ddots & \vdots \\ a_{n1} & a_{n2} & \cdots & a_{ni} & \cdots & a_{nj} & \cdots &a_{nn}\end{bmatrix} \mathbf{S}
\\ 
&= \begin{bmatrix} a_{11}^- & a_{12}^- & \cdots & a_{1i}^- & \cdots & a_{1j}^- & \cdots & a_{1n}^- \\ a_{21}^- & a_{22}^- & \cdots & a_{2i}^- & \cdots & a_{2j}^- & \cdots & a_{2n}^- \\ \vdots & \vdots & \ddots & \vdots & \ddots & \vdots & \ddots & \vdots \\ a_{i1}^- & a_{i2}^- & \cdots & a_{ii}^- & \cdots & a_{ij}^- & \cdots & a_{in}^- \\ \vdots & \vdots & \ddots & \vdots & \ddots & \vdots & \ddots & \vdots \\ a_{j1}^- & a_{j2}^- & \cdots & a_{ji}^- & \cdots & a_{jj}^- & \cdots & a_{jn}^- \\ \vdots & \vdots & \ddots & \vdots & \ddots & \vdots & \ddots & \vdots \\ a_{n1}^- & a_{n2}^- & \cdots & a_{ni}^- & \cdots & a_{nj}^- & \cdots &a_{nn}\end{bmatrix}\begin{bmatrix} 1 & 0 & \cdots & 0 & \cdots & 0 & \cdots & 0 \\ 0 & 1 & \cdots & 0 & \cdots & 0 & \cdots & 0 \\ \vdots & \vdots & \ddots & \vdots & \ddots & \vdots & \ddots & \vdots \\ 0 & 0 & \cdots & s_{ii} & \cdots & s_{ij} & \cdots & 0 \\ \vdots & \vdots & \ddots & \vdots & \ddots & \vdots & \ddots & \vdots \\ 0 & 0 & \cdots & s_{ji} & \cdots & s_{jj} & \cdots & 0 \\ \vdots & \vdots & \ddots & \vdots & \ddots & \vdots & \ddots & \vdots \\ 0 & 0 & \cdots & 0 & \cdots & 0 & \cdots & 1 \end{bmatrix}
\\
&= \begin{bmatrix} a_{11}^- & a_{12}^- & \cdots & a_{1i}^- s_{ii} + a_{1j}^- s_{ji} & \cdots & a_{1i}^- s_{ij} + a_{1j}^- s_{jj} & \cdots & a_{1n}^- \\  a_{21}^- & a_{22}^- & \cdots & a_{2i}^- s_{ii} + a_{2j}^- s_{ji} & \cdots & a_{2i}^- s_{ij} + a_{2j}^- s_{jj} & \cdots & a_{2n}^- \\ \vdots & \vdots & \ddots & \vdots & \ddots & \vdots & \ddots & \vdots \\  a_{i1}^- & a_{i2}^- & \cdots & a_{ii}^- s_{ii} + a_{ij}^- s_{ji} & \cdots & a_{ii}^- s_{ij} + a_{ij}^- s_{jj} & \cdots & a_{in}^- \\ \vdots & \vdots & \ddots & \vdots & \ddots & \vdots & \ddots & \vdots \\ a_{j1}^- & a_{j2}^- & \cdots & a_{ji}^- s_{ii} + a_{jj}^- s_{ji} & \cdots & a_{ji}^- s_{ij} + a_{jj}^- s_{jj} & \cdots & a_{jn}^- \\ \vdots & \vdots & \ddots & \vdots & \ddots & \vdots & \ddots & \vdots \\ a_{n1}^- & a_{n2}^- & \cdots & a_{ni}^- s_{ii} + a_{nj}^- s_{ji} & \cdots & a_{ni}^- s_{ij} + a_{nj}^- s_{jj} & \cdots & a_{nn}^-\end{bmatrix} \,,
\end{align*}

which gives us the matrix elements of the similarity matrix $\mathbf{B}$

\begin{align}
b_{k\ell} = \begin{cases} s_{ii}^- a_{ii} s_{ii} + s_{ij}^- a_{ji} s_{ii} + s_{ii}^- a_{ij} s_{ji} + s_{ij}^- a_{jj} s_{ji} & \text{for $k=\ell=i$} \\ s_{ii}^- a_{ii} s_{ij} + s_{ij}^- a_{ji} s_{ij} + s_{ii}^- a_{ij} s_{jj} + s_{ij}^- a_{jj} s_{jj} & \text{for $k=i$ and $\ell = j$} \\ s_{ji}^- a_{ii} s_{ii} + s_{jj}^- a_{ji} s_{ii} + s_{ji}^- a_{ij} s_{ji} + s_{jj}^- a_{jj} s_{ji} & \text{for $k=j$ and $\ell=i$} \\ s_{ji}^- a_{ii} s_{ij} + s_{jj}^- a_{ji} s_{ij} + s_{ji}^- a_{ij} s_{jj} + s_{jj}^- a_{jj} s_{jj} & \text{for $k=\ell=j$} \\ s_{ki}^- a_{i \ell} + s_{kj}^- a_{j\ell} & \text{for $k\in\kr{i,j}$ and $\ell\in\mathbb{N}_1^n / \kr{i,j}$} \\ a_{ki} s_{i\ell} + a_{kj}s_{j\ell} & \text{for $\ell\in\kr{i,j}$ and $k\in\mathbb{N}_1^n / \kr{i,j}$} \\ a_{k\ell} & \text{elsewhere.}\end{cases}
\label{eq_2}
\end{align}

What we want our matrix similarity transformation to do is to set $b_{ij} = b_{ji} = 0$, but we also need to ensure that the basis matrix $\mathbf{S}$ for the transformation is invertible. Hence we have the following three equation we want to solve 

\begin{align}
s_{ii} s_{jj} - s_{ij} s_{ji} &= 1 
\label{eq_3}\\
a_{ji} {s_{ij}}^2-\p{a_{ii}-a_{jj}}s_{jj}s_{ij} - a_{ij}{s_{jj}}^2 &= 0 \label{eq_4}\\
a_{ij} {s_{ji}}^2 - \p{a_{jj}-a_{ii}}s_{ii}s_{ji} - a_{ji}{s_{ii}}^2 &= 0 \,. \label{eq_5}
\end{align} 

If the matrix $\mathbf{A}$ is indeed diagonalizable then the above three equations above solvable. We further observer that we have four unknowns in the three equations, which means that we can choose one of them as we want, I suggest $s_{ii}=1$. I will not show the solution here, but it involves in principle in two second order equations $a x^2 + b x + c = 0$ which has a well known analytical solution $x = \frac{-b\pm\sqrt{b^2 - 4ac}}{2a}$. Be aware of that the coefficient $a$ may be zero which would make it a first order equations to solve instead. If the coefficient $b=0$ in addition to $a=0$, then the matrix $\mathbf{A}$ is not diagonalizable. \linebreak

There are on last thing we need to consider regarding the diagonalizing algorithm, and that is that the matrix elements $b_{ix}$, $b_{jx}$, $b_{xi}$ and $b_{xj}$ for all $x\in\mathbb{N}_1^n / \kr{i,j}$ may change from zero to a value if we are not careful. Which may undermine the work we try to do when we diagonalize the matrix. What we see in \eqref{eq_2} is if we lock $k$ to $i$ and let $\ell$ go through all possibilities $\ell\in\mathbb{N}_1^n / \kr{k}$ exactly once, we ensure that the values we have zeroed out for $k=i$ does not change when we do successive transformation with the same $k=i$;

\begin{align*}
&\begin{bmatrix} a_{11} & a_{12} & \cdots & a_{1n} \\ a_{21} & a_{22} & \cdots & a_{2n} \\ \vdots & \vdots & \ddots & \vdots \\ a_{n1} & a_{n2} & \cdots & a_{nn} \end{bmatrix} 
\sim \begin{bmatrix} b_{11}^{(1)} & b_{12}^{(1)} & \cdots & 0 \\ b_{21}^{(1)} & b_{22}^{(1)} & \cdots & b_{2n}^{(1)} \\ \vdots & \vdots & \ddots & \vdots \\ 0 & b_{n2}^{(1)} & \cdots & b_{nn}^{(1)} \end{bmatrix}
\sim \begin{bmatrix} b_{11}^{(n-1)} & 0 & \cdots & 0 \\ 0 & b_{22}^{(n-1)} & \cdots & b_{2n}^{(n-1)} \\ \vdots & \vdots & \ddots & \vdots \\ 0 & b_{n2}^{(n-1)} & \cdots & b_{nn}^{(n-1)} \end{bmatrix} \to \begin{bmatrix} a_{11} & 0 & \cdots & 0 \\ 0 & a_{22} & \cdots & a_{2n} \\ \vdots & \vdots & \ddots & \vdots \\ 0 & a_{n2} & \cdots & a_{nn} \end{bmatrix} 
\end{align*} 

Note that we have now transformed such that $a_{ki} = a_{ii}\delta_{ik} = a_{ki}$, where $\delta_{ik}$ is the Kronecker delta function. Now we change indices to $i^\prime\neq i$ and let $j\in\mathbb{N}_1^n / \kr{i,i'}$, because we are done with the row and column $i$.  What we now realize in \eqref{eq_2} is that the elements of matrix $\mathbb{B}$ in the row or column $i$ is given by  $b_{i\ell}= a_{ii^\prime} s_{i^\prime \ell} + a_{ij} s_{j\ell} = 0$ because $k\not\in\kr{i^\prime, j}$ and $i^\prime \neq i \neq j$ which means that $a_{ii^\prime}=0$ and $a_{ij}=0$, and $b_{ki} = s_{ki^\prime}^- a_{i^\prime i} + s_{kj}^- a_{ji} = 0$ because $\ell\not\in\kr{i^\prime, j}$ and $i^\prime \neq i \neq j$ which means that $a_{i^\prime i} = 0$ and $a_{ji} = 0$. In other words the matrix elements in row or column $i$ that was zeroed out does not change when we continue to zero out matrix elements on row and column $i^\prime$, which means that we can continue the iteration until the matrix $\mathbf{A}$ is completely diagonalized;

\begin{align*}
\begin{bmatrix} a_{11} & 0 & 0 & \cdots & 0 \\ 0 & a_{22} & a_{23} & \cdots & a_{2n} \\ 0 & a_{32} & a_{33} & \cdots & a_{3n} \\ \vdots & \vdots & \vdots & \ddots & \vdots \\ 0 & a_{n2} & a_{n3} & \cdots & a_{nn} \end{bmatrix}
&\sim \begin{bmatrix} b_{11}^{(1)} & 0 & 0 & \cdots & 0 \\ 0 & b_{22}^{(1)} & b_{23}^{(1)} & \cdots & 0 \\ 0 & b_{32}^{(1)} & b_{33}^{(1)} & \cdots & b_{3n}^{(1)} \\ \vdots & \vdots & \vdots & \ddots & \vdots \\ 0 & 0 & b_{n3}^{(1)} & \cdots & b_{nn}^{(1)} \end{bmatrix}
\sim \begin{bmatrix} b_{11}^{(n-2)} & 0 & 0 & \cdots & 0 \\ 0 & b_{22}^{(n-2)} & 0 & \cdots & 0 \\ 0 & 0 & b_{33}^{(n-2)} & \cdots & b_{3n}^{(n-2)} \\ \vdots & \vdots & \vdots & \ddots & \vdots \\ 0 & 0 & b_{n3}^{(n-2)} & \cdots & b_{nn}^{(n-2)} \end{bmatrix}
\\
&\sim \begin{bmatrix} \lambda_1 & 0 & 0 & \cdots & 0 \\ 0 & \lambda_2 & 0 & \cdots & 0 \\ 0 & 0 & \lambda_3 & \cdots & 0 \\ \vdots & \vdots & \vdots & \ddots & \vdots \\ 0 & 0 & 0 & \cdots & \lambda_n \end{bmatrix} \,.
\end{align*}

\subsection{Jacobi's eigenvalue algorithm}

The Jacobi's eigenvalue algorithm is a special case of diagonalize by matrix similarity algorithm discussed in section \ref{sec_Diagonalize}, where we use a two dimensional rotation matrix in $n$ dimensional space

\begin{align}
\mathbf{R}_{i,j,\theta} = \begin{bmatrix} 1 & 0 & \cdots & 0 & \cdots & 0 & \cdots & 0 \\ 0 & 1 & \cdots & 0 & \cdots & 0 & \cdots & 0 \\ \vdots & \vdots & \ddots & \vdots & \ddots & \vdots & \ddots & \vdots \\ 0 & 0 & \cdots & r_{ii} = \cos \theta & \cdots & r_{ij} = -\sin\theta & \cdots & 0 \\ \vdots & \vdots & \ddots & \vdots & \ddots & \vdots & \ddots & \vdots \\ 0 & 0 & \cdots & r_{ji} = \sin \theta & \cdots & r_{jj} = \cos\theta & \cdots & 0 \\ \vdots & \vdots & \ddots & \vdots & \ddots & \vdots & \ddots & \vdots \\ 0 & 0 & \cdots & 0 & \cdots & 0 & \cdots & 1 \end{bmatrix} \,,
\label{eq_6}
\end{align} 

which satisfies the basis matrix $\mathbf{S}$ in \eqref{eq_1}. The rotation matrix can easily be shown to be a orthogonal matrix by realizing $\mathbf{R}_{-\theta} = {\mathbf{R}_{\theta}}^{-1} = {\mathbf{R}_{\theta}}^{\mathrm{T}}$, because when we rotate an angle $\theta$, we need to rotate an angle $-\theta$ to get back (or you could realize this by a more cumbersome way by actually do the matrix inversion). \linebreak

By doing the matrix similarity transformation $\mathbf{B} = \mathbf{R}_{i,j,-\theta}\mathbf{A}\mathbf{R}_{i,j,\theta}$, we get the following similar matrix elements according to \eqref{eq_2}

\begin{align*}
b_{k\ell} = \begin{cases} a_{ii} \cos^2\theta  + a_{jj}\sin^2 \theta + \p{a_{ij} + a_{ji}}\sin\theta\cos\theta & \text{for $k=\ell=i$} \\ a_{ij}\cos^2\theta - a_{ji}\sin^2\theta + \p{a_{jj}-a_{ii}}\sin\theta\cos\theta & \text{for $k=i$ and $\ell = j$} \\ a_{ji}\cos^2\theta - a_{ij}\sin^2\theta + \p{a_{jj}-a_{ii}}\sin\theta\cos\theta & \text{for $k=j$ and $\ell=i$} \\ a_{jj} \cos^2\theta  + a_{ii}\sin^2 \theta - \p{a_{ij} + a_{ji}}\sin\theta\cos\theta & \text{for $k=\ell=j$} \\ a_{i \ell} \cos\theta +  a_{j\ell} \sin\theta & \text{for $k=i$ and $\ell\in\mathbb{N}_1^n / \kr{i,j}$} \\ a_{j\ell} \cos\theta - a_{i\ell}\sin\theta & \text{for $k=j$ and $\ell\in\mathbb{N}_1^n / \kr{i,j}$} \\ a_{ki} \cos\theta + a_{kj}\sin\theta & \text{for $\ell=i$ and $k\in\mathbb{N}_1^n / \kr{i,j}$} \\ a_{kj}\cos\theta - a_{ki}\sin\theta & \text{for $\ell=j$ and $k\in\mathbb{N}_1^n / \kr{i,j}$} \\ a_{k\ell} & \text{elsewhere.}\end{cases}
\end{align*}

\begin{align*}
b_{k\ell} = \begin{cases} a_{ii} \cos^2\theta  + a_{jj}\sin^2 \theta + \p{a_{ij} + a_{ji}}\sin\theta\cos\theta & \text{for $k=\ell=i$} \\ a_{ij}\cos^2\theta - a_{ji}\sin^2\theta + \p{a_{jj}-a_{ii}}\sin\theta\cos\theta & \text{for $k=i$ and $\ell = j$} \\ a_{ji}\cos^2\theta - a_{ij}\sin^2\theta + \p{a_{jj}-a_{ii}}\sin\theta\cos\theta & \text{for $k=j$ and $\ell=i$} \\ a_{jj} \cos^2\theta  + a_{ii}\sin^2 \theta - \p{a_{ij} + a_{ji}}\sin\theta\cos\theta & \text{for $k=\ell=j$} \\ a_{i \ell} \cos\theta +  a_{j\ell} \sin\theta & \text{for $k=i$ and $\ell\in\mathbb{N}_1^n / \kr{i,j}$} \\ a_{j\ell} \cos\theta - a_{i\ell}\sin\theta & \text{for $k=j$ and $\ell\in\mathbb{N}_1^n / \kr{i,j}$} \\ a_{ki} \cos\theta + a_{kj}\sin\theta & \text{for $\ell=i$ and $k\in\mathbb{N}_1^n / \kr{i,j}$} \\ a_{kj}\cos\theta - a_{ki}\sin\theta & \text{for $\ell=j$ and $k\in\mathbb{N}_1^n / \kr{i,j}$} \\ a_{k\ell} & \text{elsewhere.}\end{cases}
\end{align*}


The strategy is to set $b_{ij}$ and $b_{ji}$ to zero, which enables us by iteration to zero out every element in the matrix except the diagonal elements (as described in section \ref{sec_Diagonalize}). Unfortunately we have only one variable, namely $\theta$, which makes it impossible to diagonalize a general matrix $\mathbf{A}$ by similarity. However when $\mathbf{A}$ is a symmetric matrix, $a_{ij}=a_{ji}$, which gives us only one equation to solve to set both $b_{ij}$ and $b_{ji}$ to zero. For a symmetric matrix $\mathbf{A}$ we have the following elements of the similarity matrix $\mathbf{B}$

\begin{align}
b_{k\ell} = \begin{cases} a_{ii} \cos^2\theta  + a_{jj}\sin^2 \theta + 2 a_{ij}\sin\theta\cos\theta & \text{for $k=\ell=i$} \\ a_{ij}\p{\cos^2\theta - \sin^2\theta} + \p{a_{jj}-a_{ii}}\sin\theta\cos\theta & \text{for $k=i$ and $\ell = j$, or $k=j$ and $\ell = i$} \\ a_{jj} \cos^2\theta  + a_{ii}\sin^2 \theta - 2 a_{ij}\sin\theta\cos\theta & \text{for $k=\ell=j$} \\ a_{i \ell} \cos\theta +  a_{j\ell} \sin\theta & \text{for $k=i$ and $\ell\in\mathbb{N}_1^n / \kr{i,j}$} \\ a_{j\ell} \cos\theta - a_{i\ell}\sin\theta & \text{for $k=j$ and $\ell\in\mathbb{N}_1^n / \kr{i,j}$} \\ a_{ki} \cos\theta + a_{kj}\sin\theta & \text{for $\ell=i$ and $k\in\mathbb{N}_1^n / \kr{i,j}$} \\ a_{kj}\cos\theta - a_{ki}\sin\theta & \text{for $\ell=j$ and $k\in\mathbb{N}_1^n / \kr{i,j}$} \\ a_{k\ell} & \text{elsewhere.}\end{cases}
\label{eq_7}
\end{align}

Now we are able to solve the following equation

\begin{align*}
b_{ij}=b_{ji} = a_{ij}\p{\cos^2\theta - \sin^2\theta}+\p{a_{jj}-a_{ii}}\sin\theta\cos\theta = 0 \,,
\end{align*}

and by using the trigonometric relation $\cos^2\theta+\sin^2\theta = 1$ we can rewrite to

\begin{align*}
\cos^4\theta -\cos^2\theta + \frac{{a_{ij}}^2}{\p{a_{ii}-a_{jj}}^2 + 4 {a_{ij}}^2} = 1\,.
\end{align*}

By solving this as a second order equation with regard to $\cos^2\theta$ and using the trigonometric relation $\cos^2\theta+\sin^2\theta = 1$ again, we get

\begin{align}
\theta = \arccos\sqrt{\frac{1}{2}\p{1+\sqrt{\frac{\p{a_{ii}-a_{jj}}^2}{\p{a_{ii}-a_{jj}}^2+4{a_{ij}}^2}}}} = \arcsin\sqrt{\frac{1}{2}\p{1-\sqrt{\frac{\p{a_{ii}-a_{jj}}^2}{\p{a_{ii}-a_{jj}}^2+4{a_{ij}}^2}}}} \,.
\end{align}

We can diagonalize the matrix $\mathbf{A}$ by doing matrix similarity transformation iteratively as described in section \ref{sec_Diagonalize}. However we can optimize Jacobi's eigenvalue algorithm by using  the Frobenius norm.

\section{Numerical implementation}

We can optimize the calculation of diagonal elements in \eqref{eq_7} by rewriting them by using the trigonometrical relation $\cos^2\theta+\sin^2\theta = 1$;

\begin{align}
b_{ii} &= a_{ii}\cos^2\theta + a_{jj}\sin^2\theta + 2 a_{ij}\sin\theta\cos\theta 
= a_{ii}\p{1-\sin^2\theta} + a_{jj}\sin^2\theta + 2 a_{ij}\sin\theta\cos\theta
\nonumber\\
&= a_{ii} + \kl{\p{a_{jj}-a_{ii}}\sin^2\theta + 2 a_{ij}\sin\theta\cos\theta}
\\
b_{jj} &= a_{jj}\cos^2\theta + a_{ii}\sin^2\theta - 2a_{ij}\sin\theta\cos\theta
= a_{jj}\p{1-\sin^2\theta} + a_{ii}\sin^2\theta - 2a_{ij}\sin\theta\cos\theta
\nonumber\\
&= a_{jj} - \kl{\p{a_{jj}-a_{ii}}\sin^2\theta + 2a_{ij}\sin\theta\cos\theta}
\end{align}

Note that it's better to use $\sin^2\theta$ instead of $\cos^2\theta$, because when we look at 

\begin{align*}
b_{ii} &= a_{jj} + \kl{\p{a_{ii}-a_{jj}}\cos^2\theta + 2a_{ij}\cos\theta\sin\theta}
\\
b_{jj} &= a_{ii} - \kl{\p{a_{ii}-a_{jj}}\cos^2\theta + 2a_{ij}\cos\theta\sin\theta} \,,
\end{align*}

we see that $a_{jj}$ corresponds to $b_{ii}$ and $a_{ii}$ corresponds to $b_{jj}$, which means that we need to backup either $a_{ii}$ or $a_{jj}$ before we do the calculation, so we don't overwrite the value before we have used in the calculation. However for the $\sin^2\theta$ expressions we have that $a_{ii}$ corresponds to $b_{ii}$ and $a_{jj}$ corresponds to $b_{jj}$, which means that we don't need to backup $a_{ii}$ or $a_{jj}$, and we can  use the program operator \texttt{+=} directly on the diagonal elements. 


\section{Attachments}

The files produced in working with this project can be found at \href{https://github.com/Eimund/UiO/tree/master/FYS4150/Project\%202}{https://github.com/Eimund/UiO/tree/master/FYS4150/Project\%202} \linebreak

The source files developed are


\section{Resources}

\begin{enumerate}
\item{\href{http://qt-project.org/downloads}{QT Creator 5.3.1 with C11}}
\item{\href{http://arma.sourceforge.net/}{Armadillo}}
\item{\href{https://www.eclipse.org/downloads/}{Eclipse Standard/SDK  - Version: Luna Release (4.4.0) with PyDev for Python}}
\item{\href{http://www.ubuntu.com/download/desktop}{Ubuntu 14.04.1 LTS}}
\item{\href{http://shop.lenovo.com/no/en/laptops/thinkpad/w-series/w540/#tab-reseller}{ThinkPad W540 P/N: 20BG0042MN with 32 GB RAM}}
\end{enumerate}

\begin{thebibliography}{9}
\bibitem{project1}{\href{mailto:morten.hjorth-jensen@fys.uio.no}{Morten Hjorth-Jensen}, \href{http://www.uio.no/studier/emner/matnat/fys/FYS3150/h14/undervisningsmateriale/projects/project-2-deadline-september-22/project2_2014.pdf}{\emph{FYS4130 - Project 2}} - \emph{Schr\"{o}dinger's equation for two electrons in a three-dimensional harmonic oscillator well}, \href{http://www.uio.no}{University of Oslo}, 2014} 
\bibitem{lecture}{\href{mailto:morten.hjorth-jensen@fys.uio.no}{Morten Hjorth-Jensen}, \href{http://www.uio.no/studier/emner/matnat/fys/FYS3150/h14/undervisningsmateriale/Lecture\%20Notes/lecture2014.pdf}{\emph{Computational Physics - Lecture Notes Fall 2014}}, \href{http://www.uio.no}{University of Oslo}, 2014}  
\bibitem{Eigenvaule}{\href{http://en.wikipedia.org/wiki/Eigenvalues\_and\_eigenvectors}{http://en.wikipedia.org/wiki/Eigenvalues\_and\_eigenvectors}}
\bibitem{JacobiMethod}{\href{http://en.wikipedia.org/wiki/Jacobi\_eigenvalue\_algorithm}{http://en.wikipedia.org/wiki/Jacobi\_eigenvalue\_algorithm}}
\bibitem{MatrixSimularity}{\href{http://en.wikipedia.org/wiki/Matrix\_similarity}{http://en.wikipedia.org/wiki/Matrix\_similarity}}
\bibitem{RotationMatrix}{\href{http://en.wikipedia.org/wiki/Rotation\_matrix}{http://en.wikipedia.org/wiki/Rotation\_matrix}}
\bibitem{OrthogonalMatrix}{\href{http://en.wikipedia.org/wiki/Orthogonal\_matrix}{http://en.wikipedia.org/wiki/Orthogonal\_matrix}}
\bibitem{Frobenius}{\href{http://en.wikipedia.org/wiki/Matrix\_norm\#Frobenius\_norm}{http://en.wikipedia.org/wiki/Matrix\_norm\#Frobenius\_norm}}
\end{thebibliography}
\end{flushleft}
\end{document}
